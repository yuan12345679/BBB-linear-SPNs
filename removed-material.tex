
\subsection{Linear SPNs}
We are interested in understanding the security of both linear and non-linear SPNs. We now define what we mean by these terms.\\

\noindent \textbf{Definition 1}.A \emph{keyed permutation $\pi$ : $\mathbb{F}^{w} \times \mathbb{F}^{w} \rightarrow \mathbb{F}^{w}$ is} \textbf{linear} \emph{if}

$$
\begin{aligned}
\pi(k,x) = (\mathnormal{T}_{k} \cdot k) + (\mathnormal{T}_{x} \cdot x) + \Delta,
\end{aligned}
$$

\emph{where $\mathnormal{T}_{k}, \mathnormal{T}_{x} \in \mathbb{F}^{w \times w}$ are linear transformations, $\mathnormal{T}_{x}$ is invertible, and $\Delta \in \mathbb{F}^{w}$. An SPN is} \textbf{linear} \emph{ if all its round permutations $\{\pi_{i}\}_{i=0}^{r}$ are linear.}

If $\pi(k,x) = (\mathnormal{T}_{k} \cdot k) + (\mathnormal{T}_{x} \cdot x) + \Delta$ is linear, then we may write

$$
\begin{aligned}
\pi(k,x) = \mathnormal{T} \cdot ((\mathnormal{T}^{-1} \mathnormal{T}_{k} \cdot k) + (\mathnormal{T}^{-1} \cdot \Delta) + x);
\end{aligned}
$$

thus, by setting $k^{\prime} = (\mathnormal{T}^{-1} \mathnormal{T}_{k} \cdot k) + (\mathnormal{T}^{-1} \cdot \Delta)$  we can express $\pi$ as

\begin{equation}
\begin{aligned}
\pi(k^{\prime},x) = \mathnormal{T} \cdot (k^{\prime} + x).
\end{aligned}
\end{equation}

In other words, if an SPN is linear, then we may assume (by redefining the distribution $\mathcal{K}$  on keys appropriately) that each of its permutations $\pi_{i}$ takes the
form (1). This matches what is often done in practice (e.g., in AES, Serpent,PRESENT, etc.), where round permutations are computed by first performing a key-mixing step followed by an invertible linear transformation. Since the linear transformation and key mixing commute, and the adversary can compute $\mathnormal{T}$ and $\mathnormal{T}^{-1}$ on its own (as  $\mathnormal{T}$ is public), we may further assume without loss of generality that the first and last permutations involve only a key-mixing step. In other words, we may set$\pi_{i}(k_{i},x) = k_{i} + x$ for $i \in \{0,r\}$.





\subsection{Tweakable linear Substitution-Permutation Networks}
\textsc{Tweakable Permutations.} For an integer $m \geq 1$, the set of all permutations
on $\{0,1\}^{m}$ will be denoted $\operatorname{Perm}(m)$. A tweakable permutation with tweak space $\mathcal{T}$ and message space $\mathcal{X}$ is a mapping $\widetilde{P} :\mathcal{T} \times \mathcal{X} \rightarrow \mathcal{X}$ such that, for any tweak $t \in \mathcal{T}$,
$$
x \mapsto \widetilde{P}(t,x)
$$
\noindent is a permutation of $\mathcal{X}$. The set of all tweakable permutations with tweak space $\mathcal{T}$ and message space $\{0,1\}^{m}$ will be denoted $\widetilde{\operatorname{Perm}}(\mathcal{T}, m)$.
A keyed tweakable permutation with key space $\mathcal{K}$, tweak space $\mathcal{T}$ and message space $\mathcal{X}$ is a mapping $T : \mathcal{K} \times \mathcal{T} \times \mathcal{X} \rightarrow \mathcal{X}$ such that, for any key $k \in \mathcal{K}$,
$$
(t,x) \mapsto T(k,t,x)
$$
\noindent is a tweakable permutation with tweak space $\mathcal{T}$ and message space $\mathcal{X}$.\\

\textsc{Tweakable linear SPNs.} For fixed parameters $w$ and $n$, let $\mathcal{K}$ and $\mathcal{T}$ be two $w n$-bit blocks, we denote $f(k, t) = k \oplus t$ with $k \in \mathcal{K}$ and $t \in \mathcal{T}$, then let
$$
T: f(k, t) \times \{0,1\}^{w n} \rightarrow \{0,1\}^{w n}
$$
be a keyed tweakable linear permutation with key space $\mathcal{K}$, tweak space $\mathcal{T}$ and message space $\{0,1\}^{w n}$.We simply write $T(k,t,x)$ as $T_{f(k, t)}(x)$For a fixed number of rounds $r$, an $r$-round linear substitution-permutation network (SPN) based on $T$, denoted $\operatorname{SP}^T$, takes as input a set of $n$-bit permutations $\mathcal{S} = (S_1, \ldots, S_r)$, and defines a keyed tweakable  linear permutation $\operatorname{SP}^{T}[S]$ operating on $w n$-bit blocks with key space $\mathcal{K}^{r+1}$ and tweak space $\mathcal{T}$: on input $x \in \{0,1\}^{w n}$, key $\mathbf{k} = (k_0, k_1, \ldots, k_r) \in \mathcal{K}^{r+1}$ and tweak $t \in \mathcal{T}$, the output of $\operatorname{SP}^T[\mathcal{S}]$ is computed as follows.

\begin{itemize}
	\item[--]
	$y \leftarrow x$
	\item[--]
	For $i \leftarrow 1$ to $\mathnormal{r}$ do:
	\begin{itemize}
		\item[1.]
		$y \leftarrow T_{f(k_{i-1}, t)}(y)$.
		\item[2.]
		Break $y = y_1 \| \cdots \| y_w$ into $n$-bit blocks.
		\item[3.]
		$y \leftarrow S_i(y_1) \| \cdots \| S_i(y_w)$.
	\end{itemize}
	\item[--]
	$y \leftarrow T_{f(k_{r}, t)}(y)$.
\end{itemize}


