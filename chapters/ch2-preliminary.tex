

\section{Preliminaries}
\label{section:preliminary}


\subsection{(Tweakable) Substitution-Permutation Networks}
A \emph{substitution-permutation network }(SPN) defines a keyed permutation via repeated invocation of two transformations: blockwise computation of a public, cryptographic permutation called an ``S-box,'' and application of a keyed, non-cryptographic permutation. Formally, an r-round SPN taking inputs of length $w n$ where $w \in \mathcal{N}$ is the width of the network, is defined by $\mathnormal{r}+1$ keyed permutations $\left\{\pi_{i}: K_{i} \times\{0,1\}^{w n} \rightarrow\{0,1\}^{w n}\right\}_{i=0}^{r}$, a distribution $\mathcal{K}$ over $\mathcal{K}_{0} \times \dots \times \mathcal{K}_{r}$, and a permutation $\mathcal{S}:\{0,1\}^{n} \rightarrow \{0,1\}^{n}$. Given round keys $(k_{0},\dots, k_{r}) \in \mathcal{K}_{0} \times \dots \times \mathcal{K}_{r}$ and an input $x \in \{0,1\}^{w n}$, the output of the SPN is computed as follows

\begin{itemize}
  \item[--]
  Let $x_{1} := \pi_{0}(k_{0},x)$.
  \item[--]
  For $i = 1$ to $\mathnormal{r}$ do:
  \begin{itemize}
    \item[1.]
    $y_{i} := \bar{S}(x_{i})$, where $\bar{S}(x[1]\|\cdots\| x[w]) \stackrel{\text { def }}{=} S(x[1])\|\cdots\| S(x[w])$.
    \item[2.]
    $x_{i + 1} := \pi_{i}(k_{i},y_{i})$.
  \end{itemize}
  \item[--]
  The output is $x_{r+1}$.
\end{itemize}

If $\mathcal{S}$ is efficiently invertible and each $\pi_{i}$ is efficiently invertible (given the appropriate key), then the above process is reversible given the round keys $k_{0}, \dots ,k_{r}$. In our definition of an SPN, we apply a fixed permutation $\mathcal{S}$ to all $\mathnormal{w}$ blocks of the intermediate state in each round. More generally, one could consider using $\mathnormal{w}$ different functions $S_{1}, \dots ,S_{w}$ in each round, or even different S-boxes in different rounds. Our positive results hold even when a single permutation $\mathcal{S}$ is used, and our negative result holds even if multiple permutations are used.\\


\subsection{Linear SPNs}
We are interested in understanding the security of both linear and non-linear SPNs. We now define what we mean by these terms.\\

\noindent \textbf{Definition 1}.A \emph{keyed permutation $\pi$ : $\mathbb{F}^{w} \times \mathbb{F}^{w} \rightarrow \mathbb{F}^{w}$ is} \textbf{linear} \emph{if}

$$
\begin{aligned}
\pi(k,x) = (\mathnormal{T}_{k} \cdot k) + (\mathnormal{T}_{x} \cdot x) + \Delta,
\end{aligned}
$$

\emph{where $\mathnormal{T}_{k}, \mathnormal{T}_{x} \in \mathbb{F}^{w \times w}$ are linear transformations, $\mathnormal{T}_{x}$ is invertible, and $\Delta \in \mathbb{F}^{w}$. An SPN is} \textbf{linear} \emph{ if all its round permutations $\{\pi_{i}\}_{i=0}^{r}$ are linear.}

If $\pi(k,x) = (\mathnormal{T}_{k} \cdot k) + (\mathnormal{T}_{x} \cdot x) + \Delta$ is linear, then we may write

$$
\begin{aligned}
\pi(k,x) = \mathnormal{T} \cdot ((\mathnormal{T}^{-1} \mathnormal{T}_{k} \cdot k) + (\mathnormal{T}^{-1} \cdot \Delta) + x);
\end{aligned}
$$

thus, by setting $k^{\prime} = (\mathnormal{T}^{-1} \mathnormal{T}_{k} \cdot k) + (\mathnormal{T}^{-1} \cdot \Delta)$  we can express $\pi$ as

\begin{equation}
\begin{aligned}
\pi(k^{\prime},x) = \mathnormal{T} \cdot (k^{\prime} + x).
\end{aligned}
\end{equation}

In other words, if an SPN is linear, then we may assume (by redefining the distribution $\mathcal{K}$  on keys appropriately) that each of its permutations $\pi_{i}$ takes the
form (1). This matches what is often done in practice (e.g., in AES, Serpent,PRESENT, etc.), where round permutations are computed by first performing a key-mixing step followed by an invertible linear transformation. Since the linear transformation and key mixing commute, and the adversary can compute $\mathnormal{T}$ and $\mathnormal{T}^{-1}$ on its own (as  $\mathnormal{T}$ is public), we may further assume without loss of generality that the first and last permutations involve only a key-mixing step. In other words, we may set
$\pi_{i}(k_{i},x) = k_{i} + x$ for $i \in \{0,r\}$.


\subsection{Tweakable linear Substitution-Permutation Networks}
\textsc{Tweakable Permutations.} For an integer $m \geq 1$, the set of all permutations
on $\{0,1\}^{m}$ will be denoted $\operatorname{Perm}(m)$. A tweakable permutation with tweak space $\mathcal{T}$ and message space $\mathcal{X}$ is a mapping $\widetilde{P} :\mathcal{T} \times \mathcal{X} \rightarrow \mathcal{X}$ such that, for any tweak $t \in \mathcal{T}$,
$$
x \mapsto \widetilde{P}(t,x)
$$
\noindent is a permutation of $\mathcal{X}$. The set of all tweakable permutations with tweak space $\mathcal{T}$ and message space $\{0,1\}^{m}$ will be denoted $\widetilde{\operatorname{Perm}}(\mathcal{T}, m)$.
  A keyed tweakable permutation with key space $\mathcal{K}$, tweak space $\mathcal{T}$ and message space $\mathcal{X}$ is a mapping $T : \mathcal{K} \times \mathcal{T} \times \mathcal{X} \rightarrow \mathcal{X}$ such that, for any key $k \in \mathcal{K}$,
$$
(t,x) \mapsto T(k,t,x)
$$
\noindent is a tweakable permutation with tweak space $\mathcal{T}$ and message space $\mathcal{X}$.\\

\textsc{Tweakable linear SPNs.} For fixed parameters $w$ and $n$, let $\mathcal{K}$ and $\mathcal{T}$ be two $w n$-bit blocks, we denote $f(k, t) = k \oplus t$ with $k \in \mathcal{K}$ and $t \in \mathcal{T}$, then let
$$
T: f(k, t) \times \{0,1\}^{w n} \rightarrow \{0,1\}^{w n}
$$
be a keyed tweakable linear permutation with key space $\mathcal{K}$, tweak space $\mathcal{T}$ and message space $\{0,1\}^{w n}$.We simply write $T(k,t,x)$ as $T_{f(k, t)}(x)$For a fixed number of rounds $r$, an $r$-round linear substitution-permutation network (SPN) based on $T$, denoted $\operatorname{SP}^T$, takes as input a set of $n$-bit permutations $\mathcal{S} = (S_1, \ldots, S_r)$, and defines a keyed tweakable  linear permutation $\operatorname{SP}^{T}[S]$ operating on $w n$-bit blocks with key space $\mathcal{K}^{r+1}$ and tweak space $\mathcal{T}$: on input $x \in \{0,1\}^{w n}$, key $\mathbf{k} = (k_0, k_1, \ldots, k_r) \in \mathcal{K}^{r+1}$ and tweak $t \in \mathcal{T}$, the output of $\operatorname{SP}^T[\mathcal{S}]$ is computed as follows.

\begin{itemize}
  \item[--]
  $y \leftarrow x$
  \item[--]
  For $i \leftarrow 1$ to $\mathnormal{r}$ do:
  \begin{itemize}
    \item[1.]
    $y \leftarrow T_{f(k_{i-1}, t)}(y)$.
    \item[2.]
    Break $y = y_1 \| \cdots \| y_w$ into $n$-bit blocks.
    \item[3.]
   $y \leftarrow S_i(y_1) \| \cdots \| S_i(y_w)$.
  \end{itemize}
  \item[--]
  $y \leftarrow T_{f(k_{r}, t)}(y)$.
\end{itemize}


\subsection{The H-coefficient Technique}
Suppose that a distinguisher $\mathcal{D}$ makes $\mathnormal{p}$ queries to each of the S-boxes, and total $\mathnormal{q}$ queries to the construction oracles. The queries made to the construction oracle denoted $\mathcal{C}$, are recorded in a query history $$
\mathcal{Q}_{C} = (x_i,y_i)_{1 \leq i \leq q}
$$
where $\mathnormal{q}$ is the number of queries made to $\mathcal{C}$, and $(x_i,y_i)$ represents the evaluation obtained by the $\mathnormal{i}$ -th query to $\mathcal{C}$, So according to the instantiation, it implies either $\operatorname{SP}_k[\mathcal{S}](x_i) = y_i$  or $\mathcal{P}(x_i) = y_i$. For $j = 1,\cdots,r$, the queries made to $S_j$ are recorded in a query history

$$
\mathcal{Q}_{S_j} = (j, u_{j,i}, v_{j,i})_{1 \leq i \leq p}
$$
where $(j, u_{j,i}, v_{j,i})$  represents the evaluation $S_j(u_{j,i}) = v_{j,i}$ obtained by the $\mathnormal{i}$ -th query to $S_j$. Let
$$
\mathcal{Q}_{S}=\mathcal{Q}_{S_1} \cup \cdots \cup \mathcal{Q}_{S_r}
$$
Then the pair of query histories
$$
\tau = (\mathcal{Q}_{C}, \mathcal{Q}_{S})
$$
will be called the transcript of the attack: it contains all the information that  $\mathcal{D}$ has obtained at the end of the attack. In this work, we will only consider
information theoretic distinguishers. Therefore we can assume that a distinguisher is deterministic without making any redundant query, and hence the output of  $\mathcal{D}$  can be regarded as a function of $\tau$, denoted $\mathcal{D}(\tau)$ or $\mathcal{D}(\mathcal{Q}_C, \mathcal{Q}_S)$.\\

The adversarial goal is to tell apart the two worlds $\left(\operatorname{SP}_{\mathbf{k}}[\mathcal{S}],\mathcal{S}\right)$, and $\left(\mathcal{P},\mathcal{S}\right)$ by adaptively making forward and backward queries to each of the constructions and the S-boxes. Formally, $\mathcal{D}$ 's distinguishing advantage is defined by

\begin{equation}
\begin{aligned}
\operatorname{Adv}_{\mathrm{SP}}(\mathcal{D}) &=\operatorname{Pr}\left[\mathcal{P} \stackrel{s}{\leftarrow} \widehat{\operatorname{Perm}}(w n), \mathcal{S} \stackrel{\mathrm{s}}{\leftarrow} \operatorname{Perm}(n)^{r}: 1 \leftarrow \mathcal{D}^{\mathcal{S},\mathcal{P}}\right] \\
&-\operatorname{Pr}\left[\mathbf{k} \stackrel{s}{\leftarrow} \mathcal{K}^{r+1}, \mathcal{S} \stackrel{\mathrm{s}}{\leftarrow} \operatorname{Perm}(n)^{r}: 1 \leftarrow \mathcal{D}^{\mathcal{S}, \mathrm{SP}_{\mathrm{k}}[\mathcal{S}]}\right]
\end{aligned}
\end{equation}

for $p,q > 0$,we define

$$
\operatorname{Adv}_{\mathrm{SP}}(p, q)=\max _{\mathcal{D}} \operatorname{Adv}_{\mathrm{SP}}(\mathcal{D})
$$

We use the H-coefficient technique~\cite{SAC:Patarin08,EC:CheSte14} to prove various indistinguishability results. We provide a quick overview of its main ingredients here. Our presentation is essentially that of Chen and Steinberger~\cite{EC:CheSte14}; for further details, refer there or to the tutorial by Patarin~\cite{SAC:Patarin08}.
  Fix a distinguisher $\mathcal{D}$  that makes at most $\mathnormal{q}$ queries to its oracles. As in the security definition presented above, $\mathcal{D}$ 's  aim is to distinguish between two
worlds: a ``real world'' and an ``ideal world''. Assume without loss of generality that $\mathcal{D}$ is deterministic. The execution of $\mathcal{D}$  defines a transcript that includes the sequence of queries and answers received from its oracles; $\mathcal{D}$ 's output is a deterministic function of its transcript. Thus if $\mathnormal{X}$, $\mathnormal{Y}$ denote the probability distributions on transcripts induced by the real and ideal worlds, respectively, then $\mathcal{D}$ 's distinguishing advantage is upper bounded by the statistical distance

$$
\Delta(X, Y):=\frac{1}{2} \sum_{\tau}|\operatorname{Pr}[X=\tau]-\operatorname{Pr}[Y=\tau]|
$$

where the sum is taken over all possible transcripts $\tau$.
Let $\mathcal{T}$ denote the set of all transcripts that can be generated by $\mathcal{D}$ in either world. We look for a partition of $\mathcal{T}$  into two sets $\mathcal{T}_{1}$ and $\mathcal{T}_2$ of ``good'' and ``bad'' transcripts, respectively, along with a constant $\epsilon_{1} \in \left[0,1\right)$ such that

$$
\tau \in \mathcal{T}_{1} \Longrightarrow \operatorname{Pr}[X=\tau] / \operatorname{Pr}[Y=\tau] \geq 1-\epsilon_{1}
$$

It is then possible to show (see~\cite{EC:CheSte14} for details) that

$$
\Delta(X, Y) \leq \epsilon_{1}+\operatorname{Pr}\left[Y \in \mathcal{T}_{2}\right]
$$

For a transcript $\tau=\left(\mathcal{Q}_{C}, \mathcal{Q}_{S}\right)$, a key $k \in \mathcal{K}^{r+1}$, a permutation $\mathcal{P} \in \operatorname{Perm}( w n)$, a set of S-boxes $\mathcal{S} =\left(S_{1}, \ldots, S_{r}\right) \in \operatorname{Perm}( n)^{r}$ and $j \in \{1, \ldots, r\}$: if $S_{j}(u_{i})=v_{i}$ for every $i=1, \ldots, p$, then we will write $S_{j} \vdash \mathcal{Q}_{S_{j}}$. We will write $\mathcal{S} \vdash \mathcal{Q}_{S}$ if $S_{j} \vdash \mathcal{Q}_{S_{j}}$ for every $j \in \{1, \ldots, r\}$. Similarly, if $\operatorname{SP}_{\mathbf{k}}[\mathcal{S}]\left(x_{i}\right)=y_{i}$ (resp.$\mathcal{P}\left(x_{i}\right)=y_{i}$) for every $i=1, \ldots, q$, then we will write $\operatorname{SP}_{\mathbf{k}}[\mathcal{S}] \vdash \mathcal{Q}_{C}$(resp.$\mathcal{P} \vdash \mathcal{Q}_{C}$).\\

Then for $Y \in \mathcal{T}_{2}$, we have
$$
\begin{aligned}
&\operatorname{Pr}_{r e}(\tau, k) = \operatorname{Pr}\left[\mathbf{k} \stackrel{s}{\leftarrow} \mathcal{K}^{r+1}, \mathcal{S} \stackrel{\mathrm{s}}{\leftarrow} \operatorname{Perm}(n)^{r}: \operatorname{SP}_{\mathbf{k}}[\mathcal{S}] \vdash \mathcal{Q}_{C}, \mathcal{S} \vdash \mathcal{Q}_{S}\right]\\
&\operatorname{Pr}_{i d}(\tau, k) = \operatorname{Pr}\left[\mathcal{P} \stackrel{s}{\leftarrow} \widehat{\operatorname{Perm}}(w n), \mathcal{S} \stackrel{\mathrm{s}}{\leftarrow} \operatorname{Perm}(n)^{r}: \mathcal{P} \vdash \mathcal{Q}_{C}, \mathcal{S} \vdash \mathcal{Q}_{S}\right]\\
\end{aligned}
$$


\subsection{Indistinguishability in the Multi-user Setting}
We concentrate on the MU security with $\mathnormal{m}$  users. The SU security definition corresponds to the special case of $m=1$. Concretely, let $\mathrm{SP}_k[\mathcal{S}]$ be an $\mathnormal{r}$ -round SPN based on a set of S-boxes  $\mathcal{S}=(S_1, \cdots  ,S_r)$. In  the multi-user setting, let $\ell$ denote the number of users. In the real world $\ell$ secret keys  $k_1,\cdots ,k_{\ell} \in \mathcal{K}^{r+1}$  are chosen independently at random. A set of independent S-boxes $\mathcal{S}=(S_1, \cdots  ,S_r)$ is also randomly chosen from  $\operatorname{Perm(n)}^r$. A distinguisher $\mathcal{D}$  is given oracle access to $(\operatorname{SP}_{k_1}[\mathcal{S}], \cdots ,\operatorname{SP}_{k_{\ell}}[\mathcal{S}])$ as well as $\mathcal{S}=(S_1, \cdots  ,S_r)$. In the ideal world, $\mathcal{D}$ is given a set of independent random permutation $\mathcal{P} = (\operatorname{P}_1, \cdots ,\operatorname{P}_{\ell}) \in \widetilde{\operatorname{Perm}}(wn)^l$ instead of $(\operatorname{SP}_{k_1}[\mathcal{S}], \cdots ,\operatorname{SP}_{k_{\ell}}[\mathcal{S}])$. However, oracle access to $\mathcal{S}=(S_1, \cdots  ,S_r)$ is still allowed in the world. \\

The adversarial goal is to tell apart the two worlds, the $(\operatorname{SP}_{k_1}[\mathcal{S}], \cdots ,\operatorname{SP}_{k_l}[\mathcal{S}],\mathcal{S})$ and the ideal world $(\operatorname{P}_1, \cdots ,\operatorname{P}_{\ell},\mathcal{S})$ by adaptively making forward and backward queries to each of the constructions and the S-boxes. Formally, $\mathcal{D}$ 's distinguishing advantage is defined by
$$
\begin{aligned}
\operatorname{Adv}_{\mathrm{SP}}^{mu}(\mathcal{D}) &=\operatorname{Pr}\left[\widetilde{\operatorname{P}}_{1}, \ldots, \widetilde{\operatorname{P}}_{\ell} \stackrel{\mathrm{s}}{\leftarrow} \widetilde{\operatorname{Perm}}(w n)^l, \mathcal{S} \stackrel{\mathrm{s}}{\leftarrow} \operatorname{Perm}(n)^{r}: 1 \leftarrow \mathcal{D}^{\mathcal{S}, \widetilde{\operatorname{P}}_{1}, \ldots, \widetilde{\operatorname{P}}_{\ell}}\right] \\
&-\operatorname{Pr}\left[\mathbf{k}_{1}, \ldots, \mathbf{k}_{\ell} \stackrel{s}{\leftarrow} \mathcal{K}^{r+1}, \mathcal{S} \stackrel{s}{\leftarrow} \operatorname{Perm}(n)^{r}: 1 \leftarrow \mathcal{D}^{\mathcal{S}, \operatorname{SP}_{k_{1}}[\mathcal{S}], \ldots, \operatorname{SP}_{k_{\ell}}[\mathcal{S}]}\right]
\end{aligned}                                                                                                                                                                                                                        $$                                                                                                                                                                                                                                                                                                                                                                                                                                                                                                                                                                                                                                                                                                                                               for $p,q > 0$, we define

$$
\operatorname{Adv}^{mu}_{\mathrm{SP}}(p, q) = \max _{\mathcal{D}} \operatorname{Adv}_{\mathrm{SP}}(\mathcal{D})
$$
where the maximum is taken over all adversaries $\mathcal{D}$ making at most $\mathnormal{p}$ queries to each of the S-boxes and at most $\mathnormal{q}$ queries to the outer permutations. In the single-user setting with $\emph{l} = 1$. $\operatorname{Adv}^{mu}_{\mathrm{SP}}(\mathcal{D})$ and $\operatorname{Adv}^{mu}_{\mathrm{SP}}(p, q)$  will also be written as $\operatorname{Adv}^{su}_{\mathrm{SP}}(\mathcal{D})$ and $\operatorname{Adv}^{su}_{\mathrm{SP}}(p, q)$, respectively.


