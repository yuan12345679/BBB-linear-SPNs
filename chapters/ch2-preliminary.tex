

\section{Preliminaries}
\label{sec:preliminary}

Throughout this work, we fix positive integers $w$ and $n$, and let $N=2^n$. An element $x$ in $\{0,1\}^{wn}$ can be viewed as a concatenation of $w$ blocks of length $n$. The $i$th block of this representation will be denoted $x[i]$ for $i=1,\ldots,w$, so we have $x=x[1]\|x[2]\|\ldots\|x[w]$. For any integer $r$ such that $r\geq s$, we will write $(r)_s = r!/(r-s)!$, and define $(r)_0:=1$ for completeness.





\arrangespace


%\medskip
\noindent{\bf Tweakable Blockciphers.}
%
%
For an integer $m\geq1$, the set of all permutations on $\{0, 1\}^m$ will be denoted $\textsf{Perm}(m)$. A tweakable permutation with tweak space
$\mathcal{T}$ and message space $\mathcal{X}$ is a mapping $\widetilde{P}:\mathcal{T}\times\mathcal{X}\rightarrow\mathcal{X}$ such that, for any tweak
$t\in\mathcal{T}$,
%
$$x\mapsto\widetilde{P}(t,x)$$
%
is a permutation of $\mathcal{X}$. The set of all tweakable permutations with tweak space
$\mathcal{T}$ and message space $\{0, 1\}^m$ will be denoted $\widetilde{\textsf{Perm}}(\mathcal{T},m)$.


A tweakable blockcipher, or a keyed tweakable permutation, with key space $\mathcal{K}$, tweak space $\mathcal{T}$ and message
space $\mathcal{X}$ is a mapping $T:\mathcal{K}\times\mathcal{T}\times\mathcal{X}\rightarrow\mathcal{X}$ such that, for any key $k\in\mathcal{K}$,
%
$$(t,x)\mapsto T(k,t,x).$$
%
is a tweakable permutation with tweak space $\mathcal{T}$ and message space $\mathcal{X}$.




\arrangespace
%\medskip
\noindent{\bf(Tweakable) Linear Substitution-Permutation Networks.}
%
A \emph{substitution-permutation network} (SPN) defines a keyed permutation via repeated invocation of two transformations: blockwise computation of a public, cryptographic permutation called an ``S-box,'' and application of a keyed, non-cryptographic permutation. In this paper we will only introduce a model of linear SPNs. Formally, let $\calK$ and $\calT$ be two
sets, and let $\bff= (f_0,...,f_r)$ be a $(r+1)$-tuple of functions from $\calK\times\calT$ to $\{0, 1\}^n$.

%The $r$-round tweakable Even-Mansour construction TEM[n, r, f] specifies, from an r-tuple P = (P1, . . . , Pr)
%of permutations of {0, 1}n, a tweakable block cipher with key space K, tweak space T , and
%message space {0, 1}n, simply denoted TEMP in the following (parameters [n, r, f] will always
%be clear from the context) which maps a key k 2 K, a tweak t 2 T , and a plaintext x 2 {0, 1}n
%to the ciphertext defined as (see Figure 1):




Formally, an $r$-round SPN taking inputs of length $w n$ is defined by $r+1$ round keys $tk_0,tk_1,\ldots,tk_r\in\{0,1\}^{wn}$, $r$ permutations $S_1,\ldots,S_r:\{0,1\}^{n} \rightarrow \{0,1\}^{n}$, and $\mathnormal{r}-1$ invertible linear permutations $T_1,\ldots,T_{r-1}\in\mathbb{F}^{w\times w}$. Given an input $x \in \{0,1\}^{w n}$, the output of the SPN is computed as follows:

\begin{itemize}
  \item[--]
  Let $x_1 := x$.
  \item[--]
  For $i = 1$ to $r-1$ do:
  \begin{itemize}
  	\item[1.] $y_{i} := \overline{S_i}(x_{i}\xor tk_{i-1})$, where $\overline{S_i}(x[1]\xor tk_{i-1}[1]\|\ldots\| x[w]\xor tk_{i-1}[w]) \stackrel{\text { def }}{=} S_i(x[1]\xor tk_{i-1}[1])\|\ldots\| S_i(x[w]\xor tk_{i-1}[w])$.
    \item[2.] 
    $x_{i + 1} := T_i\cdot y_i$.
  \end{itemize}
  \item[--] $x_{r+1} := \overline{S_r}(x_r\xor tk_{r-1})\xor tk_r$.
  \item[--]
  The output is $x_{r+1}$.
\end{itemize}

Note that this model matches the structure of popular SPN ciphers such as the AES, Serpent, the ISO/IEC lightweight standard PRESENT, and the popular tweakable blockciphers Skinny. Also note that our model follows~\cite[Sect. 4.2]{C:CDKLST18} and uses different S-boxes in different rounds. We remark that some other~\cite[Sect. 3]{C:CDKLST18} assumed the same S-box in every round. Finally, we refer to~\cite[Sect. 2.1]{EPRINT:DKSTZ17} for a more general model of SPNs and its connection to the above model.


We will mostly be interested in the case where
%K = ({0, 1}n)a and T = ({0, 1}n)b for
%integers a, b  1. In this setting, we will denote k = (k0, . . . , ka−1) and t = (t0, . . . , tb−1), all
%ki’s and tj ’s being n-bit strings, or simply k = k, resp. t = t when a = 1, resp. b = 1. When all
%fi’s are linear over ({0, 1}n)a+b,
we say that the construction has linear tweak and key mixing.








\arrangespace

%\medskip
\noindent{\bf Multi-user Security Definitions.}
%
Let $\spn^T[\mathcal{S}]$ be an $r$-round SPN based on a set of S-boxes $\mathcal{S}=(S_1, \ldots  ,S_r)$ and an invertible linear permutation $T$ with key space $\calK$ and tweak space $\calT$. So $\spn^T[\mathcal{S}]$
becomes a keyed tweakable permutation on $\{0, 1\}^{wn}$ with key space $\calK^{r+1}$ and tweak space \calT.


In the multi-user setting, let $\ell$ denote the number of users. In the real
world, $\ell$ secret keys $\bfk_1,\ldots,\bfk_\ell\in\calK^{r+1}$ are chosen independently at random.
A set of independent S-boxes $\mathcal{S}=(S_1,\ldots,S_r)$ is also randomly chosen from $\textsf{Perm}(n)^r$. A distinguisher \dis is given oracle access to $(\spn_{\bfk_1}^T[\mathcal{S}],\ldots,\spn_{\bfk_\ell}^T[\mathcal{S}])$ as
well as $\mathcal{S}=(S_1,\ldots,S_r)$. In the ideal world, \dis is given a set of independent
random tweakable permutations $\widetilde{\calP}=(\widetilde{P}_1,\ldots,\widetilde{P}_{\ell})\in\widetilde{\operatorname{Perm}}(\calT,wn)^\ell$ instead of $(\spn_{\bfk_1}^T[\mathcal{S}],\ldots,\spn_{\bfk_\ell}^T[\mathcal{S}])$. Oracle access to $\mathcal{S}=(S_1,\ldots,S_r)$ is still allowed in this world.


The adversarial goal is to tell apart the two worlds $(\spn_{\bfk_1}^T[\mathcal{S}],\ldots,\spn_{\bfk_\ell}^T[\mathcal{S}],\mathcal{S})$ and $(\widetilde{P}_1,\ldots,\widetilde{P}_{\ell},\mathcal{S})$ by adaptively making forward and backward queries to each
of the constructions and the S-boxes. Formally, $\dis$'s distinguishing advantage is
defined by
%
$$
\begin{aligned}
\operatorname{Adv}_{\spn^T}^{\mathrm{mu}}(\mathcal{D}) &=\operatorname{Pr}\left[\widetilde{P}_1,\ldots,\widetilde{P}_{\ell} \stackrel{\$}{\leftarrow} \widetilde{\mathsf{Perm}}(w n)^\ell, \mathcal{S} \stackrel{\$}{\leftarrow} \mathsf{Perm}(n)^{r}: 1 \leftarrow \mathcal{D}^{\mathcal{S}, \widetilde{P}_{1}, \ldots, \widetilde{P}_{\ell}}\right] \\
&-\operatorname{Pr}\left[\mathbf{k}_{1}, \ldots, \mathbf{k}_{\ell} \stackrel{\$}{\leftarrow} \mathcal{K}^{\ell}, \mathcal{S} \stackrel{\$}{\leftarrow} \operatorname{Perm}(n)^{r}: 1 \leftarrow \mathcal{D}^{\mathcal{S}, \spn_{k_{1}}^T[\mathcal{S}], \ldots, \spn_{k_{\ell}}^T[\mathcal{S}]}\right].
\end{aligned}
$$
%
For $p,q > 0$, we define
%
$$
\operatorname{Adv}^{\mathrm{mu}}_{\spn^T}(p, q) = \max _{\mathcal{D}} \operatorname{Adv}_{\spn^T}(\mathcal{D})
$$
%
where the maximum is taken over all adversaries $\mathcal{D}$ making at most $\mathnormal{p}$ queries to each of the S-boxes and at most $\mathnormal{q}$ queries to the outer tweakable permutations. In the single-user setting with $\emph{l} = 1$. $\operatorname{Adv}^{\mathrm{mu}}_{\spn^T}(\mathcal{D})$ and $\operatorname{Adv}^{\mathrm{mu}}_{\spn^T}(p, q)$  will also be written as $\operatorname{Adv}^{\mathrm{su}}_{\spn^T}(\mathcal{D})$ and $\operatorname{Adv}^{\mathrm{su}}_{\spn^T}(p, q)$, respectively.



Note that, if we set $\calT$ to a singleton set, then the classical definition of strong pseudorandom permutations is recovered. This will be the target of Sect. \ref{section:security of 4-round SPNs}.






\arrangespace

%\medskip
\noindent{\bf The H-coefficient Technique.}
%
Suppose that a distinguisher $\mathcal{D}$ makes $\mathnormal{p}$ queries to each of the S-boxes, and total $\mathnormal{q}$ queries to the construction oracles. The queries made to the $j$-th construction oracle, denoted $C_j$, are recorded in a query history
%
$$
\mathcal{Q}_{C_j} = (j,t_{j,i},x_{j,i},y_{j,i})_{1 \leq i \leq q_j}
$$
%
for $j=1,...,\ell$, where $q$ is the number of queries made to $C_j$ and $(j,t_{j,i},x_{j,i},y_{j,i})$ represents the evaluation obtained by the $i$th query to $C_j$. So according to the instantiation, it implies either $\spn_{\bfk_j}^T[\mathcal{S}](t_{j,i},x_{j,i}) = y_{j,i}$  or $\widetilde{P}_j(t_{j,i},x_{j,i}) = y_{j,i}$. Let
%
$$\mathcal{Q}_{C}=\mathcal{Q}_{C_1}\cup\ldots\cup\mathcal{Q}_{C_\ell}.$$
%
For $j = 1,\ldots,r$, the queries made to $S_j$ are recorded in a query history
%
$$
\mathcal{Q}_{S_j} = (j, u_{j,i}, v_{j,i})_{1 \leq i \leq p}
$$
%
where $(j, u_{j,i}, v_{j,i})$  represents the evaluation $S_j(u_{j,i}) = v_{j,i}$ obtained by the $\mathnormal{i}$th query to $S_j$. Let
%
$$
\mathcal{Q}_{S}=\mathcal{Q}_{S_1} \cup \ldots \cup \mathcal{Q}_{S_r}
$$
%
Then the pair of query histories
%
$$
\tau = (\mathcal{Q}_{C}, \mathcal{Q}_{S})
$$
%
will be called the transcript of the attack: it contains all the information that $\mathcal{D}$ has obtained at the end of the attack. In this work, we will only consider
information theoretic distinguishers. Therefore we can assume that a distinguisher is deterministic without making any redundant query, and hence the output of  $\mathcal{D}$  can be regarded as a function of $\tau$, denoted $\mathcal{D}(\tau)$ or $\mathcal{D}(\mathcal{Q}_C, \mathcal{Q}_S)$.





Fix a transcript $\tau = (\mathcal{Q}_C,\mathcal{Q}_S)$, a key $\bfk \in \mathcal{K}^{r+1}$, a tweakable permutation $\widetilde{P} \in \widetilde{\mathsf{Perm}}(\mathcal{T}, w n)$,  a set of S-boxes $\mathcal{S}=(S_1, \ldots  ,S_r) \in \mathsf{Perm}(n)^r $ and $j \in \{1, \ldots, \ell \}$: if $S_j(u_{j,i})=v_{j,i}$ for every $i = 1, . . . , p$, then we will write $S_j\vdash\mathcal{Q}_{S_j}$. We will write $\mathcal{S}\vdash\mathcal{Q}_S$ if $S_j\vdash\mathcal{Q}_{S_j}$ for every $j = 1, . . . , r$. Similarly, if
$\spn_{\bfk}^T[\mathcal{S}](t_{j,i}, x_{j,i}) = y_{j,i}$ (resp. $\widetilde{P}(t_{j,i}, x_{j,i}) = y_{j,i}$) for every $i = 1, . . . , q_j$, then we will write $\spn_{\bfk}^T[\mathcal{S}]\vdash\mathcal{Q}_{C_j}$ (resp. $\widetilde{P}\vdash\mathcal{Q}_{C_j}$).


Let $\bfk_1, \ldots ,\bfk_{\ell} \in \mathcal{K}^{\ell+1}$ and $\widetilde{\mathcal{P}} = (\widetilde{P}_1, \ldots ,\widetilde{P}_\ell) \in \widetilde{\mathsf{Perm}}(\mathcal{T},wn)^\ell$, if $\spn_{\bfk_j}^T[\mathcal{S}]\vdash\mathcal{Q}_{C_j}$ (resp. $\widetilde{P}_j\vdash\mathcal{Q}_{C_j}$) for every $j = 1, \ldots ,\ell$, then we will write $(\spn_{\bfk_j}^{T}[\mathcal{S}])_{j = 1, \ldots ,\ell} \vdash \mathcal{Q}_C$ (resp. $\widetilde{P}\vdash\mathcal{Q}_{C}$).


If there exist $\widetilde{\mathcal{P}} \in \widetilde{\mathsf{Perm}}(\mathcal{T},w n)^\ell$ and $\mathcal{S} \in \mathsf{Perm}(\emph{n})^r$ that outputs $\tau$ at the end of the interaction with $\mathcal{D}$, then we will call the transcript $\tau$ attainable. So for any attainable transcript $\tau= (\mathcal{Q}_C,\mathcal{Q}_S)$, there exist $\widetilde{\mathcal{P}} \in \widetilde{\mathsf{Perm}}(\mathcal{T},w n)^\ell$ and $\mathcal{S} \in \mathsf{Perm}(n)^r$ such that $\widetilde{\mathcal{P}}\vdash\mathcal{Q}_C$ and $\mathcal{S}\vdash\mathcal{Q}_S$. For an attainable transcript $\tau = (\mathcal{Q}_C,\mathcal{Q}_S)$, let
%
%
$$
\begin{aligned}
&\mathsf{p}_{1}(\tau)=\operatorname{Pr}\left[\widetilde{\mathcal{P}} \stackrel{\$}{\leftarrow} \widetilde{{\mathsf{Perm}}}(\mathcal{T}, w n)^{\ell}, \mathcal{S} \stackrel{\$}{\leftarrow} \mathsf{Perm}(n)^{r}: \widetilde{\mathcal{P}} \vdash \mathcal{Q}_{C} \bigwedge \mathcal{S} \vdash \mathcal{Q}_{S}\right],\\
&\mathsf{p}_{2}(\tau)=\operatorname{Pr}\left[\bfk_{1}, \ldots, \bfk_{\ell} \stackrel{\$}{\leftarrow} \mathcal{K}^{\ell}, \mathcal{S} \stackrel{\$}{\leftarrow} \mathsf{Perm}(n)^{r}:(\spn_{\bfk_{j}}^{T}[\mathcal{S}])_{j} \vdash \mathcal{Q}_{C} \bigwedge \mathcal{S} \vdash \mathcal{Q}_{S}\right].
\end{aligned}
$$
%
%
With these definitions, the core lemma of the H-coefficients technique (without defining ``bad'' transcripts) is stated as follows.


\begin{lemma}
	\label{lemma:h-coeff}
	
	Let $\varepsilon \geq 0$. Suppose that for any attainable transcript $\tau = (\mathcal{Q}_C,\mathcal{Q}_S)$,
	\begin{align}
	\mathsf{p}_{2}(\tau) \geq (1 - \varepsilon) \mathsf{p}_{1}(\tau).
	\label{eq:h-ratio}
	\end{align}
	Then one has
	$$
	\operatorname{Adv}^{\mathrm{mu}}_{\spn^T}(\mathcal{D}) \leq \varepsilon.
	$$
\end{lemma}
%
%
The lower bound (\ref{eq:h-ratio}) is called \emph{$\epsilon$-point-wise proximity} of the transcript $\tau = (\mathcal{Q}_C, \mathcal{Q}_S)$. The point-wise proximity of a transcript in the multi-user setting is guaranteed by the point-wise proximity of $(\mathcal{Q}_{C_{j}}, \mathcal{Q}_S)$ for each $j = 1, \ldots ,\ell$ in the single user setting. The following lemma is a restatement of Lemma 3 in~\cite{C:HoaTes16}.


\begin{lemma}
	\label{lemma:point-wise}
	
	Let $\varepsilon : \mathbb{N} \times \mathbb{N} \rightarrow \mathbb{R}^{\geq 0}$ be a function such that
	\begin{itemize}
		\item[1.] $\varepsilon (x, y) + \varepsilon (x, z) \leq \varepsilon (x, y + z)$ for every $x, y, z \in \mathbb{N}$,
		\item[2.] $\varepsilon (\cdot, z)$ and $\varepsilon (z, \cdot)$ are non-decreasing functions on $\mathbb{N}$ for every $z \in \mathbb{N}$.
	\end{itemize}
	Suppose that for any distinguisher $\mathcal{D}$ in the single-user setting that makes p primitive queries to each of the underlying S-boxes and makes q construction queries, and for any attainable transcript $\tau$ obtained by $\mathcal{D}$, one has
	%
	$$
	\mathsf{p}_{2}\left(\tau\right) \geq (1 - \varepsilon(p,q)) \mathsf{p}_{1}\left(\tau\right).
	$$
	%
	Then for any distinguisher $\mathcal{D}$ in the multi-user setting that makes $\mathnormal{p}$ primitive queries to each of the underlying S-boxes and makes total $\mathnormal{q}$ construction queries, and for any attainable transcript $\tau$ obtained by $\mathcal{D}$, one has
	%
	$$
	\mathsf{p}_{2}\left(\tau\right) \geq (1 - \varepsilon(p + wq,q)) \mathsf{p}_{1}\left(\tau\right).
	$$
\end{lemma}



%
%For any extended transcript $\tau^{\prime} = (\mathcal{Q}_C, \mathcal{Q}_S, \mathcal{Q}_S^{\prime},k)$, where $\mathcal{Q}_{S}^{(1)} = \mathcal{Q}_S \cup \mathcal{Q}_S^{\prime}$, denote
%
%$$
%\mathrm{p}\left(\tau^{\prime}\right)=\operatorname{Pr}\left[\mathcal{S} \stackrel{\mathrm{s}}{\leftarrow} \operatorname{Perm}(n)^{2}: \mathrm{\operatorname{SP}}^{T}_{k}[\mathcal{S}] \vdash \mathcal{Q}_{C} |\left(S_{1} \vdash \mathcal{Q}_{S_{1}}^{(1)}\right) \wedge\left(S_{2} \vdash \mathcal{Q}_{S_{2}}^{(1)}\right)\right].
%$$
%
%Then we will get the following lemma:
%
%\begin{lemma}
%\label{lemma:ratio-2-rounds}
%
%For any good extended transcript $\tau^{\prime}$, one has
%$$
%\left(2^{w n}\right)_{q} \mathrm{p}\left(\tau^{\prime}\right) \geq 1-\frac{q^{2}}{2^{w n}}-\frac{q\left(2 w p+6 w^{2} q\right)^{2}}{2^{2 n}}.
%$$
%\end{lemma}
%

