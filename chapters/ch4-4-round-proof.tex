
\section{SPRP Security of 4-Round SPNs}
\label{section:security of 4-round SPNs}

In this section, we prove beyond-birthday-bound SPRP security for 4-round linear SPNs. Concretely, let $\spn_{\bfk}[\mathcal{S}]$ be the 4-round SPN using any linear transformations $T$. I.e.,
%
\begin{align}
\spn_{\bfk}^T[\mathcal{S}](x):=k_4\xor\overline{S_4}(k_3\xor T(\overline{S_3}(k_2\xor T(\overline{S_2}(k_1\xor T(\overline{S_1}(k_0\xor x))))))).
\label{eq:defn-4-round-spn}
\end{align}
%
We show that $\spn^T$ is an SPRP as long as: (i) the linear layer $T$ contains no zero entries (Miles and Viola~\cite{miles2015substitution} show that matrices with maximal branch number~\cite{daemen1995cipher} satisfy this property), and (ii) the round keys $\{k_i\}(i=0, 1, 2, 3, 4)$ are uniform and independent.


\begin{theorem}
\label{theorem:4-round-spn}

Assume $w\geq2$, and $p+wq\leq N/2$. Let $\spn_{\bfk}[\mathcal{S}]$ be a 4-round, linear SPN as defined by Eq. (\ref{eq:defn-4-round-spn}). If round keys $\bfk=(k_0,k_1,k_2,k_3,k_4)$ are uniform and independent, and $T$ contains no zero entries, then
%
\begin{align}
\operatorname{Adv}_{\spn^T}^{\mathrm{su}}(p, q) &\leq \frac{q^2}{2^{n w}} + \frac{8 w^2 q(p+wq)^2+w^2 q}{2^n}\\
&+ \frac{16 w^2 q(p+w q)(p+w q +3 q)+4 w^2 q(p+3 wq)^2+ w^2q(p+w q)(3p+w q)}{2^{2 n}}.
\label{eq:bound-4-rounds}
\end{align}
\end{theorem}
The proof of Theorem \ref{eq:bound-4-rounds} relies on the following lemma and on Lemma \ref{lemma:h-coeff} and Lemma \ref{lemma:point-wise}.


\begin{lemma}
	\label{lemma:proximity-4-round}
	
	Assume $p+wq\leq N/2$. Let $\dis$ be a distinguisher in the single-user setting that makes $p$ primitive queries to each of $S_1$ and $S_2$ and makes $q$ construction queries. Then for any attainable
	transcript $\tau=(\mathcal{Q}_C,\mathcal{Q}_S)$, one has
	\begin{align}
	\frac{\mathsf{p}_{2}\left(\mathcal{Q}_{C} | \mathcal{Q}_{S}\right)}{\mathsf{p}_{1}\left(\mathcal{Q}_{C} | \mathcal{Q}_{S}\right)}
	\geq 1 - xxx.
	\label{eq:bound-proximity-4-round}
	\end{align}
\end{lemma}




\subsection{Outline of the Proof}
\label{sec:proof-sketch-4-rounds}

Throughout the proof, we fix a distinguisher $\mathcal{D}$ as described in the statement and fix an attainable transcript $\tau =\left(\mathcal{Q}_{C}, \mathcal{Q}_{S}\right)$ obtained $\mathcal{D}$. Let
%
$$
\begin{aligned}
&\mathcal{Q}_{S_{1}}^{(0)}=\left\{(u, v) \in\{0,1\}^{n} \times\{0,1\}^{n}:(1, u, v) \in \mathcal{Q}_{S} \right\},\\
&\mathcal{Q}_{S_{2}}^{(0)}=\left\{(u, v) \in\{0,1\}^{n} \times\{0,1\}^{n}:(2, u, v) \in \mathcal{Q}_{S} \right\},\\
&\mathcal{Q}_{S_{3}}^{(0)}=\left\{(u, v) \in\{0,1\}^{n} \times\{0,1\}^{n}:(3, u, v) \in \mathcal{Q}_{S} \right\},\\
&\mathcal{Q}_{S_{4}}^{(0)}=\left\{(u, v) \in\{0,1\}^{n} \times\{0,1\}^{n}:(4, u, v) \in \mathcal{Q}_{S} \right\}
\end{aligned}
$$
%
and let       {\small
%
\begin{align*}
&U_{1}^{(0)}=\left\{u_{1} \in\{0,1\}^{n}:\left(1, u_{1}, v_{1}\right) \in \mathcal{Q}_{S_{1}}^{(0)}\right\}, \quad V_{1}^{(0)}=\left\{v_{1} \in\{0,1\}^{n}:\left(1, u_{1}, v_{1}\right) \in \mathcal{Q}_{S_{1}}^{(0)}\right\},\\
&U_{2}^{(0)}=\left\{u_{2} \in\{0,1\}^{n}:\left(2, u_{2}, v_{2}\right) \in \mathcal{Q}_{S_{2}}^{(0)}\right\}, \quad V_{2}^{(0)}=\left\{v_{2} \in\{0,1\}^{n}:\left(2, u_{2}, v_{2}\right) \in \mathcal{Q}_{S_{2}}^{(0)}\right\},\\
&U_{3}^{(0)}=\left\{u_{3} \in\{0,1\}^{n}:\left(3, u_{3}, v_{3}\right) \in \mathcal{Q}_{S_{3}}^{(0)}\right\}, \quad V_{3}^{(0)}=\left\{v_{3} \in\{0,1\}^{n}:\left(3, u_{3}, v_{3}\right) \in \mathcal{Q}_{S_{3}}^{(0)}\right\},\\
&U_{4}^{(0)}=\left\{u_{4} \in\{0,1\}^{n}:\left(4, u_{4}, v_{4}\right) \in \mathcal{Q}_{S_{4}}^{(0)}\right\}, \quad V_{4}^{(0)}=\left\{v_{4} \in\{0,1\}^{n}:\left(4, u_{4}, v_{4}\right) \in \mathcal{Q}_{S_{4}}^{(0)}\right\}.
\end{align*}
}%
%
Denote the domains and ranges of $\mathcal{Q}_{S_{1}}^{(0)}, \mathcal{Q}_{S_{2}}^{(0)}, \mathcal{Q}_{S_{3}}^{(0)}, \mathcal{Q}_{S_{4}}^{(0)}$, respectively.



This type of lemma is usually proved by defining a large enough set of ``good'' keys, and then, for each choice of a good key, lower bounding the probability of observing this transcript. A key is usually said to be good if the adversary cannot use the transcript to follow the path of computation of the encryption/decryption of a query up to a contradiction. To this end, we follow~\cite[Sect. 4.2]{C:CDKLST18} and define an extension of the transcript in order to gather enough information to allow simple definition of bad keys/transcripts. Then, instead of summing over the choice of the key, we will define an extension of the transcript, that will provide the necessary information, and then sum over every possible good extension.


We will first define what we mean by an extension of the transcript $\tau$. Finally, we will show that the number of bad extended transcripts is small enough in Lemma 6, and then show that the probability to obtain any good extension in the real world is sufficiently close to the probability to obtain $\tau$ the ideal world in Lemma 7. We stress that extended transcripts are completely virtual and are not disclosed to the adversary. They are just an artificial intermediate step to lower bound the probability to observe transcript $\tau$ in the real world.


We will extend the transcript $\tau$ of the attack via a certain randomized process. We begin with choosing a pair of keys $\left(k_{0}, k_{4}\right) \in \mathcal{K}^{2}$ uniformly at random. Once these keys have been chosen, some construction queries will become involved in collisions. A colliding query is defined as a construction query $(x, y) \in \mathcal{Q}_{C}$ such that one of the following conditions holds:

\begin{itemize}
  \item[1.]
  there exist an S-box query $(1, u, v) \in \mathcal{Q}_{S}$ and an integer $i \in\{1, \ldots, w\}$ such that $\left(x \oplus k_{0}\right)[i]=u$.
  \item[2.]
  there exist an S-box query $(4, u, v) \in \mathcal{Q}_{S}$ and an integer $i \in\{1, \ldots, w\}$ such that $\left(y \oplus k_{4}\right)[i]=v$.
%  \item[3.]
%  there exist a construction query $\left(x^{\prime}, y^{\prime}\right) \in \mathcal{Q}_{C}$ and an integer $i,j \in\{1, \ldots, w\}$ such that $(x, y, i) \neq\left(x^{\prime}, y^{\prime}, j\right)$ and $\left(x \oplus k_{0}\right)[i] = \left(x' \oplus k_{0}\right)[j]$.
%  \item[4.]
%  there exist a construction query $\left(x^{\prime}, y^{\prime}\right) \in \mathcal{Q}_{C}$ and an integer $i,j \in\{1, \ldots, w\}$ such that $(x, y, i) \neq\left(x^{\prime}, y^{\prime}, j\right)$ and $i \in\{1, \ldots, w\}$ such that $\left(y \oplus k_{4}\right)[i] = \left(y' \oplus k_{4}\right)[j]$.
\end{itemize}

We are now going to build a new set $\mathcal{Q}_{S_{outer}}^{\prime}$ of S-box evaluations that will play the role of an extension of $\mathcal{Q}_{S}$. For each colliding query $(x, y) \in \mathcal{Q}_{C}$, we will add tuples $\left(1, \left(x \oplus k_{0}\right)[i], v^{\prime}\right)_{1 \leq i \leq w}$ (if ($\mathit{x}$, $\mathit{y}$) collides at the input of $S_1$) or $\left(4, u^{\prime}, \left(y \oplus k_{4}\right)[i]\right)_{1 \leq i \leq w}$ (if ($\mathit{x}$, $\mathit{y}$) collides at the output of $S_4$) by lazy sampling $v^{\prime}=S_{1}(\left(x \oplus k_{0}\right)[i])$ or $u^{\prime}=S_{4}^{-1}(\left(y \oplus k_{4}\right)[i])$, as long as it has not been determined by any existing query in $\mathcal{Q}_{S}$. Then we choose the key $k_1,k_3$ uniformly at random. An extended transcript of $\tau$ will be defined as a tuple $\tau^{\prime}=\left(\mathcal{Q}_{C}, \mathcal{Q}_{S}, \mathcal{Q}_{S_{outer}}^{\prime}, \mathbf{k}\right)$ where $\mathbf{k}=\left(k_{0}, k_{1},k_{3},k_{4}\right)$. For each collision between a construction query and a primitive query, or between two construction queries, the extended transcript will contain enough information to compute a complete round of the evaluation of the SPN. This will be useful to lower bound the probability to get the transcript $\tau$ in the real world.





\subsection{Bad Transcript Extensions and Probability}
\label{sec:bad-tau-4-rounds}

The first step is to define the set of bad transcripts. Let $\tau = (\mathcal{Q}_C, \mathcal{Q}_{S})$ be an attainable transcript, with $|\mathcal{Q}_C| = q$ and $|\mathcal{Q}_{S_i}| = p$ for $i = 1, \ldots, 4$. Let
%
$$
\begin{aligned}
&\mathcal{Q}_{S_{1}}^{(1)}=\left\{(u, v) \in\{0,1\}^{n} \times\{0,1\}^{n}:(1, u, v) \in \mathcal{Q}_{S} \cup \mathcal{Q}_{S_{outer}}^{\prime}\right\}\\
&\mathcal{Q}_{S_{4}}^{(1)}=\left\{(u, v) \in\{0,1\}^{n} \times\{0,1\}^{n}:(4, u, v) \in \mathcal{Q}_{S} \cup \mathcal{Q}_{S_{outer}}^{\prime}\right\}
\end{aligned}
$$
%
In words, $\mathcal{Q}_{S_{i}}^{(1)}$ summarizes each constraint that is forced on $S_{i}$ by $\mathcal{Q}_{S}$ and $\mathcal{Q}_{S_{outer}}^{\prime}$. Let 
%
$$
\begin{aligned}
&U_{1}=\left\{u_{1} \in\{0,1\}^{n}:\left(1, u_{1}, v_{1}\right) \in \mathcal{Q}_{S_{1}}^{(1)}\right\}, \quad V_{1}=\left\{v_{1} \in\{0,1\}^{n}:\left(1, u_{1}, v_{1}\right) \in \mathcal{Q}_{S_{1}}^{(1)}\right\},\\
&U_{4}=\left\{u_{4} \in\{0,1\}^{n}:\left(4, u_{4}, v_{4}\right) \in \mathcal{Q}_{S_{4}}^{(1)}\right\}, \quad V_{4}=\left\{v_{4} \in\{0,1\}^{n}:\left(4, u_{4}, v_{4}\right) \in \mathcal{Q}_{S_{4}}^{(1)}\right\}
\end{aligned}
$$
%
be the domains and ranges of $\mathcal{Q}_{S_{1}}^{(1)}$ and $\mathcal{Q}_{S_{4}}^{(1)}$, respectively.

%We define two quantities characterizing an extended transcript $\tau^{\prime}$, namely
%$$
%\begin{aligned}
%&\alpha_{1} \stackrel{\text { def }}{=} |\left\{(x, y) \in \mathcal{Q}_{C}: \left(x \oplus k_{0}\right)[i] \in U_{1} \text { for some } i \in\{1, \ldots, w\}\right\} |\\
%&\alpha_{4} \stackrel{\text { def }}{=} |\left\{(x, y) \in \mathcal{Q}_{C}: \left(y \oplus k_{4}\right)[i] \in V_{4} \text { for some } i \in\{1, \ldots, w\}\right\} |
%\end{aligned}
%$$
%%
%In words, $\alpha_1$ (resp. $\alpha_4$) is the number of queries $(x, y) \in \mathcal{Q}_{C}$ which collide with a query $\left(u_{1}, v_{1}\right) \in \mathcal{Q}_{S_{1}}^{(1)}$ (resp. which collide with a query $\left(u_{4}, v_{4}\right) \in \mathcal{Q}_{S_{4}}^{(1)}$) in the extended transcript. This corresponds to the number of queries $(x, y) \in \mathcal{Q}_{C}$ which collide with either an original query $\left(u_{1}, v_{1}\right) \in \mathcal{Q}_{S_{1}}^{(0)}$ (resp. which collide with a query $\left(u_{4}, v_{4}\right) \in \mathcal{Q}_{S_{4}}^{(0)}$) or with a query $\left(x^{\prime}, y^{\prime}\right) \in \mathcal{Q}_{C}$ at an input of $S_1$ (resp. at the output of $S_4$ ), once the choice of $\left(k_{0}, k_{4}\right)$  has been made. We will also denote
%%
%$$
%\beta_{i}=\left|\mathcal{Q}_{S_{i}}^{(1)}\right|-\left|\mathcal{Q}_{S_{i}}^{(0)}\right|=\left|\mathcal{Q}_{S_{i}}^{(1)}\right|-p.
%$$
%%
%for $i=1, 4$ the number of additional queries included in the extended transcript.



\begin{definition}
\label{defn:bad-tau-4-rounds}

We say an extended transcript $\tau^{\prime}$ is bad if at least one of the following conditions is fulfilled:
\begin{itemize}
	\item[\bone]
	there exists $(x, y) \in \mathcal{Q}_{C}, u_1 \in U_1, v_4 \in V_4$, and index $i, j \in \{1, \ldots, w\}$ such that $\left(x \oplus k_{0}\right)[i]=u_1$ and $\left(y \oplus k_{4}\right)[j]=v_4$.
	\item[\btwo]
	there exists $(x,y) \in \mathcal{Q}_{C}, u_1 \in U_1, u_2\in U_2^{(0)}$, and index $i, j \in \{1, \ldots, w\}$ such that $\left(x \oplus k_{0}\right)[i]=u_1$ and $\left(T_{1}\left(\overline{S_1}\left(x \oplus k_{0}\right) \oplus k_{1}\right)\right)[j]=u_2$.
	\item[\bthree]
	there exists $(x,y) \in \mathcal{Q}_{C}, v_{3}\in V_3^{(0)}, v_{4}\in V_4$, and index $i, j \in \{1, \ldots, w\}$ such that $\left(y \oplus k_{4}\right)[j]=v_4$ and $\left(T_{3}^{-1}\left(\overline{S_4^{-1}}\left(y \oplus k_{4}\right)\right)\xor T_{3}^{-1}(k_3)\right) [i]=v_3$.
	\item[\bfour]
	there exists $(x,y) \in \mathcal{Q}_{C}$ and distinct indices $i, j \in \{1, \ldots, w\}$ such that $(x\xor k_0)[i]=(x\xor k_0)[j]$, or $(y\xor k_4)[i]=(y\xor k_4)[j]$.
\end{itemize}
Any extended transcript that is not bad will be called good. Given an original transcript $\tau$, we denote $\Theta_{good}(\tau)$ (resp. $\Theta_{bad}(\tau)$) the set of good (resp. bad) extended transcripts of $\tau$ and $\Theta^{'}(\tau)$ the set of all extended transcripts of $\tau$.
\end{definition}



We start by upper bounding the probability of getting bad transcripts in the ideal world.

\begin{lemma}
	\label{lemma:bad-tau-4-rounds}
	
	One has
	\begin{align}
	\operatorname{Pr}[\tau^{\prime} \in \Theta_{bad}(\tau)] \leq \frac{3w^{2} q \left(p+w q\right)^{2}}{N^{2}} + \frac{w^{2} q}{N}.
	\label{eq:bound-bad-tau-4-rounds}
	\end{align}
\end{lemma}
\begin{proof}
We upper bound the probabilities of the conditions in turn. Consider \bone first. For each fixed choice of $(x, y) \in \mathcal{Q}_{C}, \left(u_{1}, v_{1}\right) \in \mathcal{Q}_{S_{1}}^{(1)}, \left(u_{4}, v_{4}\right) \in \mathcal{Q}_{S_{4}}^{(1)}$ and indices $i, j \in \{1, \ldots, w\}$, since $k_{0}$ and $k_{4}$ are uniform and independent, the probability to have $(x \oplus k_{0})[i]=u_1$ and $(y \oplus k_{4})[j]=v_4$ is $1/N^2$. Since we have at most $w^{2} q \left(p+w q\right)^{2}$ such choices, we have
%
$$
\operatorname{Pr}\left[\bone\right] \leq \frac{w^{2} q \left(p+w q\right)^{2}}{N^{2}}.
$$
%

Similarly, since $k_{0}$ and $k_{1}$ are uniform and independent, and we have at most $w^{2} q p \left(p+w q\right)$ for $(x, y) \in \mathcal{Q}_{C}, \left(u_{1}, v_{1}\right) \in \mathcal{Q}_{S_{1}}^{(1)}, \left(u_{2}, v_{2}\right) \in \mathcal{Q}_{S_{2}}$ and indices $i, j \in \{1, \ldots, w\}$, we have $\operatorname{Pr}\left[\btwo\right] \leq \frac{w^{2} q p \left(p+w q\right)}{N^{2}}\leq\frac{w^{2} q \left(p+w q\right)^{2}}{N^{2}}$; since $k_3$ and $k_4$ are uniform and independent, we have $\operatorname{Pr}\left[\bthree\right] \leq \frac{w^{2} q \left(p+w q\right)^{2}}{N^{2}}$ by symmetry.


Finally, for \bfour, since $k_{0}$ and $k_{4}$ are uniform, for each choice of $(x,y) \in \mathcal{Q}_{C}$ and $i, j \in \{1, \ldots, w\}$, the probability to have $(x\xor k_0)[i]=(x\xor k_0)[j]$ or $(y\xor k_4)[i]=(y\xor k_4)[j]$ is $2/N$. Since the number of such choices is $q{w\choose 2}\leq w^2q/2$, we have $\operatorname{Pr}\left[\bfour\right] \leq \frac{w^{2} q}{N}$. Summing over the above yields (\ref{eq:bound-bad-tau-4-rounds}).       \qed
\end{proof}






\subsection{Analysis of Good Transcripts}


Fix a good transcript and a good round-key vector $\mathnormal{k}$, we are to derive a lower bound for the probability  $\operatorname{Pr}\left[\mathcal{S} \stackrel{\mathbf{s}}{\leftarrow}(\mathcal{S}(n))^{4}: \mathrm{SP}_{k}[\mathcal{S}] \vdash \mathcal{Q}_{C} | \mathcal{S} \vdash \mathcal{Q}_{S}\right]$. It consists of two steps. In the first step, we will lower bound the probability that a pair of functions $(S_{1}, S_{4})$  satisfies certain ``bad'' conditions that will be defined. With the values given by a ``good'' pair of functions $(S_{1}, S_{4})$, a transcript of the distinguisher on 4 rounds can be transformed into a special transcript on 2 rounds; in this sense, we ``peel off'' the outer two rounds. Then in the second step, assuming $(S_{1}, S_{4})$ is good, we will lower bound the probability, over the choice of $S_2$ and $S_3$, that $\calC\vdash\mathcal{Q}_{C}$. For this second step, we will directly appeal to a previous result by
Cogliati et al.~\cite{C:CDKLST18}.





\paragraph{\textsc{Further extension, and the outer two rounds.}} Pick a pair of S-box $(S_1, S_4)$ such that $S_{1} \vdash \mathcal{Q}_{S_{1}}^{(0)}$ and $S_{4} \vdash \mathcal{Q}_{S_{4}}^{(0)}$, and for each $ (x, y) \in \mathcal{Q}_{C}$ we set
%
$$a=T_1\big(\overline{S_1}\left(x \oplus k_{0}\right)\big),\ \ \  b=T_3^{-1}\big(\overline{S_{4}^{-1}}\left(y \oplus k_{4}\right)\big).$$
%
In this way we obtain $\mathnormal{q}$ tuples of the form $(a,b)$; for convenience we denote the set of such induced tuples by $\mathcal{Q}_{C}^{*}\left(S_{1}, S_{4}\right)$. Similarly, we also extended the inner two rounds:


A colliding query is defined as a construction query $(a, b) \in \mathcal{Q}_{C}^{*}\left(S_{1}, S_{4}\right)$ as follows:
%
\begin{itemize}
	\item[1.]
	there exist an S-box query $(2, u, v) \in \mathcal{Q}_{S}$ and an integer $i \in\{1, \ldots, w\}$ such that $\left(a \oplus k_1\right)[i]=u$.
	\item[2.]
	there exist an S-box query $(3, u, v) \in \mathcal{Q}_{S}$ and an integer $i \in\{1, \ldots, w\}$ such that $\left(b \oplus T^{-1}(k_3)\right)[i]=v$.
	\item[3.] there exist a construction query $\left(a^{\prime}, b^{\prime}\right) \in \mathcal{Q}_{C}$ and an integer $i,j \in\{1, \ldots, w\}$ such that $(a, b, i) \neq\left(a^{\prime}, b^{\prime}, j\right)$ and $\left(a \oplus k_1\right)[i] = \left(a' \oplus k_1\right)[j]$.
	\item[4.] there exist a construction query $\left(a^{\prime}, b^{\prime}\right) \in \mathcal{Q}_{C}$ and an integer $i,j \in\{1, \ldots, w\}$ such that $(a, b, i) \neq\left(a^{\prime}, b^{\prime}, j\right)$ and $i \in\{1, \ldots, w\}$ such that $\left(b \oplus T^{-1}(k_3)\right)[i] = \left(b' \oplus T^{-1}(k_3)\right)[j]$.
\end{itemize}
%
%
Recall that the domains and ranges of $\mathcal{Q}_{S_{2}}^{(0)}$ and $\mathcal{Q}_{S_{3}}^{(0)}$ and denoted by $U_{2}^{(0)},V_{2}^{(0)},U_{3}^{(0)}$, and $V_{3}^{(0)}$ respectively. We will further extend the extended transcript $\tau'$. In detail, we build a new set $\mathcal{Q}_{S_{inner}}^{\prime}$ of S-box evaluations that will play the role of an extension of $\mathcal{Q}_{C}^{*}\left(S_{1}, S_{4}\right)$. For each colliding query $(a, b) \in \mathcal{Q}_{C}^{*}\left(S_{1}, S_{4}\right)$, we will add tuples $\left(2, T_1\left(a \oplus k_{1}\right)[i], v^{\prime}\right)_{1 \leq i \leq w}$(if $(a, b)$ collides at the input of $S_2$) or $\left(3, u^{\prime}, T_{3}^{-1}\left(b\right) \oplus k_{3}[i]\right)_{1 \leq i \leq w}$ (if $(a, b)$ collides at the output of $S_3$) by lazy sampling, as long as it has not been determined by any existing query in $\mathcal{Q}_{C}^{*}\left(S_{1}, S_{4}\right)$, $v^{\prime}=S_{2}(\left(T_1\left(a \oplus k_{1}\right)\right))[i]$ or $u^{\prime}=S_{3}^{-1}(\left(T_{3}^{-1}\left(b\right) \oplus k_{3}\right)[i]$. Then we choose the key $k_2$ uniformly at random. An extended transcript of $\tau_{inner}$ will be defined as a tuple $\tau_{inner}^{\prime}=\left(\mathcal{Q}_{C}^{*}\left(S_{1}, S_{4}\right), \mathcal{Q}_{S_{inner}}, \mathcal{Q}_{S_{inner}}^{\prime}, \mathbf{k}\right)$ where $\mathbf{k}=\left(k_{1}, k_{2}, k_{3}\right)$. For each collision between a construction query and a primitive query, or between two construction queries, the extended transcript will contain enough information to compute a complete round of the evaluation of the SPN. This will be useful to lower bound the probability to get the transcript $\tau_{inner}$ in the real world. At this stage, let
%
$$
\begin{aligned}
&\mathcal{Q}_{S_2}^{(1)}=\left\{(u, v) \in\{0,1\}^{n} \times\{0,1\}^{n}:(2, u, v) \in \mathcal{Q}_{S} \cup \mathcal{Q}_{S_{inner}}^{\prime}\right\}\\
&\mathcal{Q}_{S_3}^{(1)}=\left\{(u, v) \in\{0,1\}^{n} \times\{0,1\}^{n}:(3, u, v) \in \mathcal{Q}_{S} \cup \mathcal{Q}_{S_{inner}}^{\prime}\right\}
\end{aligned}
$$
%
Also let 
%
$$
\begin{aligned}
&U_2=\left\{u_2 \in\{0,1\}^{n}:\left(2, u_2, v_2\right) \in \mathcal{Q}_{S_2}^{(1)}\right\}, \quad V_2=\left\{v_2 \in\{0,1\}^{n}:\left(2, u_2, v_2\right) \in \mathcal{Q}_{S_2}^{(1)}\right\},\\
&U_3=\left\{u_3 \in\{0,1\}^{n}:\left(3, u_3, v_3\right) \in \mathcal{Q}_{S_3}^{(1)}\right\}, \quad V_3=\left\{v_3 \in\{0,1\}^{n}:\left(3, u_3, v_3\right) \in \mathcal{Q}_{S_3}^{(1)}\right\}
\end{aligned}
$$
%
be the domains and ranges of $\mathcal{Q}_{S_2}^{(1)}$ and $\mathcal{Q}_{S_3}^{(1)}$, respectively.

%We define two quantities characterizing an extended transcript $\tau^{\prime}$, namely
%%
%$$
%\begin{aligned}
%&\alpha_{1} \stackrel{\text { def }}{=} |\left\{(x, y) \in \mathcal{Q}_{C}: \left(x \oplus k_{0}\right)[i] \in U_{1} \text { for some } i \in\{1, \ldots, w\}\right\} |\\
%&\alpha_{4} \stackrel{\text { def }}{=} |\left\{(x, y) \in \mathcal{Q}_{C}: \left(y \oplus k_{4}\right)[i] \in V_{4} \text { for some } i \in\{1, \ldots, w\}\right\} |
%\end{aligned}
%$$
%%
%In words, $\alpha_1$ (resp. $\alpha_4$) is the number of queries $(x, y) \in \mathcal{Q}_{C}$ which collide with a query $\left(u_{1}, v_{1}\right) \in \mathcal{Q}_{S_{1}}^{(1)}$ (resp. which collide with a query $\left(u_{4}, v_{4}\right) \in \mathcal{Q}_{S_{4}}^{(1)}$) in the extended transcript. This corresponds to the number of queries $(x, y) \in \mathcal{Q}_{C}$ which collide with either an original query $\left(u_{1}, v_{1}\right) \in \mathcal{Q}_{S_{1}}^{(0)}$ (resp. which collide with a query $\left(u_{4}, v_{4}\right) \in \mathcal{Q}_{S_{4}}^{(0)}$) or with a query $\left(x^{\prime}, y^{\prime}\right) \in \mathcal{Q}_{C}$ at an input of $S_1$ (resp. at the output of $S_4$ ), once the choice of $\left(k_{0}, k_{4}\right)$  has been made. We will also denote
%
%$$
%\beta_{i}=\left|\mathcal{Q}_{S_{i}}^{(1)}\right|-\left|\mathcal{Q}_{S_{i}}^{(0)}\right|=\left|\mathcal{Q}_{S_{i}}^{(1)}\right|-p.
%$$
%
%for $i=1, 4$ the number of additional queries included in the extended transcript.


%
%$$
%\begin{aligned}
%&\mathcal{Q}_{S_{2}}^{(1)}=\left\{(u, v) \in\{0,1\}^{n} \times\{0,1\}^{n}:(2, u, v) \in \mathcal{Q}_{S} \cup \mathcal{Q}_{S_{inner}}^{\prime}\right\},\\
%&\mathcal{Q}_{S_{3}}^{(1)}=\left\{(u, v) \in\{0,1\}^{n} \times\{0,1\}^{n}:(3, u, v) \in \mathcal{Q}_{S} \cup \mathcal{Q}_{S_{inner}}^{\prime}\right\}.\\
%\end{aligned}
%$$
%
%$$
%\begin{aligned}
%&U_{2}=\left\{u_{2} \in\{0,1\}^{n}:\left(2, u_{2}, v_{2}\right) \in \mathcal{Q}_{S_{2}}^{(1)}\right\}, \quad V_{2}=\left\{v_{2} \in\{0,1\}^{n}:\left(2, u_{2}, v_{2}\right) \in \mathcal{Q}_{S_{2}}^{(1)}\right\},\\
%&U_{3}=\left\{u_{3} \in\{0,1\}^{n}:\left(3, u_{3}, v_{3}\right) \in \mathcal{Q}_{S_{3}}^{(1)}\right\}, \quad V_{3}=\left\{v_{3} \in\{0,1\}^{n}:\left(3, u_{3}, v_{3}\right) \in \mathcal{Q}_{S_{3}}^{(1)}\right\}.
%\end{aligned}
%$$
%
%We will also denote
%%
%$$
%\beta_{i}=\left|\mathcal{Q}_{S_{i}}^{(1)}\right|-\left|\mathcal{Q}_{S_{i}}^{(0)}\right|=\left|\mathcal{Q}_{S_{i}}^{(1)}\right|-p.
%$$
%%
%for $i=2, 3$ the number of additional queries included in the extended transcript.



Then we define a predicate $\operatorname{Bad}\left(S_{1},S_{4}\right)$ on the pair $(S_1, S_4)$ as follows.


\begin{definition}
	\label{defn:bad-outer-4-rounds}
	
	Given $(S_1, S_4)$, the predicate $\operatorname{Bad}\left(S_{1},S_{4}\right)$ is fulfilled, if the corresponding induced set $\mathcal{Q}_{C}^{*}\left(S_{1}, S_{4}\right)$ fulfills at least one of the following nine ``collision'' conditions:
	\begin{itemize}
		\item[\cone]
		there exist $(a, b) \in \mathcal{Q}_{C}^{*}\left(S_{1}, S_{4}\right)$, $i, j \in\{1, \ldots, w\}$, $u_{2} \in U_{2}$ and $v_{3} \in V_{3}$ such that $(a \oplus k_1)[i] = u_2$ and $(b\xor T_{3}^{-1}(k_3))[j] = v_3$.
		\item[\ctwo]
		there exist $(a, b) \in \mathcal{Q}_{C}^{*}\left(S_{1}, S_{4}\right)$, $i, j \in\{1, \ldots, w\}$, $u_{2} \in U_{2}$ and $u_{3} \in U_{3}$ such that $(a \oplus k_1)[i] = u_2$ and $(T_{2}(\overline{S_2}(a \oplus k_1))\xor k_2)[j] = u_3$.
		\item[\cthree]
		there exist $(a, b) \in \mathcal{Q}_{C}^{*}\left(S_{1}, S_{4}\right)$, $i, j \in\{1, \ldots, w\}$, $v_{2} \in V_{2}$ and $v_{3} \in V_{3}$ such that $(b\xor T_{3}^{-1}(k_3))[j] = v_3$ and $\left(T_{2}^{-1}(b\xor T_{3}^{-1}(k_3)) \oplus T_{2}^{-1}(k_2)\right)[j] = v_2$.
		\item[\cfour]
		there exist $(a, b) \in \mathcal{Q}_{C}^{*}\left(S_{1}, S_{4}\right)$, distinct $i, i^{\prime}\in\{1, \ldots, w\}$, $u_{2},u_{2}' \in U_{2}^{(0)}$ such that
		$$(a \oplus k_1)[i] = u_2,\text{ and }
		(a \oplus k_1)[i'] = u_2'.$$
		\item[\cfive] \textcolor{red}{there exist distinct $(a, b),(a',b') \in \mathcal{Q}_{C}^{*}\left(S_{1}, S_{4}\right)$, distinct $i, i^{\prime}\in\{1, \ldots, w\}$, $u_{2} \in U_{2}$ such that}
		$$\left(T_1\left(a \oplus k_{1}\right)\right)[i] = u_2,\text{ and }
		\left(T_1\left(a \oplus k_{1}\right)\right)[i'] = \left(T_1\left(a' \oplus k_{1}\right)\right)[i'].$$
		\item[\csix]
		there exist $(a, b), (a^{\prime}, b^{\prime}) \in \mathcal{Q}_{C}^{*}\left(S_{1}, S_{4}\right)$, $i, i^{\prime},j, j^{\prime} \in\{1, \ldots, w\}$, with $(a,b, j) \neq \left(a^{\prime}, b',j^{\prime}\right)$, $u_{2}, u_{2}^{\prime} \in U_{2}$ such that $(a \oplus k_1)[i] = u_2, (a' \oplus k_1)[i'] = u_2'$, and
		%
		$$\big(T_2(\overline{S_2}(a\xor k_1))\xor k_2\big)[j]=\big(T_2(\overline{S_2}(a'\xor k_1))\xor k_2\big)[j'].
		$$
		%
		\item[\cseven]
		there exist $(a, b) \in \mathcal{Q}_{C}^{*}\left(S_{1}, S_{4}\right)$, distinct $i, i^{\prime}\in\{1, \ldots, w\}$, $v_{3},v_{3}' \in V_{3}$ such that
		$$\left(T_{3}^{-1}\left(b\right) \oplus k_{3}\right)[i] = v_3,\text{ and }
		\left(T_{3}^{-1}\left(b\right) \oplus k_{3}\right)[i'] = v_3'.$$
		\item[\ceight]
		\textcolor{red}{there exist distinct $(a, b),(a',b') \in \mathcal{Q}_{C}^{*}\left(S_{1}, S_{4}\right)$, distinct $i, i^{\prime}\in\{1, \ldots, w\}$, $v_{3} \in V_{3}$ such that}
		$$\left(T_{3}^{-1}\left(b\right) \oplus k_{3}\right)[i] = v_3,\text{ and }
		\left(T_{3}^{-1}\left(b\right) \oplus k_{3}\right)[i] =\left(T_{3}^{-1}\left(b'\right) \oplus k_{3}\right)[i'].$$
		\item[\cnine]
		there exist $(a, b), (a^{\prime}, b^{\prime}) \in \mathcal{Q}_{C}^{*}\left(S_{1}, S_{4}\right)$, $i, i^{\prime}, j, j^{\prime} \in\{1, \ldots, w\}$, with $(a,b, j) \neq \left(a',b^{\prime}, j^{\prime}\right)$, $v_{3},v_{3}^{\prime} \in V_{3}$ such that $\big(b \oplus T^{-1}(k_3)\big)[i] = v_3, \big(b' \oplus T^{-1}(k_3)\big)[i'] = v_3'$, and
		%
		$$\big(T^{-1}(\overline{S_3^{-1}}(b \oplus T^{-1}(k_3))\xor k_2)\big)[j]=\big(T^{-1}(\overline{S_3^{-1}}(b' \oplus T^{-1}(k_3))\xor k_2)\big)[j'].
		$$
	\end{itemize}
	Otherwise we say that $(S_{1}, S_{4})$ is good, we denote $\Pi_{good}$, resp. $\Pi_{bad}$ the set of good, resp. bad pairs of permutations $(S_{1}, S_{4})$ such that $S_{1} \vdash \mathcal{Q}_{S_{1}}, S_{4} \vdash \mathcal{Q}_{S_{4}}$.
\end{definition}




In all the following, we denote \emph{$\Pi$} the set of pairs of permutations $(S_{1}, S_{4})$ such that $S_{1} \vdash \mathcal{Q}_{S_{1}}^{(1)}, S_{4} \vdash \mathcal{Q}_{S_{4}}^{(1)}$. The next step is to bound $\operatorname{Pr}_{S_1,S_4}[(S_1,S_4) \in \Pi_{bad}]$.

\begin{lemma}
	\label{lemma:bad-outer-4-rounds}
	
	Assuming $p+wq\leq N/2$, then it holds
	\begin{align}
	 \operatorname{Pr}_{S_1,S_4}[(S_1,S_4) \in \emph{$\Pi_{bad}$}\mid (S_{1}, S_{4})\in\Pi]    
	\leq    \frac{9w^2 q (p+w q)^{2}}{N^2}+ \frac{16w^3q^2p}{N^2} .
	\label{eq:bound-bad-outer-4-rounds}
	\end{align}
\end{lemma}
\begin{proof}
We bound the probabilities of the nine conditions in turn.


\smallskip
\noindent \textsc{\cone}. Consider the probability that a construction query $(x,y)$ fulfills \cone. By $(T_1(\overline{S_1}(x\xor k_0))\xor k_1)[i]\in U_2$, it has to be $(x\xor k_0)[i_0]\notin U_1$ for any $i_0\in\{1,\ldots,w\}$, as otherwise it contradicts $\neg\btwo$. Thus conditioned on $S_{1} \vdash \mathcal{Q}_{S_{1}}$, the value of $S_1((x \oplus k_0)[i_0])$ remains uniform in $\{0, 1\}^{n} \backslash V_1^{(1)}$ for any fixed $i_0$. Because every entry in the $i_{0}$th column of $T_{1}$ is nonzero, the probability to have $(T_1(\overline{S_1}(x\xor k_0))\xor k_1)[i]=u_2$ is at most $1/(N-p-wq)$. Similarly, the probability of $(b\xor T_{3}^{-1}(k_3))[j] = v_3$ is at most $1/(N-p-wq)$. Since we have at most $q(p+wq)^2w^2$ choices for $(x,y)$ (or $(a, b)$), $i, j \in\{1, \ldots, w\}$, $u_{2} \in U_{2}$ and $v_{3} \in V_{3}$, we have
%
%
$$
\operatorname{Pr}\left[\cone\right] \leq \frac{w^{2} q (p+w q)^{2}}{(N-p-wq)^{2}}.
$$



\smallskip
\noindent \textsc{\ctwo and \cthree}. Consider the probability that a construction query $(x,y)$ fulfills \ctwo. Following the analysis of \cone, the probability to have $(T_1(\overline{S_1}(x\xor k_0))\xor k_1)[i]=u_2$ is at most $1/(N-p-wq)$. On the other hand, since $k_{2}$ is uniform and independent from the queries and from $k_{0}$, $k_{1}$, the probability to have $(T_{2}(\overline{S_2}(a \oplus k_1))\xor k_2)[j] = u_3$ is $1/N$. Since we have at most $q(p+wq)^2w^2$ choices for $(x,y)$ (or $(a, b)$), $i, j \in\{1, \ldots, w\}$, $u_{2} \in U_{2}$ and $u_{3} \in U_{3}$, we have
%
%
$$
\operatorname{Pr}\left[\ctwo\right] \leq \frac{w^{2} q (p+w q)^{2}}{N  (N-p-wq)}.
$$
%
Similarly,
%
$$
\operatorname{Pr}\left[\cthree\right] \leq \frac{w^{2} q (p+w q)^{2}}{N  (N-p-wq)}.
$$




\smallskip
\noindent \textsc{\cfour and \cseven}. For any of the $qw^2p^2/2$ choices of $(a, b) \in \mathcal{Q}_{C}^{*}\left(S_{1}, S_{4}\right)$, distinct $i, i^{\prime}\in\{1, \ldots, w\}$, and $u_{2},u_{2}' \in U_{2}^{(0)}$, the probability to have $(a \oplus k_1)[i] = u_2$ and $(a \oplus k_1)[i'] = u_2'$ is $1/N^2$, since $k_1$ is uniform and independent of $S_1$. By these, we have
%
$$
\operatorname{Pr}\left[\cfour\right] \leq \frac{w^2qp^2}{2N^2}.
$$
Similarly, using the uniformness of $k_3$, we have
%
$$
\operatorname{Pr}\left[\cseven\right] \leq \frac{w^2qp^2}{2N^2}.
$$




%
%\smallskip
%\noindent\textsc{\cfive and \ceight}. Consider any of the $qw^2p^2/2$ choices of $(a, b),(a',b') \in \mathcal{Q}_{C}^{*}\left(S_{1}, S_{4}\right)$, distinct $i, i^{\prime}\in\{1, \ldots, w\}$, and $u_{2} \in U_{2}$.
%
%
%By $\neq\btwo$, to have $\left(T_1\left(a \oplus k_{1}\right)\right)[i] = u_2$, it has to be that $(x\xor k_0)[i_0]\notin U_1$ for any $i_0$.
%
%
%There exists $i_0$ such that $(x\xor k_0)[i_0]\neq(x'\xor k_0)[i_0]$.










%$$\left(T_1\left(a \oplus k_{1}\right)\right)[i] = u_2,\text{ and }
%\left(T_1\left(a \oplus k_{1}\right)\right)[i'] = \left(T_1\left(a' \oplus k_{1}\right)\right)[i'].$$



We have the value $\left(S_{1}\left(x \oplus k_{0}\right)\right)[i]$ remain uniform in $\{0, 1\}^{n} \verb|\| (\mathcal{Q}_{S_{1}} \cup \mathcal{Q}_{S_{4}})$.  Because of the fact that there exists $i$ with $x[i]\neq x'[i]$, that is there exist $i$ with $\left(a \oplus k_{1}\right)[i'] \neq \left(a' \oplus k_{1}\right)[i']$. So,

$$
\operatorname{Pr}\left[\tau_{inner} \in \Theta_{5}\right] \leq \frac{w^{2} q^{2} (p+w q)}{(N-p)^2}.
$$





\smallskip
\noindent\textsc{Conditions \csix and \cnine}. Consider \csix first. For any choice of $(a, b), (a^{\prime}, b^{\prime}) \in \mathcal{Q}_{C}^{*}\left(S_{1}, S_{4}\right)$, $i, i^{\prime},j, j^{\prime} \in\{1, \ldots, w\}$, with $(a, j) \neq \left(a^{\prime}, j^{\prime}\right)$, $u_{2}, u_{2}^{\prime} \in U_{2}$, denote by $\coll(a,a',i,i',j,j',u_2,u_2')$ the event that
%
%
$$(a \oplus k_1)[i] = u_2\bigwedge(a' \oplus k_1)[i'] = u_2'\bigwedge\big(T_2(\overline{S_2}(a\xor k_1))\xor k_2\big)[j]=\big(T_2(\overline{S_2}(a'\xor k_1))\xor k_2\big)[j'].
$$
%
%
With this, we derive the probability via the following calculations.
%
%
\begin{align*}
 \Pr[\csix]  
=   &  \underbrace{\sum_{(a,b),(a',b')}\sum_{i,i'}\sum_{(u_2,v_2),(u_2',v_2')}\sum_{j\neq j'}\Pr[\coll(a,a',i,i',j,j',u_2,u_2')]}_{A_1}    \\
&  + \underbrace{\sum_{(a,b)\neq(a',b')}\sum_{i,i'}\sum_{(u_2,v_2)}\sum_{j}\Pr[\coll(a,a',i,i',j,j',u_2,u_2')]}_{A_2}      \\
  &  +  \underbrace{\sum_{(a,b)\neq(a',b')}\sum_{i\neq i'}\sum_{(u_2,v_2)\neq(u_2',v_2')}\sum_{j}\Pr[\coll(a,a',i,i',j,j',u_2,u_2')]}_{A_3}       \\
    &  + \underbrace{\sum_{(a,b)\neq(a',b')}\sum_{i}\sum_{(u_2,v_2)\neq(u_2',v_2')}\sum_{j}\Pr[\coll(a,a',i,i',j,j',u_2,u_2')]}_{A_4}   .
\end{align*}



Regarding the term $A_1$, i.e., $j\neq j'$, we have
%
$$\Pr\bigg[\Big(\big(T_2(\overline{S_2}(a\xor k_1))\xor k_2\big)\xor k_3\Big)[j]=\Big(\big(T_2(\overline{S_2}(a'\xor k_1))\xor k_2\big)\xor k_3\Big)[j']\bigg]=\frac{1}{N}$$
simply by that $k_2[j]$ and $k_2[j']$ are uniform and independent. Moreover, as argued before, the probability to have $(T_1(\overline{S_1}(x\xor k_0))\xor k_1)[i]=u_2$ is at most $1/(N-p-wq)$. By this,
%
$$\Pr\big[A_1\big]\leq\frac{w^3q^2p^2}{N(N-p-wq)^2}\leq\frac{w^3q^2p}{N(N-p-wq)}.$$



Regarding the term $A_2$, since the number of choices for $(a, b), (a^{\prime}, b^{\prime}) \in \mathcal{Q}_{C}^{*}\left(S_{1}, S_{4}\right)$, $i, i^{\prime},j, j^{\prime} \in\{1, \ldots, w\}$, with $(a, j) \neq \left(a^{\prime}, j^{\prime}\right)$, $u_{2}=u_{2}^{\prime} \in U_{2}$ is at most $w^4q^2p$, we have
%
$$\Pr\big[A_2\big]\leq\frac{w^4q^2p}{N(N-p-wq)}.$$








Regarding the term $A_3$, since $(a\xor k_1)[i]\in U_2^{(0)}$, it holds $(a\xor k_1)[i']\notin U_2^{(0)}$ by $\neg\cfour$. Therefore, a random output $v_2^*$ will be sampled during the extension process, so that $((a\xor k_1)[i'],v_2^*)\in\mathcal{Q}_{S_2}^{(1)}$. Conditioned on the values in $V_2^{(0)}$ (which includes $v_2'$), $v_2^*$ is uniform in {\it at least} $N-p-wq$ possibilities. This means the probability to have $T(\overline{S_2}(a\xor k_1))[j]=T(\overline{S_2}(a'\xor k_1))[j]$ is at most $1/(N-p-wq)$, as every entry in the $i'$th column of $T$ is nonzero. Moreover, as argued before, the probability to have $(T_1(\overline{S_1}(x\xor k_0))\xor k_1)[i]=u_2$ is at most $1/(N-p-wq)$. By this,
%
$$\Pr\big[A_3\big]\leq\frac{w^3q^2p^2}{N(N-p-wq)^2}\leq\frac{w^3q^2p}{N(N-p-wq)}.$$




Finally, consider the term $A_4$. Assume that $\overline{S_2}(a\xor k_1)=\bfv_1\|v_1\|\bfv_2$ and
$\overline{S_2}(a\xor k_1)=\bfv_1'\|v_1'\|\bfv_2'$ after the extension process.

Then the equality $T(\overline{S_2}(a\xor k_1))[j]=T(\overline{S_2}(a'\xor k_1))[j]$ implies
%
%
\begin{align}
\bft_1^*\cdot\bfv_1\xor t^*\xor v_1\xor\bft_2^*\cdot\bfv_2=\bft_1^*\cdot\bfv_1'\xor t^*\xor v_1'\xor\bft_2^*\cdot\bfv_2'.
\label{eq:interm-eq-c5}
\end{align}
%
%
for two vectors $\bft_1^*,\bft_2^*$ and $t^*\in\{0,1\}^n$. Now:
\begin{itemize}
	\item If $\bfv_1=\bfv_1'$ and $\bfv_2=\bfv_2'$, then Eq. (\ref{eq:interm-eq-c5}) collapses to $t^*\xor v_1=t^*\xor v_1'$ which is not possible;
	\item Else, Eq. (\ref{eq:interm-eq-c5}) holds with probability at most $1/(N-p-wq)$.
\end{itemize}
Moreover, as argued before, the probability to have $(T_1(\overline{S_1}(x\xor k_0))\xor k_1)[i]=u_2$ is at most $1/(N-p-wq)$. By this,
%
$$\Pr\big[A_4\big]\leq\frac{w^3q^2p^2}{N(N-p-wq)^2}\leq\frac{w^3q^2p}{N(N-p-wq)}.$$
%
%

Summing over the above, we reach
%
$$\Pr[\cfive]\leq\frac{4w^3q^2p}{N(N-p-wq)}.$$
%

The analysis of \cseven is similar by symmetry, resulting in
%
%
$$\Pr[\cseven]\leq\frac{4w^3q^2p}{N(N-p-wq)}.$$
%



\arrangespace


By the above, we have
%
\begin{align*}
&  \operatorname{Pr}_{S_1,S_4}[(S_1,S_4) \in \emph{$\Pi_{bad}$}\mid (S_{1}, S_{4})\in\Pi]  \leq \sum_{i=1}^7\Pr[\ci]       \\
\leq  & \frac{w^2 q (p+w q)^{2}}{(N-p-wq)^{2}}+
\frac{2w^2 q (p+w q)^{2}}{N  (N-p-wq)}+
\frac{w^2qp^2}{N^2}+
\frac{8w^3q^2p}{N(N-p-wq)}      \\
\leq  & \frac{9w^2 q (p+w q)^{2}}{N^2}+ \frac{16w^3q^2p}{N^2} .
\end{align*}
%
as claimed.       \qed
\end{proof}





\paragraph{\textsc{The inner two rounds.}}


We are now ready for the second step of the reasoning. Let
$\calS^*=(S_1,S_2,S_3,S_4)$, and define     {\small
	$$\textsf{p}(\tau,S_1,S_4)=\Pr[\calS^*\xleftarrow{\$}\mathsf{Perm}(n)^4:\spn_{\bfk}^T[\calS^*]\vdash\mathcal{Q}_C^*(S_1,S_4)
	\mid S_i\vdash\mathcal{Q}_{S_i},i=1,2,3,4].$$
}%
This captures the probability that the inner two SPN rounds ``extend'' the tuples in $\mathcal{Q}_C^*(S_1,S_4)$. The probability
$\mathsf{p}_{2}\left(\mathcal{Q}_{C} | \mathcal{Q}_{S}\right)$ can be related to it.

\begin{lemma}
	\label{lemma:reduce-to-two-rounds}
	
	It holds
	\begin{align}
	\frac{\mathsf{p}_{2}\left(\mathcal{Q}_{C} | \mathcal{Q}_{S}\right)}{\mathsf{p}_{1}\left(\mathcal{Q}_{C} | \mathcal{Q}_{S}\right)}
	\geq  &  (N^w)_q{\mathsf{p}(\tau,S_1,S_4)}  - \operatorname{Pr}_{S_1,S_4}[(S_1,S_4) \in \emph{$\Pi_{bad}$}\mid (S_{1}, S_{4})\in\Pi].
	\label{eq:function-dependent-bound-on-ratio}
	\end{align}
%	Then we have
%	\begin{align*}
%	\frac{{\Pr}_{re}(\tau,\kvector)}{{\Pr}_{id}(\tau,\kvector)}
%	\geq 1 & -\Pr[\textsf{Bad}(\mathbf{F}_1,\mathbf{F}_6)  \mid
%	\mathbf{F}_1\vdash\mathcal{Q}_{F_1},\mathbf{F}_6\vdash\mathcal{Q}_{F_6}]  \\
%	&  - \mathbb{E}_{\mathbf{F}_1,\mathbf{F}_6}[\epsilon(\mathbf{F}_1,\mathbf{F}_6,k)\mid
%	\mathbf{F}_1\vdash\mathcal{Q}_{F_1},\mathbf{F}_6\vdash\mathcal{Q}_{F_6}].
%	\end{align*}
%	
%	
%	
%	
%	Assume that there exists a function
%	$\epsilon:\mathsf{Perm}(n)^2\times\mathcal{K}^5\rightarrow[0,\infty)$ such that for any good $(S_1,S_4)$, it holds
%	\begin{align}
%	
%	\frac{\mathsf{p}_{2}\left(\mathcal{Q}_{C} | \mathcal{Q}_{S}\right)}{\mathsf{p}_{1}\left(\mathcal{Q}_{C} | \mathcal{Q}_{S}\right)}
%	\geq 1 - xxx.
%	
%	{\mathsf{p}(\tau,S_1,S_4)}\bigg/{\prod_{i=0}^{q-1}\left(\frac{1}{N^2-i}\right)}\geq1-\epsilon(\mathbf{F}_1,\mathbf{F}_6,k).
%	\label{ieq:function-dependent-bound-on-ratio}
%	\end{align}
%	Then we have
%	\begin{align*}
%	\frac{{\Pr}_{re}(\tau,\kvector)}{{\Pr}_{id}(\tau,\kvector)}
%	\geq 1 & -\Pr[\textsf{Bad}(\mathbf{F}_1,\mathbf{F}_6)  \mid
%	\mathbf{F}_1\vdash\mathcal{Q}_{F_1},\mathbf{F}_6\vdash\mathcal{Q}_{F_6}]  \\
%	&  - \mathbb{E}_{\mathbf{F}_1,\mathbf{F}_6}[\epsilon(\mathbf{F}_1,\mathbf{F}_6,k)\mid
%	\mathbf{F}_1\vdash\mathcal{Q}_{F_1},\mathbf{F}_6\vdash\mathcal{Q}_{F_6}].
%	\end{align*}
\end{lemma}
\begin{proof}
	Define
	$\mathsf{p}(S_1,S_4)\xlongequal{\text{def}}\Pr[(S_1^*,S_4^*)\xleftarrow{\$}\mathsf{Perm}(n)^2:(S_1^*,S_4^*)=(S_1,S_4)]$ for convenience. Then, clearly, once $S_1$ and $S_4$ are fixed
	such that $S_1\vdash\mathcal{Q}_{S_1}$ and
	$S_4\vdash\mathcal{Q}_{S_4}$, the event
	$\spn_{\bfk}^T\vdash\mathcal{Q}_C$ is
	equivalent to
	$\spn_{\bfk}^T\vdash\mathcal{Q}_C^*(\mathbf{F}_1,\mathbf{F}_6)$.
	Hence
	\begin{align*}
	{{\Pr}_{re}(\tau,\kvector)}
	&  \geq\sum_{\mathbf{F}_1\vdash\mathcal{Q}_{F_1},\mathbf{F}_6\vdash\mathcal{Q}_{F_6}
		:(\mathbf{F}_1,\mathbf{F}_6)\text{ good}} \textsf{p}(\mathbf{F}_1,\mathbf{F}_6) \cdot
	\frac{\textsf{p}(\tau,\mathbf{F}_1,\mathbf{F}_6)}{|\mathcal{K}|\cdot N^{4q_f}}  .
	\end{align*}
	Therefore,
	\begin{align*}
	\frac{{{\Pr}_{re}(\tau,\kvector)}}{{{\Pr}_{id}(\tau,\kvector)}}
	&  \geq \frac{\sum_{\mathbf{F}_1\vdash\mathcal{Q}_{F_1},\mathbf{F}_6\vdash\mathcal{Q}_{F_6}
			:(\mathbf{F}_1,\mathbf{F}_6)\text{ good}} \textsf{p}(\mathbf{F}_1,\mathbf{F}_6) \cdot
		\textsf{p}(\tau,\mathbf{F}_1,\mathbf{F}_6)}{   \Pr[\mathbf{F}_1\vdash\mathcal{Q}_{F_1},\mathbf{F}_6\vdash\mathcal{Q}_{F_6}]
		\cdot \prod_{i=0}^{q_e-1}\frac{1}{N^2-i}   }     \\
	&  \geq   \frac{\sum_{\mathbf{F}_1\vdash\mathcal{Q}_{F_1},\mathbf{F}_6\vdash\mathcal{Q}_{F_6}
			:(\mathbf{F}_1,\mathbf{F}_6)\text{ good}} \textsf{p}(\mathbf{F}_1,\mathbf{F}_6)  (1-\epsilon(\mathbf{F}_1,\mathbf{F}_6,k))}{\Pr[\mathbf{F}_1\vdash\mathcal{Q}_{F_1},\mathbf{F}_6\vdash\mathcal{Q}_{F_6}]} \text{ } (\text{by }(\ref{ieq:function-dependent-bound-on-ratio}))      \\
	&  \geq 	 1-\Pr[\textsf{Bad}(\mathbf{F}_1,\mathbf{F}_6)\mid
	\mathbf{F}_1\vdash\mathcal{Q}_{F_1},\mathbf{F}_6\vdash\mathcal{Q}_{F_6}]      \\
	& \textcolor{white}{\geq11} -\underbrace{  \frac{\sum_{\mathbf{F}_1\vdash\mathcal{Q}_{F_1},\mathbf{F}_6\vdash\mathcal{Q}_{F_6}} \textsf{p}(\mathbf{F}_1,\mathbf{F}_6) \epsilon(\mathbf{F}_1,\mathbf{F}_6,k)}{\Pr[\mathbf{F}_1\vdash\mathcal{Q}_{F_1},\mathbf{F}_6\vdash\mathcal{Q}_{F_6}]}.}_{\leq\sum_{\mathbf{F}_1,\mathbf{F}_6}\textsf{p}(\mathbf{F}_1,\mathbf{F}_6\mid\mathbf{F}_1\vdash\mathcal{Q}_{F_1},\mathbf{F}_6\vdash\mathcal{Q}_{F_6})\epsilon(\mathbf{F}_1,\mathbf{F}_6,k)= \mathbb{E}_{\mathbf{F}_1,\mathbf{F}_6}[\epsilon(\mathbf{F}_1,\mathbf{F}_6,k)\mid
		\mathbf{F}_1\vdash\mathcal{Q}_{F_1},\mathbf{F}_6\vdash\mathcal{Q}_{F_6}]} 
	\end{align*}
	as claimed.         \qed
\end{proof}




We appeal to~\cite{C:CDKLST18} to have a concrete bound on Eq. (\ref{eq:function-dependent-bound-on-ratio}).

\begin{lemma}
	\label{lemma:bound-middle-two-rounds}
	
	Assume $p+wq\leq N/2$, then
	\begin{align}
	(N^w)_q{\mathsf{p}(\tau,S_1,S_4)} \geq 1-\frac{q^2}{N^w}-\frac{q(2wp+6w^2q)^2}{N^2}.
	\label{eq:bound-on-epsilon-middle}
	\end{align}
\end{lemma}
\begin{proof}
It can be checked that, the transcript $\mathcal{Q}_C^*(S_1,S_4)=(\mathcal{Q}_C,\mathcal{Q}_{S_2},\mathcal{Q}_{S_3})$ satisfies exactly the conditions defining a good transcript as per~\cite[page 740]{C:CDKLST18}. Moreover,
the ratio ${\mathsf{p}(\tau,S_1,S_4)}/(1/(N^w)_q)$ is exactly the ratio of the probabilities to get $\tau$ in the real and in the ideal world once a good pair $(S_1,S_4)$ is fixed. The result thus immediately follows from~\cite[Lemma 9]{C:CDKLST18}.       \qed
\end{proof}




%
%We define $\mathrm{p}\left(S_{1}, S_{4}\right) = \operatorname{Pr}\left[\left(S_{1}^{*}, S_{4}^{*}\right)
%\stackrel{\mathrm{S}}{\leftarrow}(\mathcal{S}(n))^{2}: \left(S_{1}^{*}, S_{4}^{*}\right)=\left(S_{1}, S_{4}\right)\right]$ for convenience. Next we need to consider  $\frac{\operatorname{Pr}_{r e}(\tau, k)}{\operatorname{Pr}_{i d}(\tau, k)}$. Clearly, once $S_{1}$ and $S_{4}$ are fixed such that $S_{1} \vdash \mathcal{Q}_{S_{1}}$ and $S_{4} \vdash \mathcal{Q}_{S_{4}}$, the event
%$\mathrm{SP}_{k}[\mathcal{S}] \vdash \mathcal{Q}_{C}$ is equivalent to $\mathrm{SP}_{k}^{S^{*}} \vdash \mathcal{Q}_{C}^{*}\left(S_{1}, S_{4}\right)$. Hence,
%
%$$
%\operatorname{Pr}_{r e}(\tau, k) \geq \sum_{S_{1} \vdash \mathcal{Q}_{S_{1}}, S_{4} \vdash \mathcal{Q}_{S_{4}}:\left(S_{1}, S_{4}\right) good }
%\mathrm{p}\left(S_{1}, S_{4}\right) \cdot \mathrm{p}\left(\tau, S_{1}, S_{4}\right) \cdot \mathrm{p}\left(S_{2} \vdash \mathcal{Q}_{S_{2}},S_{3} \vdash \mathcal{Q}_{S_{3}}\right).
%$$
%
%\noindent Because of $\operatorname{Pr} (\mathcal{P}  \vdash \mathcal{Q}_{C}) \le \frac{1}{\left(2^{w n}\right)_{q}}$. Therefore,\\
%
%\noindent \textbf{Lemma 7} \emph{Once $S_{1}$ and $S_{4}$ are fixed such that $S_{1} \vdash \mathcal{Q}_{S_{1}}$ and $S_{4} \vdash \mathcal{Q}_{S_{4}}$, we have}
%
%\begin{equation}
%\begin{aligned}
%\frac{\operatorname{Pr}_{r e}(\tau, k)}{\operatorname{Pr}_{i d}(\tau, k)} &\geq \frac{\sum_{S_{1} \vdash \mathcal{Q}_{S_{1}}, S_{4} \vdash \mathcal{Q}_{S_{4}}: \left(S_{1}, S_{4}\right)good } \mathrm{p}\left(S_{1}, S_{4}\right) \cdot \mathrm{p}\left(\tau, S_{1}, S_{4}\right)}{\operatorname{Pr}(S_{1} \vdash \mathcal{Q}_{S_{1}}, S_{4} \vdash \mathcal{Q}_{S_{4}}) \cdot \frac{1}{\left(2^{w n}\right)_{q}}}\\
%&\geq \left(1-\operatorname{Pr}\left[\operatorname{Bad}\left(S_{1},S_{4}\right) | S_{1} \vdash \mathcal{Q}_{S_{1}}, S_{4} \vdash \mathcal{Q}_{S_{4}}\right]\right) \cdot
%\frac{\mathrm{p}\left(\tau, S_{1}, S_{4}\right)}{\frac{1}{\left(2^{w n}\right)_{q}}}.
%\end{aligned}
%\end{equation}
%
%\noindent \emph{Proof:} For any attainable transcript $\tau_{inner}=\left(\mathcal{Q}_{C}^{*}\left(S_{1},S_{4}\right), \mathcal{Q}_{S_{inner}}, \mathcal{Q}_{S_{inner}}^{\prime}, \mathbf{k}\right)$, let
%
%$$
%\begin{aligned}
%&\mathrm{p}_{1}\left(\mathcal{Q}_{C} | \mathcal{Q}_{S}\right)=\operatorname{Pr}\left[\widetilde{\mathcal{P}} \stackrel{s}{\leftarrow} \widetilde{\operatorname{Perm}}(\mathcal{T}, w n)^{\ell}, \mathcal{S} \stackrel{s}{\leftarrow} \operatorname{Perm}(n)^{r}: \tilde{\mathcal{P}} \vdash \mathcal{Q}_{C} | \mathcal{S} \vdash \mathcal{Q}_{S}\right],\\
%&\mathrm{p}_{2}\left(\mathcal{Q}_{C} | \mathcal{Q}_{S}\right)=\operatorname{Pr}\left[k_{1}, \ldots, k_{\ell} \leftarrow \mathcal{K}^{r+1}, \mathcal{S} \stackrel{s}{\leftarrow} \operatorname{Perm}(n)^{r}:\left(\mathrm{SP}_{k_{j}}[\mathcal{S}]\right)_{j} \vdash \mathcal{Q}_{C} | \mathcal{S} \vdash \mathcal{Q}_{S}\right].
%\end{aligned}
%$$
%We...    
%



For any extended transcript $\tau_{inner}^{\prime}=\left(\mathcal{Q}_{C}^{*}\left(S_{1}, S_{4}\right), \mathcal{Q}_{S_{inner}}, \mathcal{Q}_{S_{inner}}^{\prime}, \mathbf{k}\right)$, let us denote

$$
\begin{aligned}
&\operatorname{p} =  \operatorname{Pr}\left[\left(\mathbf{k}, \mathcal{S}\right) \stackrel{s}{\leftarrow} \mathcal{K}^{3} \times \operatorname{Perm}(n)^{2}:
\left(\mathcal{S} \vdash \mathcal{Q}_{S_{inner}} \cup \mathcal{Q}_{S_{inner}}^{\prime}\right) \wedge \left(\operatorname{SP}_{\mathbf{k}}[\mathcal{S}] \vdash \mathcal{Q}_{C}^{*}\left(S_{1},S_{4}\right)\right) \wedge\mathbf{k}\right]\\
&\operatorname{p}\left(\tau_{inner}^{\prime}\right) =\operatorname{Pr}\left[\mathcal{S} \stackrel{\mathbf{s}}{\leftarrow} \operatorname{Perm}(n)^{2}: \operatorname{SP}_{\mathbf{k}}[\mathcal{S}] \vdash \mathcal{Q}_{C}^{*}\left(S_{1},S_{4}\right) |\left(S_{2} \vdash \mathcal{Q}_{S_{2}}^{(1)}\right) \wedge\left(S_{3} \vdash \mathcal{Q}_{S_{3}}^{(1)}\right)\right].
\end{aligned}
$$

we will write $(r)_{s} = \frac{r!}{(r-s)!}$. Note that one has

$$
\begin{aligned}
&\operatorname{Pr}\left[(\mathcal{P}, \mathcal{S}) \stackrel{s}{\leftarrow} \mathcal{\operatorname{Perm}}(w n) \times \operatorname{Perm}(n)^{2}:\left(\mathcal{S} \vdash \mathcal{Q}_{S_{inner}}\right) \wedge\left(\operatorname{P} \vdash \mathcal{Q}_{C}^{*}\left(S_{1},S_{4}\right)\right)\right] \\
&\leq  \frac{1}{\left(2^{w n}\right)_{q}\left(2^{n}\right)_{p}\left(2^{n}\right)_{p}}. \\
&\operatorname{Pr}\left[(\mathbf{k}^{\prime}, \mathcal{S}) \stackrel{s}{\leftarrow} \mathcal{K}^{3} \times \operatorname{Perm}(n)^{2}:\left(\mathcal{S} \vdash \mathcal{Q}_{S_{inner}}\right) \wedge\left(\operatorname{SP}_{\mathbf{k}}[\mathcal{S}] \vdash \mathcal{Q}_{C}^{*}\left(S_{1},S_{4}\right)\right)\right] \\
& \geq \sum_{\tau_{inner}^{\prime} \in \Theta_{\mathrm{Good}}(\tau_{inner})} \operatorname{p}\\
&\geq \sum_{\tau_{inner}^{\prime} \in \Theta_{\mathrm{Good}}(\tau_{inner})} \frac{1}{|\mathcal{K}|^{3}\left(2^{n}\right)_{p+\beta_{2}}\left(2^{n}\right)_{p+\beta_{3}}} \mathrm{p}\left(\tau_{inner}^{\prime}\right).
\end{aligned}
$$

Define

$$
\begin{aligned}
\mathrm{p}_{1}=\operatorname{Pr}\left[\operatorname{P} \stackrel{s}{\leftarrow} \operatorname{Perm}(w n), \mathcal{S} \stackrel{s}{\leftarrow} \operatorname{Perm}(n)^{2}: \widetilde{\operatorname{P}} \vdash \mathcal{Q}_{C}^{*}\left(S_{1},S_{4}\right) | \mathcal{S}_{inner} \vdash \mathcal{Q}_{S_{inner}}\right],\\
\mathrm{p}_{2}=\operatorname{Pr}\left[\mathbf{k}^{\prime} \stackrel{s}{\leftarrow} \mathcal{K}^{3}, \mathcal{S} \stackrel{s}{\leftarrow} \operatorname{Perm}(n)^{r}:\left(\mathrm{SP}_{\mathbf{k}}[\mathcal{S}]\right) \vdash \mathcal{Q}_{C}^{*}\left(S_{1},S_{4}\right) | \mathcal{S}_{inner} \vdash \mathcal{Q}_{S_{inner}}\right].
\end{aligned}
$$

Then

$$
\begin{aligned}
&\mathrm{p}_{1} \leq \left(2^{w n}\right)_{p},\\
&\mathrm{p}_{2} \geq  \sum_{\tau_{inner}^{\prime} \in \Theta_{\mathrm{Good}}(\tau_{inner})} \frac{1}{|\mathcal{K}|^{3}\left((2^{n}-p)\right)_{\beta_{2}}\left((2^{n}-p)\right)_{\beta_{3}}} \mathrm{p}\left(\tau_{inner}^{\prime}\right).
\end{aligned}
$$

Thus one has

$$
\begin{aligned}
\frac{\mathrm{p}_{2}}{\mathrm{p}_{1}} & \geq \frac{\left(2^{w n}\right)_{p}}{\mathcal{K}|^{3}\left((2^{n}-p)\right)_{\beta_{2}}\left((2^{n}-p)\right)_{\beta_{3}}} \mathrm{p}\left(\tau_{inner}^{\prime}\right)\\
& \geq \min _{\tau_{inner}^{\prime} \in \Theta_{\text {good }}(\tau_{inner})}\left(\left(2^{w n}\right)_{q} \mathrm{p}\left(\tau_{inner}^{\prime}\right)\right) \sum_{\tau_{inner}^{\prime} \in \Theta_{\text {good }}(\tau_{inner})} \frac{1}{|\mathcal{K}|^{3}\left(2^{n}-p\right)_{\beta_{2}}\left(2^{n}-p\right)_{\beta_{3}}}.
\end{aligned}
$$

Note that the weighted sum $\sum_{\tau_{inner}^{\prime} \in \Theta_{\text {good }}(\tau_{inner})} \frac{1}{|\mathcal{K}|^{3}\left(2^{n}-p\right)_{\beta_{2}}\left(2^{n}-p\right)_{\beta_{3}}}$ corresponds exactly to the probability that a random inner extended transcript is good when it is sampled as follows:

\begin{itemize}
  \item[1.]
  choose keys $k_{1}, k_{3} \in \mathcal{K}$  uniformly and independently at random;
  \item[2.]
  choose the partial extension of the S-box queries based on the new collisions $\mathcal{Q}_{S_{inner}}^{\prime}$ uniformly at random (meaning that each possible $\mathnormal{u}$ or $\mathnormal{v}$ is chosen uniformly at random in the set of its authorized values);
  \item[3.]
  finally choose $k_{2} \in \mathcal{K}$ uniformly at random, independently from everything else.
\end{itemize}

Thus, the exact probability of observing the inner extended transcript $\tau_{inner}^{\prime}$ is

$$
\begin{aligned}
\frac{1}{|\mathcal{K}|^{3}\left(2^{n}-p\right)_{\beta_{2}}\left(2^{n}-p\right)_{\beta_{3}}}.
\end{aligned}
$$

and we have

$$
\begin{aligned}
\sum_{\tau_{inner}^{\prime} \in \Theta_{\text {good }}(\tau_{inner})} \frac{1}{|\mathcal{K}|^{3}\left(2^{n}-p\right)_{\beta_{2}}\left(2^{n}-p\right)_{\beta_{3}}} = \operatorname{Pr}\left[ \tau_{inner}^{\prime} \in \Theta_{\text {good }}(\tau_{inner})\right].
\end{aligned}
$$

One finally gets

\begin{equation}
\begin{aligned}
\frac{\mathrm{p}_{2}}{\mathrm{p}_{1}} \geq \operatorname{Pr}\left[ \tau_{inner}^{\prime} \in \Theta_{\text {good }}(\tau_{inner})\right] \cdot \min _{\tau_{inner}^{\prime} \in \Theta_{\text {good }}}((2^{w n})_{q} p(\tau_{inner}^{\prime})).
\end{aligned}
\end{equation}

The previous proof is conditioned on $S_{1} \vdash \mathcal{Q}_{S_{1}}, S_{4} \vdash \mathcal{Q}_{S_{4}}$, but $\operatorname{Pr}\left[ \tau_{inner}^{\prime} \in \Theta_{\text {good }}(\tau_{inner})\right]$, we need to consider $S_{1} \vdash \mathcal{Q}_{S_{1}}^{(1)}, S_{4} \vdash \mathcal{Q}_{S_{4}}^{(1)}$. That is the probability $\left(T_{1}\left(S_{1}\left(x \oplus k_{0}\right) \oplus k_{1}\right)\right)[i]=u_2$ or $\left(T_{3}^{-1}\left(S_{4}^{-1}\left(y \oplus k_{4}\right)\right) \oplus k_{3}\right)[j]=v_3$ hold is at most $\frac{1}{(N-p-w q)}$, so

\begin{equation}
\begin{aligned}
\operatorname{Pr}\left[ \tau_{inner}^{\prime} \in \Theta_{\text {good }}(\tau_{inner})\right] \geq 1&- \frac{2 w^{2} q (p+w q)^{2}}{(N-p-w q)} -\frac{2 w^{2} q (p+w q)(p+w q+2 q)}{N \cdot (N-p-w q)}\\
&- \frac{w^{2} q (p+w q)(p+w q+2 q)}{(N-p-w q)^2} - \frac{2 w^{2} q^{2} (p+w q)}{(N- p- wq)^2}.
\end{aligned}
\end{equation}




Combining all of (9), (10), we can obtain
$$
\begin{aligned}
\frac{\mathrm{p}\left(\tau, S_{1}, S_{4}\right)}{\left(2^{w n}\right)_{q}} &\geq (1-\frac{q^{2}}{2^{w n}}-\frac{q\left(2 w p+6 w^{2} q\right)^{2}}{2^{2 n}})\\
&\cdot (1- \frac{2 w^{2} q (p+w q)^{2}}{(N-p-w q)} -\frac{2 w^{2} q (p+w q)(p+w q+2 q)}{N \cdot (N-p-w q)}\\
&- \frac{w^{2} q (p+w q)(p+w q+2 q)}{(N-p-w q)^2} - \frac{2 w^{2} q^{2} (p+w q)}{(N- p- wq)^2})\\
&\geq 1- \frac{2 w^{2} q (p+w q)^{2}}{(N-p-w q)} -\frac{2 w^{2} q (p+w q)(p+w q+2 q)}{N \cdot (N-p-w q)}\\
&- \frac{w^{2} q (p+w q)(p+w q+2 q)}{(N-p-w q)^2} - \frac{2 w^{2} q^{2} (p+w q)}{(N- p- wq)^2}\\
& -\frac{q^{2}}{2^{w n}}-\frac{q\left(2 w p+6 w^{2} q\right)^{2}}{2^{2 n}}\\
&\geq 1- \frac{4 w^2 q(p+wq)^2}{N} - \frac{8 w^2 q(p+w q)(p+w q +2 q)}{N^2}\\
&- \frac{8 w^2 q^2(p+w q)}{N^2} - \frac{q^{2}}{2^{w n}}-\frac{q\left(2 w p+6 w^{2} q\right)^{2}}{2^{2 n}}\\
&= 1 - \frac{q^2}{N^w} - \frac{4 w^2 q(p+wq)^2}{N}\\
&- \frac{8 w^2 q(p+w q)(p+w q +3 q)+4 w^2 q(p+3 wq)^2}{N^2}.
\end{aligned}
$$

\noindent then combine (7), (8), we can obtain

$$
\begin{aligned}
\operatorname{Adv}_{\mathcal{C}}\left(p, q\right) &\leq \frac{q^2}{2^{n w}} + \frac{8 w^2 q(p+wq)^2+w^2 q}{2^n}\\
&+ \frac{16 w^2 q(p+w q)(p+w q +3 q)+4 w^2 q(p+3 wq)^2+ w^2q(p+w q)(3p+w q)}{2^{2 n}}.
\end{aligned}
$$




