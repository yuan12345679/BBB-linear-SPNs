
\section{SPRP Security of 4-Round SPNs}
\label{section:security of 4-round SPNs}

In this section, we prove beyond-birthday-bound SPRP security for 4-round linear SPNs. Concretely, let $\spn_{\bfk}[\mathcal{S}]$ be the 4-round SPN using any linear transformations $T$. I.e.,
%
\begin{align}
\spn_{\bfk}^T[\mathcal{S}](x):=k_4\xor\overline{S_4}(k_3\xor T(\overline{S_3}(k_2\xor T(\overline{S_2}(k_1\xor T(\overline{S_1}(k_0\xor x))))))).
\label{eq:defn-4-round-spn}
\end{align}
%
We show that $\spn^T$ is an SPRP as long as: (i) the linear layer $T$ contains no zero entries,
% (Miles and Viola~\cite{miles2015substitution} show that matrices with maximal branch number~\cite{daemen1995cipher} satisfy this property), 
and (ii) the round keys $k_0,k_1,k_2,k_3,k_4$ are uniform and independent.


\begin{theorem}
\label{theorem:4-round-spn}

Assume $w\geq2$, and $p+wq\leq N/2$. Let $\spn_{\bfk}[\mathcal{S}]$ be a 4-round, linear SPN as defined by Eq. (\ref{eq:defn-4-round-spn}). If round keys $\bfk=(k_0,k_1,k_2,k_3,k_4)$ are uniform and independent, and $T$ contains no zero entries, then
%
\begin{align}
\operatorname{Adv}_{\spn^T}^{\mathrm{su}}(p, q) \leq& \frac{q^2}{2^{n w}} + \frac{8 w^2 q(p+wq)^2+w^2 q}{2^n}   .  
\notag   \\
\operatorname{Adv}_{\spn^T}^{\mathrm{mu}}(p, q) \leq& \frac{q^2}{2^{n w}} + \frac{8 w^2 q(p+wq)^2+w^2 q}{2^n}    \notag   \\
&+ \frac{16 w^2 q(p+w q)(p+w q +3 q)+4 w^2 q(p+3 wq)^2}{2^{2 n}}.
\notag
\end{align}
\end{theorem}
The proof of Theorem \ref{theorem:4-round-spn} relies on the following lemma and on Lemmas \ref{lemma:h-coeff} and \ref{lemma:point-wise}.


\begin{lemma}
	\label{lemma:proximity-4-round}
	
	Assume $p+wq\leq N/2$. Let $\dis$ be a distinguisher in the single-user setting that makes $p$ primitive queries to each of $S_1,S_2,S_3$, and $S_4$, and makes $q$ construction queries. Then for any attainable
	transcript $\tau=(\mathcal{Q}_C,\mathcal{Q}_S)$, one has
	\begin{align}
	\frac{\mathsf{p}_{2}(\tau)}{\mathsf{p}_{1}(\tau)}
	\geq 1 - xxx.
	\label{eq:bound-proximity-4-round}
	\end{align}
\end{lemma}




\subsection{Outline of the Proof}
\label{sec:proof-sketch-4-rounds}

Throughout the proof, we fix a distinguisher $\mathcal{D}$ as described in the statement and fix an attainable transcript $\tau =\left(\mathcal{Q}_{C}, \mathcal{Q}_{S}\right)$ obtained $\mathcal{D}$. Let
%
$$
\begin{aligned}
&\mathcal{Q}_{S_{1}}^{(0)}=\left\{(u, v) \in\{0,1\}^{n} \times\{0,1\}^{n}:(1, u, v) \in \mathcal{Q}_{S} \right\},\\
&\mathcal{Q}_{S_{2}}^{(0)}=\left\{(u, v) \in\{0,1\}^{n} \times\{0,1\}^{n}:(2, u, v) \in \mathcal{Q}_{S} \right\},\\
&\mathcal{Q}_{S_{3}}^{(0)}=\left\{(u, v) \in\{0,1\}^{n} \times\{0,1\}^{n}:(3, u, v) \in \mathcal{Q}_{S} \right\},\\
&\mathcal{Q}_{S_{4}}^{(0)}=\left\{(u, v) \in\{0,1\}^{n} \times\{0,1\}^{n}:(4, u, v) \in \mathcal{Q}_{S} \right\}
\end{aligned}
$$
%
and denote the domains and ranges of $\mathcal{Q}_{S_{1}}^{(0)}, \mathcal{Q}_{S_{2}}^{(0)}, \mathcal{Q}_{S_{3}}^{(0)}, \mathcal{Q}_{S_{4}}^{(0)}$ by        {\small
%
\begin{align*}
&U_{1}^{(0)}=\left\{u_{1} \in\{0,1\}^{n}:\left(1, u_{1}, v_{1}\right) \in \mathcal{Q}_{S_{1}}^{(0)}\right\}, \quad V_{1}^{(0)}=\left\{v_{1} \in\{0,1\}^{n}:\left(1, u_{1}, v_{1}\right) \in \mathcal{Q}_{S_{1}}^{(0)}\right\},\\
&U_{2}^{(0)}=\left\{u_{2} \in\{0,1\}^{n}:\left(2, u_{2}, v_{2}\right) \in \mathcal{Q}_{S_{2}}^{(0)}\right\}, \quad V_{2}^{(0)}=\left\{v_{2} \in\{0,1\}^{n}:\left(2, u_{2}, v_{2}\right) \in \mathcal{Q}_{S_{2}}^{(0)}\right\},\\
&U_{3}^{(0)}=\left\{u_{3} \in\{0,1\}^{n}:\left(3, u_{3}, v_{3}\right) \in \mathcal{Q}_{S_{3}}^{(0)}\right\}, \quad V_{3}^{(0)}=\left\{v_{3} \in\{0,1\}^{n}:\left(3, u_{3}, v_{3}\right) \in \mathcal{Q}_{S_{3}}^{(0)}\right\},\\
&U_{4}^{(0)}=\left\{u_{4} \in\{0,1\}^{n}:\left(4, u_{4}, v_{4}\right) \in \mathcal{Q}_{S_{4}}^{(0)}\right\}, \quad V_{4}^{(0)}=\left\{v_{4} \in\{0,1\}^{n}:\left(4, u_{4}, v_{4}\right) \in \mathcal{Q}_{S_{4}}^{(0)}\right\},
\end{align*}
}%
%




Point-wise proximity is usually established by enhancing the transcripts with auxiliary random variables, defining a large enough set of ``good'' randomness, and then, for each choice of a good random variable, lower bounding the probability of observing this transcript. Such random variables typically include the keys, and are usually called good if the adversary cannot use the randomness to follow the path of computation of the encryption/decryption of a query up to a contradiction. To this end, we follow~\cite[Sect. 4.2]{C:CDKLST18} and define an extension of the transcript in order to gather enough information to allow simple definition of bad randomness. Then, instead of summing over the choice of the randomness, we will define an extension of the transcript, that will provide the necessary information, and then sum over every possible good extension. In detail, a transcript $\tau$ is extended in the following manner:
\begin{itemize}
	\item At the end of the interaction between \dis and the real world $(\mathcal{S},\spn_{\bfk}^T[\mathcal{S}])$, we append $\tau$ with the keys $\bfk=(k_0,k_1,k_2,k_3,k_4)$ and the two random permutations $S_1,S_4$ in use;
	\item At the end of the interaction between \dis and the ideal world $(\mathcal{S},\widetilde{P})$, we append $\tau$ with randomly sampled keys $\bfk=(k_0,k_1,k_2,k_3,k_4)$ and the two random permutations $S_1,S_4$ in use.
\end{itemize}
Note that, in either case, it is equivalent to sampling two new random permutations $S_1,S_4$ such that $S_1\vdash\mathcal{Q}_{S_{1}}$ and $S_4\vdash\mathcal{Q}_{S_4}$ and appending them to $\tau$. With the above, for any $(x,y)\in\mathcal{Q}_C$ we define
%
$$a=T\big(\overline{S_1}\left(x \oplus k_{0}\right)\big),\ \ \  b=T^{-1}\big(\overline{S_{4}^{-1}}\left(y \oplus k_{4}\right)\big).$$
%
This extends the list $\mathcal{Q}_C$ into a list as follows:
%
$$\mathcal{Q}_C'=\big((x_1,a_1,b_1,y_1),\ldots,(x_q,a_q,b_q,y_q)\big).$$
%
With this new list, a colliding query is defined as a construction query $(x,y,a,b)\in\mathcal{Q}_C'$ as follows:
%
\begin{itemize}
	\item[1.]
	there exist an S-box query $(u,v)\in\mathcal{Q}_{S_2}^{(0)}$ and an integer $i \in\{1, \ldots, w\}$ such that $\left(a \oplus k_1\right)[i]=u$.
	\item[2.]
	there exist an S-box query $(u,v)\in\mathcal{Q}_{S_3}^{(0)}$ and an integer $i \in\{1, \ldots, w\}$ such that $\left(b \oplus T^{-1}(k_3)\right)[i]=v$.
	\item[3.] there exist a construction query $\left(a^{\prime}, b^{\prime}\right) \in \mathcal{Q}_{C}$ and an integer $i,j \in\{1, \ldots, w\}$ such that $(a, b, i) \neq\left(a^{\prime}, b^{\prime}, j\right)$ and $\left(a \oplus k_1\right)[i] = \left(a' \oplus k_1\right)[j]$.
	\item[4.] there exist a construction query $\left(a^{\prime}, b^{\prime}\right) \in \mathcal{Q}_{C}$ and an integer $i,j \in\{1, \ldots, w\}$ such that $(a, b, i) \neq\left(a^{\prime}, b^{\prime}, j\right)$ and $i \in\{1, \ldots, w\}$ such that $\left(b \oplus T^{-1}(k_3)\right)[i] = \left(b' \oplus T^{-1}(k_3)\right)[j]$.
\end{itemize}
%
%
Now we further introduce a new set $\mathcal{Q}_{S}'$ of S-box evaluations to complete the transcript extension. In detail, for each colliding query $(x,a,b,y)\in\mathcal{Q}_C'$, we will add tuples $\left(2, (a \oplus k_1)[i], v^{\prime}\right)_{1 \leq i \leq w}$ (if $(a, b)$ collides at the input of $S_2$) or $\left(3, u^{\prime}, (b \oplus T^{-1}(k_3))[i]\right)_{1 \leq i \leq w}$ (if $(a, b)$ collides at the output of $S_3$) to $\mathcal{Q}_{S}'$ by lazy sampling $v^{\prime}=S_2(\left(a \oplus k_1\right)[i])$ or $u^{\prime}=S_3^{-1}(\left(b \oplus k_3\right)[i])$, as long as it has not been determined by any existing query in $\mathcal{Q}_S$.


An extended transcript of $\tau$ includes all the above additional information, i.e.,
%
$$\tau'=(\mathcal{Q}_{C}',\mathcal{Q}_{S},\mathcal{Q}_{S}',S_1,S_4,\bfk).$$
%
For each collision between a construction query and a primitive query, or between two construction queries, the extended transcript will contain enough information to compute a complete round of the evaluation of the SPN. This will be useful to lower bound the probability to get the transcript $\tau$ in the real world.


Below in Sect. \ref{sec:bad-tau-4-rounds}, we will show that the number of bad extended transcripts is small enough; then in Sect. \ref{sec:good-tau-4-rounds}, we show that the probability to obtain good extension in the real world is sufficiently close to that in the ideal world. These will complete the proof.




\subsection{Bad Transcript Extensions and Probability}
\label{sec:bad-tau-4-rounds}

The first step is to define the set of bad extended transcripts. Consider an attainable extended transcript $\tau'=(\mathcal{Q}_{C}',\mathcal{Q}_{S},\mathcal{Q}_{S}',S_1,S_4,\bfk)$. Let
%
$$
\begin{aligned}
%&\mathcal{Q}_{S_{1}}^{(1)}=\left\{(u, v) \in\{0,1\}^{n} \times\{0,1\}^{n}:(1, u, v) \in \mathcal{Q}_{S} \cup \mathcal{Q}_{S_{outer}}^{\prime}\right\}\\
&\mathcal{Q}_{S_2}^{(1)}=\left\{(u, v) \in\{0,1\}^{n} \times\{0,1\}^{n}:(2, u, v) \in \mathcal{Q}_{S} \cup \mathcal{Q}_{S}^{\prime}\right\}\\
&\mathcal{Q}_{S_3}^{(1)}=\left\{(u, v) \in\{0,1\}^{n} \times\{0,1\}^{n}:(3, u, v) \in \mathcal{Q}_{S} \cup \mathcal{Q}_{S}^{\prime}\right\}.
%\\
%&\mathcal{Q}_{S_{4}}^{(1)}=\left\{(u, v) \in\{0,1\}^{n} \times\{0,1\}^{n}:(4, u, v) \in \mathcal{Q}_{S} \cup \mathcal{Q}_{S_{outer}}^{\prime}\right\}
\end{aligned}
$$
%
In words, $\mathcal{Q}_{S_{i}}^{(1)}$ summarizes each constraint that is forced on $S_{i}$ by $\mathcal{Q}_{S}$ and $\mathcal{Q}_{S}^{\prime}$. Let        {\small
%
$$
\begin{aligned}
%&U_{1}=\left\{u_{1} \in\{0,1\}^{n}:\left(1, u_{1}, v_{1}\right) \in \mathcal{Q}_{S_{1}}^{(1)}\right\}, \quad V_{1}=\left\{v_{1} \in\{0,1\}^{n}:\left(1, u_{1}, v_{1}\right) \in \mathcal{Q}_{S_{1}}^{(1)}\right\},\\
&U_2^{(1)}=\left\{u_2 \in\{0,1\}^{n}:\left(2, u_2, v_2\right) \in \mathcal{Q}_{S_2}^{(1)}\right\}, \quad V_2^{(1)}=\left\{v_2 \in\{0,1\}^{n}:\left(2, u_2, v_2\right) \in \mathcal{Q}_{S_2}^{(1)}\right\},\\
&U_3^{(1)}=\left\{u_3 \in\{0,1\}^{n}:\left(3, u_3, v_3\right) \in \mathcal{Q}_{S_3}^{(1)}\right\}, \quad V_3^{(1)}=\left\{v_3 \in\{0,1\}^{n}:\left(3, u_3, v_3\right) \in \mathcal{Q}_{S_3}^{(1)}\right\}.
%\\
%&U_{4}=\left\{u_{4} \in\{0,1\}^{n}:\left(4, u_{4}, v_{4}\right) \in \mathcal{Q}_{S_{4}}^{(1)}\right\}, \quad V_{4}=\left\{v_{4} \in\{0,1\}^{n}:\left(4, u_{4}, v_{4}\right) \in \mathcal{Q}_{S_{4}}^{(1)}\right\}
\end{aligned}
$$
}%
%
be the domains and ranges of $\mathcal{Q}_{S_2}^{(1)}$ and $\mathcal{Q}_{S_3}^{(1)}$ respectively.




\begin{definition}
\label{defn:bad-tau-4-rounds}

We say an extended transcript $\tau^{\prime}$ is bad if at least one of the following conditions is fulfilled. The conditions are classified into two categories depending on the relevant randomness. In detail, regarding $k_0,k_1,k_3,k_4$:
\begin{itemize}[leftmargin=10mm]
	\item[(B-0)] there exist (not necessarily distinct) $(x,a,b,y),(x',a',b',y'),(x'',a'',b'',y'')\in \mathcal{Q}_{C}'$ and three distinct indices $i, i', i'' \in \{1, \ldots, w\}$ such that:
	\begin{itemize}
		\item $(x\xor k_0)[i]=(x'\xor k_0)[i']=(x''\xor k_0)[i'']$, or
		\item $(y\xor k_4)[i]=(y'\xor k_4)[i']=(y''\xor k_4)[i'']$.
	\end{itemize}
	\item[\bone] there exist $(x,a,b,y) \in \mathcal{Q}_{C}'$ and distinct $i, i' \in \{1, \ldots, w\}$ such that:
	\begin{itemize}
		\item $(x\xor k_0)[i]\in U_1^{(0)}$ and $(x\xor k_0)[i']\in U_1^{(0)}$, or
		\item $(y\xor k_4)[i]\in V_4^{(0)}$ and $(y\xor k_4)[i']\in V_4^{(0)}$.
	\end{itemize}
	\item[\btwo] there exist $(x,a,b,y) \in \mathcal{Q}_{C}', u_1 \in U_1^{(0)}, v_4 \in V_4^{(0)}$, and index $i, j \in \{1, \ldots, w\}$ such that $\left(x \oplus k_{0}\right)[i]=u_1$ and $\left(y \oplus k_{4}\right)[j]=v_4$.
	\item[\bthree] there exist $(x,a,b,y) \in \mathcal{Q}_{C}', u_1 \in U_1^{(0)}, u_2\in U_2^{(1)}$, and index $i, j \in \{1, \ldots, w\}$ such that $\left(x \oplus k_{0}\right)[i]=u_1$ and $\left(a\oplus k_{1}\right)[j]=u_2$.
%	there exist $(x,a,b,y) \in \mathcal{Q}_{C}', u_1 \in U_1^{(0)}, u_2\in U_2^{(0)}$, and index $i, j \in \{1, \ldots, w\}$ such that $\left(x \oplus k_{0}\right)[i]=u_1$ and $\left(T\left(\overline{S_1}\left(x \oplus k_{0}\right) \oplus k_{1}\right)\right)[j]=u_2$.
	\item[\bfour] there exist $(x,a,b,y) \in \mathcal{Q}_{C}', v_{3}\in V_3^{(1)}, v_{4}\in V_4^{(0)}$, and index $i, j \in \{1, \ldots, w\}$ such that $\left(y \oplus k_{4}\right)[j]=v_4$ and $\left(b\xor T^{-1}(k_3)\right) [i]=v_3$.
%	there exist $(x,a,b,y) \in \mathcal{Q}_{C}', v_{3}\in V_3^{(0)}, v_{4}\in V_4^{(0)}$, and index $i, j \in \{1, \ldots, w\}$ such that $\left(y \oplus k_{4}\right)[j]=v_4$ and $\left(T^{-1}\left(\overline{S_4^{-1}}\left(y \oplus k_{4}\right)\right)\xor T^{-1}(k_3)\right) [i]=v_3$.
	\item[\bfive] \textcolor{red}{there exist $(x,a,b,y) \in \mathcal{Q}_{C}'$ and distinct indices $i, i' \in \{1, \ldots, w\}$ such that $(x\xor k_0)[i]=(x\xor k_0)[i']$, or $(y\xor k_4)[i]=(y\xor k_4)[i']$.}
\end{itemize}
%
%
Regarding $k_2,S_1,S_4$, and $\mathcal{Q}_S'$:
%
%
\begin{itemize}[leftmargin=10mm]
	\item[\bfive]
	there exist $(x,a,b,y) \in \mathcal{Q}_{C}'$ and $i, j \in\{1, \ldots, w\}$ such that $(a \oplus k_1)[i]\in U_{2}^{(1)}$ and $(b\xor T^{-1}(k_3))[j]\in V_{3}^{(1)}$.
	\item[\bsix]
	there exist $(x,a,b,y) \in \mathcal{Q}_{C}'$ and $i, j \in\{1, \ldots, w\}$ such that $(a \oplus k_1)[i]\in U_{2}^{(1)}$ and $(T(\overline{S_2}(a \oplus k_1))\xor k_2)[j]\in U_{3}^{(1)}$.
	\item[\bseven]
	there exist $(x,a,b,y) \in \mathcal{Q}_{C}'$ and $i, j \in\{1, \ldots, w\}$ such that $(b\xor T^{-1}(k_3))[j]\in V_{2}^{(1)}$ and $\left(T^{-1}(b\xor T^{-1}(k_3)) \oplus T^{-1}(k_2)\right)[j]\in V_{3}^{(1)}$.
	\item[\beight]
	there exist $(x,a,b,y) \in \mathcal{Q}_{C}'$, distinct $i, i^{\prime}\in\{1, \ldots, w\}$ such that:
	\begin{itemize}
		\item $(a \oplus k_1)[i]\in U_{2}^{(0)}$ and $(a \oplus k_1)[i']\in U_{2}^{(0)}$, or
		\item $\left(b\xor T^{-1}\left(k_3\right)\right)[i]\in V_3^{(0)}$ and
		$\left(b'\xor T^{-1}\left(k_3\right)\right)[i']\in V_3^{(0)}$.
	\end{itemize}
%	\item[\cfive] \textcolor{red}{there exist distinct $(a, b),(a',b') \in \mathcal{Q}_{C}^{*}\left(S_{1}, S_{4}\right)$, distinct $i, i^{\prime}\in\{1, \ldots, w\}$, $u_{2} \in U_{2}$ such that}
%	$$\left(T\left(a \oplus k_{1}\right)\right)[i] = u_2,\text{ and }
%	\left(T\left(a \oplus k_{1}\right)\right)[i'] = \left(T\left(a' \oplus k_{1}\right)\right)[i'].$$
	\item[\bnine] there exist $(x,a,b,y),(x',a',b',y') \in \mathcal{Q}_{C}'$ and $i, i^{\prime},j, j^{\prime} \in\{1, \ldots, w\}$, $(a,b, j) \neq \left(a^{\prime}, b',j^{\prime}\right)$, such that $(a \oplus k_1)[i]\in U_{2}^{(1)}, (a' \oplus k_1)[i']\in U_{2}^{(1)}$, and
	%
	$$\big(T(\overline{S_2}(a\xor k_1))\xor k_2\big)[j]=\big(T(\overline{S_2}(a'\xor k_1))\xor k_2\big)[j'].
	$$
	%
%	\item[\bten] there exist $(x,a,b,y) \in \mathcal{Q}_{C}'$, distinct $i, i^{\prime}\in\{1, \ldots, w\}$, $v_{3},v_{3}' \in V_{3}^{(0)}$ such that
%	$$\left(T^{-1}\left(b\right) \oplus k_{3}\right)[i] = v_3,\text{ and }
%	\left(T^{-1}\left(b\right) \oplus k_{3}\right)[i'] = v_3'.$$
%	\item[\ceight]
%	\textcolor{red}{there exist distinct $(a, b),(a',b') \in \mathcal{Q}_{C}^{*}\left(S_{1}, S_{4}\right)$, distinct $i, i^{\prime}\in\{1, \ldots, w\}$, $v_{3} \in V_{3}$ such that}
%	$$\left(T^{-1}\left(b\right) \oplus k_{3}\right)[i] = v_3,\text{ and }
%	\left(T^{-1}\left(b\right) \oplus k_{3}\right)[i] =\left(T^{-1}\left(b'\right) \oplus k_{3}\right)[i'].$$
	\item[\beleven] there exist $(x,a,b,y),(x',a',b',y') \in \mathcal{Q}_{C}'$ and $i, i^{\prime}, j, j^{\prime} \in\{1, \ldots, w\}$, $(a,b, j) \neq \left(a',b^{\prime}, j^{\prime}\right)$, such that $\big(b \oplus T^{-1}(k_3)\big)[i]\in V_{3}^{(1)}, \big(b' \oplus T^{-1}(k_3)\big)[i']\in V_{3}^{(1)}$, and
	%
	$$\big(T^{-1}(\overline{S_3^{-1}}(b \oplus T^{-1}(k_3))\xor k_2)\big)[j]=\big(T^{-1}(\overline{S_3^{-1}}(b' \oplus T^{-1}(k_3))\xor k_2)\big)[j'].
	$$
\end{itemize}
Any extended transcript that is not bad will be called good. Given an original transcript $\tau$, we denote $\Theta_{\mathrm{good}}(\tau)$ (resp. $\Theta_{\mathrm{bad}}(\tau)$) the set of good (resp. bad) extended transcripts of $\tau$ and $\Theta'(\tau)$ the set of all extended transcripts of $\tau$.
\end{definition}





We start by upper bounding the probability of getting bad transcripts in the ideal world.

\begin{lemma}
	\label{lemma:bad-tau-4-rounds}
	
	Assuming $p+wq\leq N/2$, then it holds
	\begin{align}
	\operatorname{Pr}[\tau^{\prime} \in \Theta_{\mathrm{bad}}(\tau)] \leq \frac{3w^{2} q \left(p+w q\right)^{2}}{N^{2}} + \frac{w^{2} q}{N} + \frac{9w^2 q (p+w q)^{2}}{N^2}+ \frac{16w^3q^2p}{N^2}.
	\label{eq:bound-bad-tau-4-rounds}
	\end{align}
\end{lemma}
\begin{proof}
We upper bound the probabilities of the conditions in turn.


\arrangespace

\noindent \textsc{\bone}.
%
For each of the $q{w\choose 2}\leq w^2q/2$ choices of $(x,a,b, y) \in \mathcal{Q}_{C}'$ and distinct $i, i' \in \{1, \ldots, w\}$, since $k_0[i]$ and $k_0[i']$ are uniform and independent, the probability to have $(x \oplus k_{0})[i]\in U_1^{(0)}$ and $(x \oplus k_0)[i']\in U_1^{(0)}$ is at most $p^2/N^2$. Similarly, the probability of $(y\xor k_4)[i]\in V_4^{(0)}$ and $(y\xor k_4)[i']\in V_4^{(0)}$ is at most $p^2/N^2$. Thus
%
$$
\operatorname{Pr}\left[\bone\right] \leq \frac{w^{2} q p^2}{N^{2}}.
$$
%



%\arrangespace

\noindent \textsc{\btwo}.
%
For each of the $w^2q$ choices of $(x,a,b,y)\in\mathcal{Q}_{C}'$ and indices $i, j \in \{1, \ldots, w\}$, since $k_{0}$ and $k_{4}$ are uniform and independent, the probability to have $(x \oplus k_{0})[i]\in U_1^{(0)}$ and $(y \oplus k_{4})[j]\in V_4^{(0)}$ is $p^2/N^2$. Thus
%
$$
\operatorname{Pr}\left[\btwo\right] \leq \frac{w^{2} q p^2}{N^{2}}.
$$
%



%\arrangespace

\noindent \textsc{\bthree and \bfour}.
%
Note that \bthree consists of three subevents:
\begin{itemize}
	\item(B-31) there exists $(x,a,b,y) \in \mathcal{Q}_{C}'$ and indices $i, j \in \{1, \ldots, w\}$ such that $\left(x \oplus k_{0}\right)[i]\in U_1^{(0)}$ and $\left(a\oplus k_{1}\right)[j]\in U_2^{(0)}$;
	\item(B-32) there exists $(x,a,b,y),(x',a',b',y')\in\mathcal{Q}_{C}'$, and $i,j,j'\in\{1,\ldots,w\}$ such that $x\neq x'$, $(x'\xor k_0)[i']\notin U_1^{(0)}$ for all $i'\in\{1,\ldots,w\}$, while $\left(x \oplus k_{0}\right)[i]\in U_1^{(0)}$ and $(a\xor k_1)[j]=(a'\xor k_1)[j']$.
	\item(B-33) there exists $(x,a,b,y),(x',a',b',y')\in\mathcal{Q}_{C}'$, and $i,i',j,j'\in\{1,\ldots,w\}$ such that $(x,j)\neq(x',j')$, while $\left(x \oplus k_{0}\right)[i]\in U_1^{(0)}$, $\left(x' \oplus k_{0}\right)[i']\in U_1^{(0)}$, and $(a\xor k_1)[j]=(a'\xor k_1)[j']$.
\end{itemize}
It is easy to see $\Pr[\text{(B-31)}]\leq w^2qp^2/N^2$: since $k_0$ and $k_1$ are uniform and independent, it holds ${\Pr}\big[(x\xor k_0)[i]\in U_1^{(0)}\wedge(a\xor k_1)[j]\in U_2^{(0)}\big]=p^2/N^2$.


For (B-32), note that $x\neq n'$ implies there exists $i_0$ such that $(x\xor k_0)[i_0]\neq(x'\xor k_0)[i_0]$. By the assumption, $(x'\xor k_0)[i_0]\notin U_1^{(0)}$. \textbf{This in particular means $(x'\xor k_0)[i_0]\neq(x\xor k_0)[i]$. By $\neg$(B-?), the exists at most 1 index $i_1$ such that $(x\xor k_0)[i_1]=(x'\xor k_0)[i_0]$. By these, we write}
%
\begin{align*}
&  T(\overline{S_1}(x\xor k_0))[j]\xor T(\overline{S_1}(x'\xor k_0))[j']       \\
= &
\Big(t_{i,1}\cdot S_1\big((x\xor k_0)[1]\big)
\xor
t_{i,i_0}\cdot S_1\big((x\xor k_0)[i_0]\big)\Big)
\xor
\bigoplus_{2\leq\ell\leq w,\ell\neq i_1}t_{i,\ell}\cdot S_1\big((x\xor k_0)[\ell]\big)     \\
= &
\Big(\big(t_{i,1}\xor t_{i,i_0}\big)\cdot S_1\big((x\xor k_0)[1]\big)\Big)
\xor
\bigoplus_{2\leq\ell\leq w,\ell\neq i_1}t_{i,\ell}\cdot S_1\big((x\xor k_0)[\ell]\big)    .
\end{align*}
%
Conditioned on $S_1\vdash\mathcal{Q}_{S_1}^{(0)}$ and on the $w-2$ values $\big\{S_1((x\xor k_0)[i'])\}_{2\leq i'\leq w,i'\neq i_1}$, \textbf{the value of $S_1((x'\xor k_0)[i_0])$ remains uniform in {\it at least} $N-p-wq$ possibilities. Moreover, the coefficient $t_{i,1}\xor t_{i,i_0}$ is non-zero as per our assumption. Therefore,}
%
%
%Therefore, conditioned on $S_1\vdash\mathcal{Q}_{S_1}^{(0)}$, the value of $S_1((x'\xor k_0)[i_0])$ remains uniform in {\it at least} $N-p-wq$ possibilities. Because every entry in the $i_0$th column of $T$ is nonzero, we have
%
$${\Pr}\big[T(\overline{S_1}(x\xor k_0))[j]\xor T(\overline{S_1}(x'\xor k_0))[j']=k_1[j]\xor k_1[j']\big]\leq\frac{1}{N-p-wq}.$$
%
Meanwhile, it remains $\Pr[\left(x \oplus k_{0}\right)[i]\in U_1^{(0)}]=p/N$. Therefore, $\Pr[\text{(B-32)}]\leq{q\choose2}w^3p/N(N-p-wq)\leq w^3q^2p/2N(N-p-wq)$.




For (B-33), we define $\pcoll_{11}(x,x',i,i',j,j')$ as the conditional probability $\Pr[(a\xor k_1)[j]=(a'\xor k_1)[j']|   
(x'\xor k_0)[i']\in U_1^{(0)}\wedge(x\xor k_0)[i]\in U_1^{(0)}]$. Then, we derive the probability as follows.        {\small
%
\begin{align*}
\Pr[\text{(B-33)}]  
=   &  \sum_{(x,a,b,y),(x',a',b',y')\in\mathcal{Q}_{C}'}\sum_{i,i',j,j'}\bigg(\underbrace{\Pr[(x\xor k_0)[i]\in U_1^{(0)}]}_{\leq p/N}     \\
   & \midindent\times
	\underbrace{\Pr[(x'\xor k_0)[i']\in U_1^{(0)}|(x\xor k_0)[i]\in U_1^{(0)}]}_{\leq1}\times\pcoll_{11}(x,x',i,i',j,j') \bigg)      \\
\leq  &  \frac{p}{N}\cdot\bigg(
\underbrace{\sum_{(x,a,b,y),(x',a',b',y')\in\mathcal{Q}_{C}'}\sum_{i,i'}\sum_{j\neq j'}\pcoll_{11}(x,x',i,i',j,j')}_{A_1}     \\
& \midindent\midindent   +  
\underbrace{\sum_{(x,a,b,y),(x',a',b',y')\in\mathcal{Q}_{C}'}\sum_{i\neq i'}\sum_{j}\pcoll_{11}(x,x',i,i',j,j)}_{A_2}      \\ 
& \midindent\midindent   +  \underbrace{\sum_{(x,a,b,y),(x',a',b',y')\in\mathcal{Q}_{C}'}\sum_{i}\sum_{j}\pcoll_{11}(x,x',i,i,j,j)}_{A_3}\bigg)     .
\end{align*}
}%
%

When $j\neq j'$, we have $\pcoll_{11}(x,x',i,i',j,j')=1/N$ by that $k_1[j]$ and $k_1[j']$ are uniform and independent. Therefore, $A_1\leq{w\choose2}w^2q^2p/N^2\leq w^4q^2p/2N^2$.


Next, consider $\pcoll_{11}(x,x',i,i',j,j)$ with $i\neq i'$. Since $x\neq x'$, there exists $i_0$ such that $(x\xor k_0)[i_0]\neq(x'\xor k_0)[i_0]$. Then either $i\neq i_0$ or $i'\neq i_0$. Wlog assume $i\neq i_0$. \textbf{Note that this means $(x'\xor k_0)[i]\neq(x\xor k_0)[i_0]$. By $\neg$(B-?), the exists at most 1 index $i_1$ such that $(x\xor k_0)[i_1]=(x'\xor k_0)[i_0]$. By these, and by an argument similar to that for (B-32) before,} we have ${\Pr}[T(\overline{S_1}(x\xor k_0))[j]\xor T(\overline{S_1}(x'\xor k_0))[j]]\leq\frac{1}{N-p-wq}$, which means $A_2\leq w^3q^2p/2N(N-p-wq)$.



Finally, consider $\pcoll_{11}(x,x',i,i,j,j)$. Assume that $\overline{S_1}(x\xor k_0)=\bfv_1\|v_1\|\bfv_2$ and
$\overline{S_1}(x'\xor k_0)=\bfv_1'\|v_1'\|\bfv_2'$, where $v_1,v_1'\in V_1^{(0)}$. Then the equality $T(\overline{S_1}(x\xor k_0))[j]=T(\overline{S_1}(x'\xor k_0))[j]$ implies
%
\begin{align}
\bft_1^*\cdot\bfv_1\xor t^*\cdot v_1\xor\bft_2^*\cdot\bfv_2=\bft_1^*\cdot\bfv_1'\xor t^*\cdot v_1'\xor\bft_2^*\cdot\bfv_2'.
\label{eq:interm-eq-b2}
\end{align}
%
%
for two vectors $\bft_1^*,\bft_2^*$ and $t^*\in\{0,1\}^n$. Now:
\begin{itemize}
	\item If $x[i_0]=x'[i_0]$ for any $i_0\neq i$, then $\bfv_1=\bfv_1'$ and $\bfv_2=\bfv_2'$, and Eq. (\ref{eq:interm-eq-b2}) collapses to $t^*\cdot v_1=t^*\cdot v_1'$ which is not possible since $t^*\neq 0$ and $v_1\neq v_1'$;
	\item Else, there exists $i_0\neq i$ such that $(x\xor k_0)[i_0]\neq(x'\xor k_0)[i_0]$. By this and $\neg\bone$, conditioned on $S_1\vdash\mathcal{Q}_{S_{1}}^{(0)}$, the value of $S_1(x\xor k_0)$ remains uniform in at least $N-p-wq$ possibilities, which means Eq. (\ref{eq:interm-eq-b2}) holds with probability at most $1/(N-p-wq)$.
\end{itemize}
Therefore, in this case, it still holds $\pcoll_{11}(x,x',i,i,j,j)\leq1/(N-p-wq)$, which means $A_3\leq{q\choose2}w^2p/N(N-p-wq)\leq w^2q^2p/2N(N-p-wq)$.


Summing over the above, we reach
%
$${\Pr}\big[\bthree\mid\neg\bone\big]\leq\frac{w^2qp^2}{N^2}
+
\frac{w^3q^2p}{N(N-p-wq)}
+
\frac{w^4q^2p}{2N^2}
+
\frac{w^2q^2p}{2N(N-p-wq)}.$$
%
Similarly, ${\Pr}\big[\bfour\mid\neg\bone\big] \leq \frac{w^{2} q \left(p+w q\right)^{2}}{N^{2}}$ by symmetry.




%
%
%
%For \btwo, we have
%%
%%
%\begin{align}
%\Pr[\btwo] = & \sum_{(x,a,b,y)\in \mathcal{Q}_{C}'}\sum_{i,j}\Big({\Pr}\big[(x\xor k_0)[i]\in U_1^{(0)}\wedge(a\xor k_1)[j]\in U_2^{(0)}\big]
%\label{eq:btwo-bound1}         \\
%& \hugeindent\codeindent\codeindent +{\Pr}\big[(x\xor k_0)[i]\in U_1^{(0)}\wedge(a\xor k_1)[j]\in (U_2^{(1)}\backslash U_2^{(0)})\big]\Big)
%\label{eq:btwo-bound2}    
%\end{align}
%
%The term in (\ref{eq:btwo-bound1}) is easy to analyze: since $k_0$ and $k_1$ are uniform and independent, it holds ${\Pr}\big[(x\xor k_0)[i]\in U_1^{(0)}\wedge(a\xor k_1)[j]\in U_2^{(0)}\big]=1/N^2$. On the other hand, the event $(a\xor k_1)[j]\in (U_2^{(1)}\backslash U_2^{(0)})$ means there exists $(x',a',b',y')\in\mathcal{Q}_{C}'$ and $j'\in\{1,\ldots,w\}$ such that $(a,j)\neq(a',j')$, while $(a\xor k_1)[j]=(a'\xor k_1)[j']$. Therefore,
%%
%\begin{align*}
%& \sum_{(x,a,b,y)\in \mathcal{Q}_{C}'}\sum_{i,j}\Big({\Pr}\big[(x\xor k_0)[i]\in U_1^{(0)}\wedge(a\xor k_1)[j]\in (U_2^{(1)}\backslash U_2^{(0)})\big]\Big)           \\
%=& \sum_{(x,a,b,y),(x',a',b',y')}\sum_{i,j,j'}\Big(\underbrace{{\Pr}\big[(x\xor k_0)[i]\in U_1^{(0)}\big]}_{=1/N}          \\
%& \hugeindent\codeindent\codeindent \times
%{\Pr}\big[(a\xor k_1)[j]\in (U_2^{(1)}\backslash U_2^{(0)})~{\big|}~(x\xor k_0)[i]\in U_1^{(0)}\big]\Big)     
%\end{align*}
%
%
%
%
%In all, we have $\operatorname{Pr}\left[\btwo\right] \leq \frac{w^{2} q p \left(p+w q\right)}{N^{2}}\leq\frac{w^{2} q \left(p+w q\right)^{2}}{N^{2}}$. Similarly, $\operatorname{Pr}\left[\bthree\right] \leq \frac{w^{2} q \left(p+w q\right)^{2}}{N^{2}}$ by symmetry.



%
%\arrangespace
%
%\noindent \textsc{\bfive}.
%Since $k_{0}$ and $k_{4}$ are uniform, for each $(x,a,b, y) \in \mathcal{Q}_{C}'$ and $i, i' \in \{1, \ldots, w\}$, the probability to have $(x\xor k_0)[i]=(x\xor k_0)[i']$ or $(y\xor k_4)[i]=(y\xor k_4)[i']$ is $2/N$. Since the number of such choices is $q{w\choose 2}\leq w^2q/2$, we have $\operatorname{Pr}\left[\bfive\right] \leq w^{2} q/N$.
%


%
\smallskip

For the remaining, define event
%
$$\coll_2(x,a,b,y)\Leftrightarrow\text{there exist }i\in\{1,\ldots,w\}
\text{ and }
u_2\in U_2
\text{ such that }
(a\xor k_1)[i]=u_2.$$
%
This event can be broken down into the following two subevents:
%
\begin{itemize}
	\item $\coll_{21}(x,a,b,y)$: there exist $i\in\{1,\ldots,w\}$, $(u_2,v_1)\in\mathcal{Q}_{S_2}^{(0)}$ such that $(a\xor k_1)[i]=u_2$;
	\item $\coll_{22}(x,a,b,y)$: there exist $(x',a',b',y')\in\mathcal{Q}_C'$, $i,i'\in\{1,\ldots,w\}$ such that $(a,b,i)\neq(a',b',i')$ and $(a\xor k_1)[i]=(a'\xor k_1)[i']$.
\end{itemize}
%
Consider the subevent $\coll_{21}(x,a,b,y)$ first. To have $(a\xor k_1)[i]=u_2$, it has to be $(x\xor k_0)[i_0]\notin U_1^{(0)}$ for any $i_0\in\{1,\ldots,w\}$, as otherwise it contradicts $\neg\btwo$. Thus conditioned on $S_{1} \vdash \mathcal{Q}_{S_{1}}$, the value of $S_1((x \oplus k_0)[i_0])$ remains uniform in $\{0, 1\}^{n} \backslash V_1^{(1)}$ for any fixed $i_0$. Because every entry in the $i_{0}$th column of $T$ is nonzero, we have
%
$$\Pr\big[\coll_{21}(x,a,b,y)\big]=\Pr\big[\exists i,u_2:(T(\overline{S_1}(x\xor k_0))\xor k_1)[i]=u_2\big]\leq\frac{wp}{N-p-wq}.$$
%


For the subevent $\coll_{22}(x,a,b,y)$, note that        {\small
%
\begin{align}
\Pr\big[\coll_{22}(x,a,b,y)\big]        
= &  \underbrace{\sum_{(x',a',b',y')\in\mathcal{Q}_C'}\sum_{i\neq i'\in\{1,\ldots,w\}}\Pr\big[(a\xor k_1)[i]=(a'\xor k_1)[i']\big]}_{\leq w^2q/2N}      
\label{eq:coll22-bound-1}       \\
 & +  \sum_{(x',a',b',y')\in\mathcal{Q}_C',x'\neq x}\sum_{i\in\{1,\ldots,w\}}\Pr\big[a[i]=a'[i]\big] ,
\label{eq:coll22-bound-2}
\end{align}
}%
%
where (\ref{eq:coll22-bound-1}) follows from that $k_1[i]$ and $k_1[i']$ are uniform and independent. For the term (\ref{eq:coll22-bound-2}),
\begin{itemize}
	\item0
	\item0
	\item0
	\item0
	\item0
\end{itemize}




Similarly, define
%
$$\coll_3(x,a,b,y)\Leftrightarrow\text{there exist }i\in\{1,\ldots,w\}
\text{ and }
v_3\in V_3
\text{ such that }
(b\xor T^{-1}(k_3))[i]=v_3.$$
%
Then it holds
%

%
by symmetry. With these, we are able to analyze the remaining conditions.





\arrangespace

\noindent \textsc{\cone}. For any construction query $(x,a,b,y)\in\mathcal{Q}_C'$, to have $(a\xor k_1)[i]\in U_2^{(1)}$, it has to be $(x\xor k_0)[i_0]\notin U_1^{(0)}$ for any $i_0\in\{1,\ldots,w\}$, as otherwise it contradicts $\neg\bthree$. \textbf{By $\neg$(B-?), the exists at most 1 index $i_1$ such that $(x\xor k_0)[i_1]=(x\xor k_0)[1]$. By these, we write}
%
\begin{align*}
& (T(\overline{S_1}(x\xor k_0)))[i]       \\
= &
\Big(t_{i,1}\cdot S_1\big((x\xor k_0)[1]\big)
\xor
t_{i,i_0}\cdot S_1\big((x\xor k_0)[i_0]\big)\Big)
\xor
\bigoplus_{2\leq\ell\leq w,\ell\neq i_1}t_{i,\ell}\cdot S_1\big((x\xor k_0)[\ell]\big)     \\
= &
\Big(\big(t_{i,1}\xor t_{i,i_0}\big)\cdot S_1\big((x\xor k_0)[1]\big)\Big)
\xor
\bigoplus_{2\leq\ell\leq w,\ell\neq i_1}t_{i,\ell}\cdot S_1\big((x\xor k_0)[\ell]\big)    .
\end{align*}
%
Conditioned on $S_1\vdash\mathcal{Q}_{S_1}^{(0)}$ and on the $w-2$ values $\big\{S_1((x\xor k_0)[i'])\}_{2\leq i'\leq w,i'\neq i_1}$, \textbf{the value of $S_1((x \oplus k_0)[0])$ remains uniform in at least $N-p-wq$ values. Moreover, the coefficient $t_{i,1}\xor t_{i,i_0}$ is non-zero as per our assumption. Therefore,} the probability to have $(a\xor k_1)[i]=(T(\overline{S_1}(x\xor k_0)))[i]\xor k_1[i]$ equal some value in $U_2^{(1)}$ is at most $(p+wq)/(N-p-wq)$. Similarly, the probability of $(b\xor T^{-1}(k_3))[j]\in U_2^{(1)}$ is at most $(p+wq)/(N-p-wq)$. Since we have at most $qw^2$ choices for $(x,a,b,y)$ and $i, j \in\{1, \ldots, w\}$, we have
%
$$
{\Pr}\big[\cone\mid\neg\bthree\wedge\neg\bfour\big] \leq \frac{w^{2} q (p+w q)^{2}}{(N-p-wq)^{2}}.
$$



%\arrangespace

\noindent \textsc{\ctwo and \cthree}. Following the analysis of \cone, for any $(x,a,b,y)\in\mathcal{Q}_C'$, the probability to have $(a\xor k_1)[i]\in U_{2}^{(1)}$ is at most $(p+wq)/(N-p-wq)$. On the other hand, since $k_{2}$ is uniform and independent from the queries and from $k_{0}$, $k_{1}$, the probability to have $(T(\overline{S_2}(a \oplus k_1))\xor k_2)[j]\in U_{3}^{(1)}$ is $1/N$. Since we have at most $qw^2$ choices for $(x,a,b,y)$ and $i, j \in\{1, \ldots, w\}$, we have
%
%
$$
{\Pr}\big[\ctwo\mid\neg\bthree\big] \leq \frac{w^{2} q (p+w q)^{2}}{N  (N-p-wq)}.
$$
%
Similarly,
%
$$
{\Pr}\big[\cthree\mid\neg\bfour\big] \leq \frac{w^{2} q (p+w q)^{2}}{N  (N-p-wq)}.
$$




%\arrangespace

\noindent \textsc{\beight}. For any of the at most $qw^2/2$ choices of $(x,a,b,y)\in\mathcal{Q}_C'$ and distinct $i, i^{\prime}\in\{1, \ldots, w\}$, the probability to have $(a \oplus k_1)[i]\in U_{2}^{(0)}$ and $(a \oplus k_1)[i']\in U_{2}^{(0)}$ is $p^2/N^2$, since $k_1$ is uniform and independent of $S_1$. Similarly, the probability to have $\left(b\xor T^{-1}\left(k_3\right)\right)[i]\in V_3^{(0)}$ and
$\left(b'\xor T^{-1}\left(k_3\right)\right)[i']\in V_3^{(0)}$ is $p^2/N^2$. By these,
%
$$
\operatorname{Pr}\left[\beight\right] \leq \frac{w^2qp^2}{N^2}.
$$


%
%\smallskip
%\noindent\textsc{\cfive and \ceight}. Consider any of the $qw^2p^2/2$ choices of $(a, b),(a',b') \in \mathcal{Q}_{C}^{*}\left(S_{1}, S_{4}\right)$, distinct $i, i^{\prime}\in\{1, \ldots, w\}$, and $u_{2} \in U_{2}$.
%
%
%By $\neq\btwo$, to have $\left(T_1\left(a \oplus k_{1}\right)\right)[i] = u_2$, it has to be that $(x\xor k_0)[i_0]\notin U_1$ for any $i_0$.
%
%
%There exists $i_0$ such that $(x\xor k_0)[i_0]\neq(x'\xor k_0)[i_0]$.






%$$\left(T_1\left(a \oplus k_{1}\right)\right)[i] = u_2,\text{ and }
%\left(T_1\left(a \oplus k_{1}\right)\right)[i'] = \left(T_1\left(a' \oplus k_{1}\right)\right)[i'].$$

%
%We have the value $\left(S_{1}\left(x \oplus k_{0}\right)\right)[i]$ remain uniform in $\{0, 1\}^{n} \verb|\| (\mathcal{Q}_{S_{1}} \cup \mathcal{Q}_{S_{4}})$.  Because of the fact that there exists $i$ with $x[i]\neq x'[i]$, that is there exist $i$ with $\left(a \oplus k_{1}\right)[i'] \neq \left(a' \oplus k_{1}\right)[i']$. So,
%
%$$
%\operatorname{Pr}\left[\tau_{inner} \in \Theta_{5}\right] \leq \frac{w^{2} q^{2} (p+w q)}{(N-p)^2}.
%$$




%\arrangespace

\noindent\textsc{\csix and \cnine}. Consider \csix first. The argument slightly resembles that of (B-33). In detail, define $\pcoll_{22}(a,a',i,i',j,j')$ as the conditional probability $\Pr\big[\big(T(\overline{S_2}(a\xor k_1))\xor k_2\big)[j]=\big(T(\overline{S_2}(a'\xor k_1))\xor k_2\big)[j']|   
(a' \oplus k_1)[i']\in U_2^{(1)}\wedge(a \oplus k_1)[i]\in U_2^{(1)}\big]$. Then we derive the probability as follows.       
%
\begin{align*}
\Pr[\csix] 
=   &  \sum_{(x,a,b,y),(x',a',b',y')\in\mathcal{Q}_{C}'}\sum_{i,i',j,j'}\bigg(\underbrace{{\Pr}\big[(a\xor k_1)[i]\in U_2^{(1)}\big]}_{\leq(p+wq)/(N-p-wq)}     \\
& \midindent\times
\underbrace{{\Pr}\big[(a'\xor k_1)[i]\in U_2^{(1)}|(a\xor k_1)[i]\in U_2^{(1)}\big]}_{\leq1}\times\pcoll_{22}(a,a',i,i',j,j') \bigg)      \\
\leq  &  \frac{p+wq}{N-p-wq}\cdot\bigg(
\underbrace{\sum_{(x,a,b,y),(x',a',b',y')\in\mathcal{Q}_{C}'}\sum_{i,i'}\sum_{j\neq j'}\pcoll_{22}(a,a',i,i',j,j')}_{B_1}     \\
& \midindent\midindent   +  
\underbrace{\sum_{(x,a,b,y),(x',a',b',y')\in\mathcal{Q}_{C}'}\sum_{i\neq i'}\sum_{j}\pcoll_{22}(a,a',i,i',j,j)}_{B_2}      \\ 
& \midindent\midindent   +  \underbrace{\sum_{(x,a,b,y),(x',a',b',y')\in\mathcal{Q}_{C}'}\sum_{i}\sum_{j}\pcoll_{22}(a,a',i,i,j,j)}_{B_3}\bigg)     .
\end{align*}
%
%

When $j\neq j'$, we have $\pcoll_{22}(a,a',i,i',j,j')=1/N$ by that $k_2[j]$ and $k_2[j']$ are uniform and independent. Therefore, $B_1\leq{w\choose2}w^2q^2p/N^2\leq w^4q^2p/2N^2$.


Next, consider $\pcoll_{22}(a,a',i,i',j,j)$ with $i\neq i'$. Since $a\neq a'$, there exists $i_0$ such that $(a\xor k_1)[i_0]\neq(a'\xor k_1)[i_0]$. Wlog assume $i\neq i_0$. Then by $\neg\beight$, conditioned on $S_2\vdash\mathcal{Q}_{S_2}^{(0)}$, $S_2((a\xor k_1)[i_0])$ remains uniform in at least $N-p-wq$ possibilities. Because every entry in the $i_0$th column of $T$ is nonzero, we have ${\Pr}[\big(T(\overline{S_2}(a\xor k_1))\xor k_2\big)[j]=\big(T(\overline{S_2}(a'\xor k_1))\xor k_2\big)[j']]\leq\frac{1}{N-p-wq}$, which means $B_2\leq w^3q^2p/2N(N-p-wq)$.



Finally, consider $\pcoll_{22}(a,a',i,i,j,j)$. Assume that $\overline{S_2}(a\xor k_1)=\bfv_3\|v_2\|\bfv_4$ and
$\overline{S_2}(a'\xor k_1)=\bfv_3'\|v_2'\|\bfv_4'$, where $v_2,v_2'\in V_2^{(0)}$. Then the equality $T(\overline{S_2}(a\xor k_1))[j]=T(\overline{S_2}(a'\xor k_1))[j]$ implies
%
\begin{align}
\bft_3^*\cdot\bfv_3\xor t^{**}\cdot v_2\xor\bft_4^*\cdot\bfv_4=\bft_3^*\cdot\bfv_3'\xor t^{**}\cdot v_2'\xor\bft_4^*\cdot\bfv_4'.
\label{eq:interm-eq-b11}
\end{align}
%
%
for two vectors $\bft_3^*,\bft_4^*$ and $t^{**}\in\{0,1\}^n$. Now:
\begin{itemize}
	\item If $a[i_0]=a'[i_0]$ for any $i_0\neq i$, then $\bfv_3=\bfv_3'$ and $\bfv_4=\bfv_4'$, and Eq. (\ref{eq:interm-eq-b11}) collapses to $t^{**}\cdot v_2=t^{**}\cdot v_2'$ which is not possible since $t^{**}\neq 0$ and $v_2\neq v_2'$;
	\item Else, there exists $i_0\neq i$ such that $(a\xor k_1)[i_0]\neq(a'\xor k_1)[i_0]$. By this and $\neg\beight$, conditioned on $S_2\vdash\mathcal{Q}_{S_2}^{(0)}$, {\bf the value of $S_2(x\xor k_0)$ remains uniform in at least $N-p-wq$ possibilities}, which means Eq. (\ref{eq:interm-eq-b11}) holds with probability at most $1/(N-p-wq)$.
\end{itemize}
Therefore, in this case, it still holds $\pcoll_{22}(x,x',i,i,j,j)\leq1/(N-p-wq)$, which means $A_3\leq{q\choose2}w^2p/N(N-p-wq)\leq w^2q^2p/2N(N-p-wq)$.



Summing over the above, we reach
%
$$\Pr[\cfive]\leq\frac{4w^3q^2p}{N(N-p-wq)}.$$
%

The analysis of \cseven is similar by symmetry, resulting in
%
%
$$\Pr[\cseven]\leq\frac{4w^3q^2p}{N(N-p-wq)}.$$
%



\arrangespace


Summing over the above yields
%
\begin{align*}
&  \Pr\big[ \tau' \in \Theta_{\text {good }}(\tau)\big]  \leq \sum_{i=1}^{11}\Pr[\bi]       \\
\leq  & \frac{w^2 q (p+w q)^{2}}{(N-p-wq)^{2}}+
\frac{2w^2 q (p+w q)^{2}}{N  (N-p-wq)}+
\frac{w^2qp^2}{N^2}+
\frac{8w^3q^2p}{N(N-p-wq)}      \\
\leq  & \frac{9w^2 q (p+w q)^{2}}{N^2}+ \frac{16w^3q^2p}{N^2} .
\end{align*}
%
as claimed.        \qed
\end{proof}









%\paragraph{\textsc{The inner two rounds.}}


\subsection{Analyzing Good Transcript Extensions}
\label{sec:good-tau-4-rounds}

We are now ready for the second step of the reasoning. Define
%
$$\calC_{\bfk}^T[\calS](a):=   \overline{S_3}(T(\overline{S_2}(a\xor k_1))\xor k_2)\xor T^{-1}(k_3).$$
%
For any attainable transcript $\tau$, the ideal world probability is easy to calculate:
%
%
\begin{align*}
\mathsf{p}_{1}(\tau)=&\operatorname{Pr}\left[(\widetilde{P},\mathcal{S})\stackrel{\$}{\leftarrow} \widetilde{{\mathsf{Perm}}}(\mathcal{T}, w n)\times\mathsf{Perm}(n)^4: (\mathcal{S} \vdash \mathcal{Q}_{S}) \wedge(\widetilde{P} \vdash \mathcal{Q}_{C})  \right]		\\
\leq&\frac{1}{(N^w)_q}\cdot\bigg(\frac{1}{(N)_p}\bigg)^4.
\end{align*}



To reach the real world probability $\mathsf{p}_2(\tau)$, for any transcript extension $\tau'=(\mathcal{Q}_{C}',\mathcal{Q}_{S},\mathcal{Q}_{S}',S_1^*,S_4^*,\bfk)$ from $\tau$, denote            {\small
%
%
\begin{align}
\mathsf{p}_{\mathrm{re}}(\tau') = & \operatorname{Pr}\Big[\left(\mathbf{k}',\mathcal{S}\right) \stackrel{\$}{\leftarrow} \big(\{0,1\}^{wn}\big)^5 \times \mathsf{Perm}(n)^4:
\Big(\big(S_1=S_1^*\big)\wedge\big(S_4=S_4^*\big)\wedge		\notag 	\\
&\codeindent\codeindent\codeindent\codeindent\codeindent\codeindent\codeindent\codeindent\big(S_2\vdash\mathcal{Q}_{S_2}^{(1)}\big)\wedge\big(S_3\vdash\mathcal{Q}_{S_3}^{(1)}\big)\wedge\big(\calC_{\bfk'}^T[\calS] \vdash \mathcal{Q}_C'\big)\wedge\big(\bfk'=\bfk\big)\Big)\Big]	 	\notag 	\\
\mathsf{p}_{\mathrm{mid}}(\tau') = & \operatorname{Pr}\Big[\mathcal{S} \stackrel{\$}{\leftarrow}\mathsf{Perm}(n)^4:(\calC_{\bfk}^T[\calS] \vdash \mathcal{Q}_C')~\Big|~
(S_1=S_1^*)\wedge (S_4=S_4^*)\wedge	 	\notag 	\\
&\codeindent\codeindent\codeindent\codeindent\codeindent\codeindent\codeindent\codeindent (S_2\vdash\mathcal{Q}_{S_2}^{(1)})\wedge (S_3\vdash\mathcal{Q}_{S_3}^{(1)})\Big].	 	\notag 	
%\label{eq:defn-p-mid}
\end{align}
}%
%
%
and let $\alpha_1=|\mathcal{Q}_{S_2}^{(1)}|-|\mathcal{Q}_{S_2}^{(0)}|=|\mathcal{Q}_{S_2}^{(1)}|-p$ and $\alpha_2=|\mathcal{Q}_{S_3}^{(1)}|-p$. With these, we have
%
%
\begin{align*}
\mathsf{p}_2(\tau)=&\operatorname{Pr}\left[\left(\mathbf{k},\mathcal{S}\right) \stackrel{\$}{\leftarrow} \big(\{0,1\}^{wn}\big)^5 \times \mathsf{Perm}(n)^4:\big(\spn_{\bfk}^{T}[\mathcal{S}] \vdash \mathcal{Q}_{C}\big) \wedge \big(\mathcal{S} \vdash \mathcal{Q}_{S}\big)\right]		\\
\geq & \sum_{\tau^{\prime} \in \Theta_{\mathrm{good}}(\tau)} \mathsf{p}_{\mathrm{re}}(\tau')  
\geq
\sum_{\tau^{\prime} \in \Theta_{\mathrm{good}}(\tau)}
%
\frac{1}{N^{5w}\big((N)_{N}\big)^2(N)_{p+\alpha_1}(N)_{p+\alpha_2}}\cdot \mathsf{p}_{\mathrm{mid}}(\tau')  .
\end{align*}
%
%
Therefore,
%
%
\begin{align*}
\frac{\mathsf{p}_{2}(\tau)}{\mathsf{p}_{1}(\tau)}   \geq  &
\sum_{\tau^{\prime} \in \Theta_{\mathrm{good}}(\tau)}
\frac{(N^w)_q\cdot\big((N)_p\big)^4}{N^{5w}\big((N)_{N}\big)^2(N)_{p+\alpha_1}(N)_{p+\alpha_2}}\cdot \mathsf{p}_{\mathrm{mid}}(\tau')         \\
\geq  &    \min_{\tau' \in \Theta_{\mathrm{good}}(\tau)}\big((N^w)_q\cdot\mathsf{p}_{\mathrm{mid}}(\tau')\big)
\underbrace{\sum_{\tau^{\prime} \in \Theta_{\mathrm{good}}(\tau)}
\frac{1}{N^{5w}\big((N-p)_{N-p}\big)^2(N-p)_{\alpha_1}(N-p)_{\alpha_2}}}_{B} .
\end{align*}



Note that, the exact probability of observing the extended transcript $\tau'$ is
%
%
$$\frac{1}{N^{5w}\big((N-p)_{N-p}\big)^2(N-p)_{\alpha_1}(N-p)_{\alpha_2}},$$
%
since:
%
\begin{itemize}
	\item[1.] sample keys $k_0,\ldots,k_4\in\{0,1\}^{wn}$ uniformly and independently at random;
	\item[2.] sample two random permutations $S_1,S_4$ from $\mathsf{Perm}(n)$ at uniform, such that $S_1\vdash\mathcal{Q}_{S_1}^{(0)},S_4\vdash\mathcal{Q}_{S_4}^{(0)}$.
	\item[3.] choose the partial extension of the S-box queries based on the new collisions $\mathcal{Q}_{S}^{\prime}$ uniformly at random (meaning that each possible $\mathnormal{u}$ or $\mathnormal{v}$ is chosen uniformly at random in the set of its authorized values).
\end{itemize}
%
%
This means the term $B$ captures the probability of good transcript extensions:
%
%
\begin{align}
B=\sum_{\tau^{\prime} \in \Theta_{\mathrm{good}}(\tau)}
	\frac{1}{N^{5w}\big((N-p)_{N-p}\big)^2(N-p)_{\alpha_1}(N-p)_{\alpha_2}}=\operatorname{Pr}\big[ \tau' \in \Theta_{\text {good }}(\tau)\big],   \notag
\end{align}
%
%
which further implies
%
%
\begin{align}
\frac{\mathsf{p}_{2}(\tau)}{\mathsf{p}_{1}(\tau)}   \geq  \Pr\big[ \tau' \in \Theta_{\text {good }}(\tau)\big]\cdot
  \min_{\tau' \in \Theta_{\mathrm{good}}(\tau)}\big((N^w)_q\cdot\mathsf{p}_{\mathrm{mid}}(\tau')\big). 
\label{eq:ratio-divide-4-rounds}
\end{align}
%
%


The term $\mathsf{p}_{\mathrm{mid}}(\tau')$ captures the probability that $\calC_{\bfk'}^T[\calS] \vdash \mathcal{Q}_C'$, i.e., the inner two SPN rounds are consistent with the pairs of inputs/outputs $(a,b)$ defined in $\mathcal{Q}_C'$. We appeal to~\cite{C:CDKLST18} to have a concrete bound on $(N^w)_q\cdot\mathsf{p}_{\mathrm{mid}}(\tau')$.

\begin{lemma}
	\label{lemma:bound-middle-two-rounds}
	
	Assume $p+wq\leq N/2$, then
	\begin{align}
	(N^w)_q\cdot\mathsf{p}_{\mathrm{mid}}(\tau') \geq 1-\frac{q^2}{N^w}-\frac{q(2wp+6w^2q)^2}{N^2}.
	\label{eq:bound-on-epsilon-mid}
	\end{align}
\end{lemma}
\begin{proof}
It can be checked that, the transcript $(\mathcal{Q}_C',\mathcal{Q}_{S_2}^{(1)},\mathcal{Q}_{S_3}^{(1)})$ satisfies exactly the conditions defining a good transcript as per~\cite[page 740]{C:CDKLST18}. Moreover,
the ratio $\mathsf{p}_{\mathrm{mid}}(\tau')/(1/(N^w)_q)$ is exactly the ratio of the probabilities to get $\tau'$ in the real and in the ideal world. The result thus immediately follows from~\cite[Lemma 9]{C:CDKLST18}.       \qed
\end{proof}



%The previous proof is conditioned on $S_{1} \vdash \mathcal{Q}_{S_{1}}, S_{4} \vdash \mathcal{Q}_{S_{4}}$, but $\operatorname{Pr}\left[ \tau_{inner}^{\prime} \in \Theta_{\text {good }}(\tau_{inner})\right]$, we need to consider $S_{1} \vdash \mathcal{Q}_{S_{1}}^{(1)}, S_{4} \vdash \mathcal{Q}_{S_{4}}^{(1)}$. That is the probability $\left(T\left(S_{1}\left(x \oplus k_{0}\right) \oplus k_{1}\right)\right)[i]=u_2$ or $\left(T^{-1}\left(S_{4}^{-1}\left(y \oplus k_{4}\right)\right) \oplus k_{3}\right)[j]=v_3$ hold is at most $\frac{1}{(N-p-w q)}$, so
%
%\begin{equation}
%\begin{aligned}
%\operatorname{Pr}\left[ \tau_{inner}^{\prime} \in \Theta_{\text {good }}(\tau_{inner})\right] \geq 1&- \frac{2 w^{2} q (p+w q)^{2}}{(N-p-w q)} -\frac{2 w^{2} q (p+w q)(p+w q+2 q)}{N \cdot (N-p-w q)}\\
%&- \frac{w^{2} q (p+w q)(p+w q+2 q)}{(N-p-w q)^2} - \frac{2 w^{2} q^{2} (p+w q)}{(N- p- wq)^2}.
%\end{aligned}
%\end{equation}



Gathering Eqs. (\ref{eq:bound-bad-tau-4-rounds}), (\ref{eq:ratio-divide-4-rounds}), and (\ref{eq:bound-on-epsilon-mid}), we obtain
%
\begin{align*}
\frac{\mathsf{p}_{2}(\tau)}{\mathsf{p}_{1}(\tau)}   \geq   & \bigg(1-\frac{3w^{2} q \left(p+w q\right)^{2}}{N^{2}} - \frac{w^{2} q}{N} - \frac{9w^2 q (p+w q)^{2}}{N^2}- \frac{16w^3q^2p}{N^2}\bigg)\cdot\bigg(1-\frac{q^2}{N^w}-\frac{q(2wp+6w^2q)^2}{N^2}\bigg)     \notag      \\
\geq  &  1-\frac{3w^{2} q \left(p+w q\right)^{2}}{N^{2}} - \frac{w^{2} q}{N} - \frac{9w^2 q (p+w q)^{2}}{N^2}- \frac{16w^3q^2p}{N^2}-\frac{q^2}{N^w}-\frac{q(2wp+6w^2q)^2}{N^2}     \notag   
\end{align*}
%
as claimed in Eq. (\ref{eq:bound-proximity-4-round}).
