
\section{Introduction}
\label{section:Introduction}


\noindent{\bf Substitution-Permutation Networks.}
Most modern blockciphers are

built via two different generic structures: Feistel networks or substitution-permutation
networks (SPNs). These two approaches revolve around the extension of a ``complex'' function or permutation on a small domain to a keyed pseudorandom
permutation on a larger domain by iterating several times simple rounds.
SPNs start with a set of public permutations on the set of n-bit strings which
are called S-boxes. These public permutations are then extended to a keyed
permutation on wn-bit inputs for some integer w by iterating the following steps:
\begin{enumerate}
	\item[1.] break down the state in w n-bit blocks;
	\item[2.] compute an S-box on each block of the state;
	\item[3.] apply a keyed permutation layer to the whole wn-bit state (which is also applied to the plaintext before the first round).
\end{enumerate}


Many well-known block ciphers including AES, Serpent and PRESENT follow
this approach.


Proving the security of a particular concrete block cipher is
currently beyond our techniques. Thus, the usual approach is to prove that the
high-level structure is sound in a relevant security model. As for Feistel networks,
a substantial line of work starting with Luby and Rackoff's seminal work [LR88]
and culminating with Patarin's results [Pat03, Pat04] proves optimal security with
a sufficient number of rounds. Numerous other articles [Pat10, HR10, HKT11,
Tes14, CHK+16] study the security of (variants of) Feistel networks in various
security models. On the other hand, SPNs have comparatively seen very little
interest which seems rather surprising.





\medskip\noindent{\bf Tweaking SPNs.}
Recently, a similar study
was undertaken for the second large class of block ciphers besides Feistel ciphers, namely
key-alternating ciphers [DR01], a super-class of Substitution-Permutation Networks (SPNs).
An r-round key-alternating cipher based on a tuple of public n-bit permutations $(P_1, . . . , P_r)$
maps a plaintext $x\in\{0,1\}^n$ to the ciphertext defined as
%
%
%
%
%
where the n-bit round keys $k_0, . . . , k_r$ are either independent or derived from a master key k.
When the $P_i$'s are modeled as public permutation oracles, construction (1) is also referred to
as the (iterated) Even-Mansour construction, in reference to Even and Mansour who pioneered
the analysis of this construction in the Random Permutation Model [EM97]. While Even
and Mansour limited themselves to proving birthday-bound security in the case r = 1, larger
numbers of rounds were studied in subsequent works [BKL+12, Ste12, LPS12]. The general
case has been recently (tightly) settled by Chen and Steinberger [CS14], who proved that the
r-round iterated Even-Mansour cipher with r-wise independent round keys ensures security
up to roughly $2^{rn}$
r+1 adversarial queries.
In order to incorporate a tweak t in the iterated Even-Mansour construction, it is tantalizing
to generalize (1) by replacing round keys $k_i$ by some function $f_i(\bfk,\bft)$ of the master key $\bfk$ and
the tweak $\bft$ (see Figure 1). We will refer to such a construction as a Tweakable Even-Mansour
(TEM) construction. This is exactly the spirit of the TWEAKEY framework introduced by
Jean et al. [JNP14]. In fact, these authors go one step further and propose to unify the key
and tweak inputs into what they dub the tweakey. The main topic of this paper being provable
security (in the traditional model where the key is secret and the tweak is chosen by the
adversary), we will not make such a bold move here, since we are not aware of any formal
security model adequately capturing what Jean et al. had in mind.



The investigation of the theoretical soundness of this design strategy was initiated in three
recent papers. First, Cogliati and Seurin [CS15], and independently Farshim and Procter [FP15],
analyzed the simple case of an n-bit key k and an n-bit tweak t simply xored together at each
round, i.e., $f_i(k,t) = k_t$ for each $i = 0, . . . , r$. They gave attacks up to two rounds, and proved
birthday-bound security for three rounds. In fact, the security of this construction caps at $2^{n/2}$
queries independently of the number of rounds. Indeed, it can be written $\widehat{E}(k, t, x) = E(k\xor t, x)$,
where $E$ is the conventional iterated Even-Mansour cipher with the trivial key-schedule (i.e.,
the same round key is xored between each round), and by a result of Bellare and Kohno [BK03,
Corollary 5.7], a tweakable block cipher of this form can never offer more than $\kappa/2$ bits of
security, where $\kappa$ is the key-length of E (i.e., $\kappa = n$ in the case at hand). Hence, if we want
beyond-birthday-bound security, we have no choice but to consider more complex functions $f_i$
(at the bare minimum, these functions, even if linear, should prevent the TBC construction
from being of the form $E(k\xor t, x)$ for some block cipher E with n-bit keys).




This was undertaken by Cogliati, Lampe, and Seurin [CLS15], who considered nonlinear
ways of mixing the key and the tweak. More specifically, they studied the case where $f_i(\bfk, t) =
H_{k_i}(t)$, where the family of functions $(H_k)$ is uniform and almost XOR-universal, and the
master key is $\bfk = (k_0, . . . , k_r)$. Cogliati et al. showed that one round is secure up to the birthday bound,
and that two rounds are secure up to roughly $2^{2n/3}$ adversarial queries. They also provided a
(non-tight) asymptotic security bound improving as the number of rounds grows. However,
implementing a xor-universal hash function might be costly, and linear functions $f_i$'s would be highly preferable for obvious efficiency reasons.




\subsection{Our Results}

%In this paper, we ask whether it is possible to come with a tweakable Even-
%Mansour construction achieving both:
%1. a linear mixing of the tweak and the key to the state;
%2. beyond-birthday-bound security.
%We answer positively, by providing a construction with 2n-bit keys and n-bit tweaks.

In this paper, we ask whether it is possible to achieve security beyond the birthday barrier with linear SPN structures. In detail, we focus on linear SPNs with independent S-boxes and independent round keys, and we will focus on the case where $w\geq2$, since, when $w = 1$, we recover the standard Even-Mansour construction that has already been the focus
of a long line of work (as briefly reviewed later). For such linear SPNs, we prove the first beyond-birthday-bound (BBB) result on 4 rounds. To tweak such linear SPNs, we consider the simplest approach, i.e., directly xoring a $wn$-bit tweak with each round key, and prove BBB result on 6 rounds. We will elaborate in detail as follows.



\medskip\noindent{\bf BBB Security for 4-round linear SPNs.}



\medskip\noindent{\bf Tweaking linear SPNs and BBB Security at 6 rounds.}








\subsection{Related Work}


Unfortunately, none of the currently known black-box
TBC constructions with beyond-birthday-bound security can be deemed truly practical (even
though some of them might come close to it [Men15]). Hence, it might be beneficial to “open
the hood” and to study how to build a TBC from some lower level primitive than a full-fledged
conventional block cipher, e.g., a pseudorandom function or a public permutation. For example,
Goldenberg et al. [GHL+07] investigated how to include a tweak in Feistel ciphers. This was
extended to generalized Feistel ciphers by Mitsuda and Iwata [MI08].



\floatstyle{boxed}
\restylefloat{figure}

