
\section{Introduction}
\label{section:Introduction}



\medskip\noindent{\bf Substitution-Permutation Networks.}

Nowadays block ciphers are mainly
built around two different generic structures: Feistel networks or substitution-permutation
networks (SPNs). These two approaches revolve around the extension of a ``complex'' function or permutation on a small domain to a keyed pseudorandom
permutation on a larger domain by iterating several times simple rounds.
SPNs start with a set of public permutations on the set of n-bit strings which
are called S-boxes. These public permutations are then extended to a keyed
permutation on wn-bit inputs for some integer w by iterating the following steps:
\begin{enumerate}
	\item[1.] break down the state in w n-bit blocks;
	\item[2.] compute an S-box on each block of the state;
	\item[3.] apply a keyed permutation layer to the whole wn-bit state (which is also applied to the plaintext before the first round).
\end{enumerate}


Many well-known block ciphers including AES, Serpent and PRESENT follow
this approach. Proving the security of a particular concrete block cipher is
currently beyond our techniques. Thus, the usual approach is to prove that the
high-level structure is sound in a relevant security model. As for Feistel networks,
a substantial line of work starting with Luby and Rackoff's seminal work [LR88]
and culminating with Patarin’s results [Pat03, Pat04] proves optimal security with
a sufficient number of rounds. Numerous other articles [Pat10, HR10, HKT11,
Tes14, CHK+16] study the security of (variants of) Feistel networks in various
security models. On the other hand, SPNs have comparatively seen very little
interest which seems rather surprising.





\medskip\noindent{\bf Tweaking SPNs.}

Recently, a similar study
was undertaken for the second large class of block ciphers besides Feistel ciphers, namely
key-alternating ciphers [DR01], a super-class of Substitution-Permutation Networks (SPNs).
An r-round key-alternating cipher based on a tuple of public n-bit permutations (P1, . . . , Pr)
maps a plaintext x 2 {0, 1}n to the ciphertext defined as









\medskip\noindent{\bf Related Work.}


Unfortunately, none of the currently known black-box
TBC constructions with beyond-birthday-bound security can be deemed truly practical (even
though some of them might come close to it [Men15]). Hence, it might be beneficial to “open
the hood” and to study how to build a TBC from some lower level primitive than a full-fledged
conventional block cipher, e.g., a pseudorandom function or a public permutation. For example,
Goldenberg et al. [GHL+07] investigated how to include a tweak in Feistel ciphers. This was
extended to generalized Feistel ciphers by Mitsuda and Iwata [MI08].



\floatstyle{boxed}
\restylefloat{figure}

