

\section{TSPRP Security of 6-Round Tweakable Linear SPNs}
\label{section:beyond birthday bound for tweakable linear SPNs}

In this section, we prove beyond-birthday-bound STPRP security for 6-round tweakable linear SPNs. Concretely, let $\spn_{\bfk}[\mathcal{S}]$ be the 6-round SPN using any linear transformations $T$. I.e.,
%
\begin{align}
\spn_{\bfk}^T[\mathcal{S}](x):=k_6\xor t\xor&\overline{S_6}(k_5\xor t\xor  T(\overline{S_5}(k_4\xor t\xor\overline{S_4}(k_3\xor t\xor T(\overline{S_3}(k_2\xor t\xor T(    \notag    \\
&  \overline{S_2}(k_1\xor t\xor T(\overline{S_1}(k_0\xor t\xor x)))))))))).
\label{eq:defn-6-round-spn}
\end{align}
%
We show that $\spn^T$ is an STPRP as long as: (i) the linear layer $T$ contains no zero entries, and (ii) the round keys $\{k_i\}(i=0, \ldots , 6)$ are uniform and independent, and (iii) the tweak $t$ is directly xored into each round key.


\begin{theorem}
\label{theorem:6-rounds}

Assume $w\geq2$, and $p+wq\leq N/2$. Let $\spn_{\bfk}[\mathcal{S}]$ be a 6-round, tweakable linear SPN as defined by Eq. (\ref{eq:defn-6-round-spn}). If round keys $\bfk=(k_0,\ldots,k_6)$ are uniform and independent, and $T$ contains no zero entries, then
%
\begin{align}
\operatorname{Adv}_{\spn^T}^{\mathrm{su}}(p, q) \leq& \frac{q^2}{2^{n w}} + \frac{8 w^2 q(p+wq)^2+w^2 q}{2^n}   .  
\notag   \\
\operatorname{Adv}_{\spn^T}^{\mathrm{mu}}(p, q) \leq& \frac{q^2}{2^{n w}} + \frac{8 w^2 q(p+wq)^2+w^2 q}{2^n}    \notag   \\
&+ \frac{16 w^2 q(p+w q)(p+w q +3 q)+4 w^2 q(p+3 wq)^2}{2^{2 n}}.
\notag
\end{align}
\end{theorem}

For the proof, we rely on the following lemma and Lemmas \ref{lemma:h-coeff} and \ref{lemma:point-wise}.


\begin{lemma}
	\label{lemma:proximity-6-round}
	
	Assume $p+wq\leq N/2$. Let $\dis$ be a distinguisher in the single-user setting that makes $p$ primitive queries to each of $S_1,\ldots,S_6$, and makes $q$ construction queries. Then for any attainable
	transcript $\tau=(\mathcal{Q}_C,\mathcal{Q}_S)$, one has
	\begin{align}
	\frac{\mathsf{p}_{2}(\tau)}{\mathsf{p}_{1}(\tau)}
	\geq 1 - xxx.
	\label{eq:bound-proximity-6-round}
	\end{align}
\end{lemma}



Similar to the 4-round case, we will first define the notion of extended transcripts. We then define bad extensions and bound their probability. The ratio $\mathsf{p}_{2}(\tau)/\mathsf{p}_{1}(\tau)$ is then derived similarly to the 4-round case.



\arrangespace

\noindent \textbf{Transcript Extension for 6 rounds}.
Fix a distinguisher $\mathcal{D}$ as described in the statement and fix an attainable transcript $\tau =\left(\mathcal{Q}_{C}, \mathcal{Q}_{S}\right)$ obtained $\mathcal{D}$. For $i \in \{1, \cdots, 6\}$, let

$$
\begin{aligned}
&\mathcal{Q}_{S_{i}}^{(0)}=\left\{(u, v) \in\{0,1\}^{n} \times\{0,1\}^{n}:(i, u, v) \in \mathcal{Q}_{S} \right\},\\
\end{aligned}
$$

\noindent and let

$$
\begin{aligned}
&U_{i}^{(0)}=\left\{u_{i} \in\{0,1\}^{n}:\left(i, u_{i}, v_{i}\right) \in \mathcal{Q}_{S_{i}}^{(0)}\right\}\\
&V_{i}^{(0)}=\left\{v_{i} \in\{0,1\}^{n}:\left(i, u_{i}, v_{i}\right) \in \mathcal{Q}_{S_{i}}^{(0)}\right\},\\
\end{aligned}
$$
%
denote the domains and ranges of $\mathcal{Q}_{S_{i}}^{(0)}, i \in \{1, \cdots, 6\}$, respectively.









In detail, a transcript $\tau$ is extended in the following manner:
\begin{itemize}
	\item At the end of the interaction between \dis and the real world $(\mathcal{S},\spn_{\bfk}^T[\mathcal{S}])$, we append $\tau$ with the keys $\bfk=(k_0,k_1,k_2,k_3,k_4)$ and the two random permutations $S_1,S_4$ in use;
	\item At the end of the interaction between \dis and the ideal world $(\mathcal{S},\widetilde{P})$, we append $\tau$ with randomly sampled keys $\bfk=(k_0,k_1,k_2,k_3,k_4)$ and the two random permutations $S_1,S_4$ in use.
\end{itemize}
Note that, in either case, it is equivalent to sampling two new random permutations $S_1,S_4$ such that $S_1\vdash\mathcal{Q}_{S_{1}}$ and $S_4\vdash\mathcal{Q}_{S_4}$ and appending them to $\tau$. With the above, for any $(t,x,y)\in\mathcal{Q}_C$ we define $\vect=(t,x,a,c,d,b,y)$, where
%
\begin{align*}
&a=T\big(\overline{S_1}\left(x \oplus k_{0}\right)\big),  
&c=T\big(\overline{S_2}\left(a \oplus k_1\right)\big),              \\
&d=T^{-1}\big(\overline{S_3^{-1}}\left(b \oplus k_3\right)\big), 
&b=T^{-1}\big(\overline{S_{4}^{-1}}\left(y \oplus k_{4}\right)\big).
\end{align*}
%
This extends the list $\mathcal{Q}_C$ into
$\mathcal{Q}_C'=\big(\vect_1,\ldots,\vect_q\big)$, where $\vect_i=(t_i,x_i,a_i,c_i,d_i,b_i,y_i)$. Then, a colliding query is defined as a construction query $\vect=(x,a,c,d,b,y)\in\mathcal{Q}_C'$ as follows:
%
\begin{itemize}
	\item[1.]
	there exist an S-box query $(u,v)\in\mathcal{Q}_{S_2}^{(0)}$ and an integer $i \in\{1, \ldots, w\}$ such that $\left(a \oplus k_1\right)[i]=u$.
	\item[2.]
	there exist an S-box query $(u,v)\in\mathcal{Q}_{S_3}^{(0)}$ and an integer $i \in\{1, \ldots, w\}$ such that $\left(b \oplus T^{-1}(k_3)\right)[i]=v$.
	\item[3.] there exist a construction query $\left(a^{\prime}, b^{\prime}\right) \in \mathcal{Q}_{C}$ and an integer $i,j \in\{1, \ldots, w\}$ such that $(a, b, i) \neq\left(a^{\prime}, b^{\prime}, j\right)$ and $\left(a \oplus k_1\right)[i] = \left(a' \oplus k_1\right)[j]$.
	\item[4.] there exist a construction query $\left(a^{\prime}, b^{\prime}\right) \in \mathcal{Q}_{C}$ and an integer $i,j \in\{1, \ldots, w\}$ such that $(a, b, i) \neq\left(a^{\prime}, b^{\prime}, j\right)$ and $i \in\{1, \ldots, w\}$ such that $\left(b \oplus T^{-1}(k_3)\right)[i] = \left(b' \oplus T^{-1}(k_3)\right)[j]$.
\end{itemize}
%
%
Now we further introduce a new set $\mathcal{Q}_{S}'$ of S-box evaluations to complete the transcript extension. In detail, for each colliding query $(x,a,b,y)\in\mathcal{Q}_C'$, we will add tuples $\left(2, (a \oplus k_1)[i], v^{\prime}\right)_{1 \leq i \leq w}$ (if $(a, b)$ collides at the input of $S_2$) or $\left(3, u^{\prime}, (b \oplus T^{-1}(k_3))[i]\right)_{1 \leq i \leq w}$ (if $(a, b)$ collides at the output of $S_3$) to $\mathcal{Q}_{S}'$ by lazy sampling $v^{\prime}=S_2(\left(a \oplus k_1\right)[i])$ or $u^{\prime}=S_3^{-1}(\left(b \oplus k_3\right)[i])$, as long as it has not been determined by any existing query in $\mathcal{Q}_S$.


An extended transcript of $\tau$ includes all the above additional information, i.e.,
%
$$\tau'=(\mathcal{Q}_{C}',\mathcal{Q}_{S},\mathcal{Q}_{S}',S_1,S_2,S_5,S_6,\bfk).$$
%
For each collision between a construction query and a primitive query, or between two construction queries, the extended transcript will contain enough information to compute a complete round of the evaluation of the SPN. This will be useful to lower bound the probability to get the transcript $\tau$ in the real world.




%We are now going to build a new set $\mathcal{Q}_{S_{outmost}}^{\prime}$ of S-box evaluations that will play the role of an extension of $\mathcal{Q}_{S}$. For each colliding query $(t, x, y) \in \mathcal{Q}_{C}$, we will add tuples $\left(1, \left(x \oplus k_{0} \oplus t\right)[i], v_1^{\prime}\right)_{1 \leq i \leq w}$ (if ($t, \mathit{x}$, $\mathit{y}$) collides at the input of $S_1$), or (if ($t, \mathit{x}, \mathit{y}$) collides at the output of $S_6$) $\left(6, u_6^{\prime}, \left(y \oplus k_{6} \oplus t\right)[i]\right)_{1 \leq i \leq w}$, by lazy sampling $v_1^{\prime}=S_{1}(\left(x \oplus k_{0} \oplus t\right)[i])$, or $u_6^{\prime}=S_{6}^{-1}(\left(y \oplus k_{6} \oplus t\right)[i])$, as long as it has not been determined by any existing query in $\mathcal{Q}_{S}$. Then we choose the key $k_1, k_2, k_3, k_4, k_5$ uniformly at random. An extended transcript of $\tau$ will be defined as a tuple $\tau^{\prime}=\left(\mathcal{Q}_{C}, \mathcal{Q}_{S}, \mathcal{Q}_{S_{outmost}}^{\prime}, \mathbf{k}\right)$ where $\mathbf{k}=\left(k_{0}, k_{1}, k_{2},k_{3},k_{4}, k_{5}, k_{6}\right)$. For each collision between a construction query and a primitive query, or between two construction queries, the extended transcript will contain enough information to compute a complete round of the evaluation of the SPN. This will be useful to lower bound the probability to get the transcript $\tau$ in the real world.








%We define two quantities characterizing an extended transcript $\tau^{\prime}$, namely
%
%$$
%\begin{aligned}
%&\alpha_{1} \stackrel{\text { def }}{=} |\left\{(t, x, y) \in \mathcal{Q}_{C}: \left(x \oplus k_{0} \oplus t\right)[i] \in U_{1} \text { for some } i \in\{1, \ldots, w\}\right\} |\\
%&\alpha_{6} \stackrel{\text { def }}{=} |\left\{(t, x, y) \in \mathcal{Q}_{C}: \left(y \oplus k_{6} \oplus t\right)[i] \in V_{6} \text { for some } i \in\{1, \ldots, w\}\right\} |
%\end{aligned}
%$$
%
%In words, $\alpha_1$ (resp. $\alpha_6$) is the number of queries $(t, x, y) \in \mathcal{Q}_{C}$ which collide with a query $\left(u_{1}, v_{1}\right) \in \mathcal{Q}_{S_{1}}^{(1)}$ (resp. $\left(u_{6}, v_{6}\right) \in \mathcal{Q}_{S_{6}}^{(1)}$) in the extended transcript. This corresponds to the number of queries $(t, x, y) \in \mathcal{Q}_{C}$ which collide with either an original query $\left(u_{1}, v_{1}\right) \in \mathcal{Q}_{S_{1}}^{(0)}$ (resp. which collide with a query $\left(u_{6}, v_{6}\right) \in \mathcal{Q}_{S_{6}}^{(0)}$) or with a query $\left(t' x^{\prime}, y^{\prime}\right) \in \mathcal{Q}_{C}$ at an input of $S_1$ (resp. at the output of $S_6$), once the choice of $\left(k_{0}, k_{6}\right)$  has been made. We will also denote
%
%$$
%\beta_{i}=\left|\mathcal{Q}_{S_{i}}^{(1)}\right|-\left|\mathcal{Q}_{S_{i}}^{(0)}\right|=\left|\mathcal{Q}_{S_{i}}^{(1)}\right|-p.
%$$
%
%for $i=1, 6$ the number of additional queries included in the extended transcript.


%\subsection{Bad Transcript for 6-rounds tweakable linear SPN}




\arrangespace

\noindent \textbf{Bad Extension for 6 rounds}.
%
%
Consider an attainable extended transcript $\tau'=(\mathcal{Q}_{C}',\mathcal{Q}_{S},\mathcal{Q}_{S}',S_1,S_2,S_5,S_6,\bfk)$. Let
%
$$
\begin{aligned}
&\mathcal{Q}_{S_3}^{(1)}=\left\{(u, v) \in\{0,1\}^{n} \times\{0,1\}^{n}:(2, u, v) \in \mathcal{Q}_{S} \cup \mathcal{Q}_{S}^{\prime}\right\}\\
&\mathcal{Q}_{S_4}^{(1)}=\left\{(u, v) \in\{0,1\}^{n} \times\{0,1\}^{n}:(3, u, v) \in \mathcal{Q}_{S} \cup \mathcal{Q}_{S}^{\prime}\right\}.
\end{aligned}
$$
%
In words, $\mathcal{Q}_{S_{i}}^{(1)}$ summarizes each constraint that is forced on $S_{i}$ by $\mathcal{Q}_{S}$ and $\mathcal{Q}_{S}^{\prime}$. Let        {\small
%
$$
\begin{aligned}
&U_3^{(1)}=\left\{u_2 \in\{0,1\}^{n}:\left(2, u_2, v_2\right) \in \mathcal{Q}_{S_2}^{(1)}\right\}, \quad V_3^{(1)}=\left\{v_2 \in\{0,1\}^{n}:\left(2, u_2, v_2\right) \in \mathcal{Q}_{S_2}^{(1)}\right\},\\
&U_4^{(1)}=\left\{u_3 \in\{0,1\}^{n}:\left(3, u_3, v_3\right) \in \mathcal{Q}_{S_3}^{(1)}\right\}, \quad V_4^{(1)}=\left\{v_3 \in\{0,1\}^{n}:\left(3, u_3, v_3\right) \in \mathcal{Q}_{S_3}^{(1)}\right\}.
\end{aligned}
$$
}%
%
be the domains and ranges of $\mathcal{Q}_{S_3}^{(1)}$ and $\mathcal{Q}_{S_4}^{(1)}$ respectively.




\begin{definition}
	\label{defn:bad-tau-6-rounds}
	
	We say an extended transcript $\tau^{\prime}$ is bad if at least one of the following conditions is fulfilled. The conditions are classified into two categories depending on the relevant randomness. In detail, regarding $k_0,k_1,k_2,k_4,k_5,k_6$:
	\begin{itemize}[leftmargin=10mm]
		\item[\cone] there exist (not necessarily distinct) $\vect,\vect',\vect''\in \mathcal{Q}_{C}'$ and distinct $i, i', i'' \in \{1, \ldots, w\}$ such t hat any of the following holds:
		\begin{itemize}
			\item $(x\xor t\xor k_0)[i]=(x'\xor t'\xor k_0)[i']=(x''\xor t''\xor k_0)[i'']$;
			\item $(a\xor t\xor k_1)[i]=(a'\xor t'\xor k_1)[i']=(a''\xor t''\xor k_1)[i'']$;
			\item $(c\xor t\xor k_2)[i]=(c'\xor t'\xor k_2)[i']=(c''\xor t''\xor k_2)[i'']$;
			\item $(d\xor t\xor T^{-1}(k_4))[i]=(c'\xor t'\xor T^{-1}(k_4))[i']=(c''\xor t''\xor T^{-1}(k_4))[i'']$;
			\item $(b\xor t\xor T^{-1}(k_5))[i]=(b'\xor t'\xor T^{-1}(k_5))[i']=(b''\xor t''\xor T^{-1}(k_5))[i'']$;
			\item $(y\xor t\xor k_6)[i]=(y'\xor t'\xor k_6)[i']=(y''\xor t''\xor k_6)[i'']$.
		\end{itemize}
		\item[\ctwo] there exist $\vect \in \mathcal{Q}_{C}'$ and distinct indices $i, i' \in \{1, \ldots, w\}$ such that:
		\begin{itemize}
			\item $(x\xor t\xor k_0)[i]\in U_1^{(0)}$ and $(x\xor t\xor k_0)[i']\in U_1^{(0)}$, or
			\item $(a\xor t \oplus k_1)[i]\in U_{2}^{(0)}$ and $(a\xor t \oplus k_1)[i']\in U_{2}^{(0)}$, or
			\item $(c\xor t \oplus k_2)[i]\in U_{3}^{(0)}$ and $(c\xor t \oplus k_2)[i']\in U_{3}^{(0)}$, or
			\item $(d\xor t\xor T^{-1}(k_4))[i]\in V_4^{(0)}$ and
			$(d\xor t\xor T^{-1}(k_4))[i']\in V_4^{(0)}$, or
			\item $(b\xor t\xor T^{-1}(k_5))[i]\in V_5^{(0)}$ and
			$(b\xor t\xor T^{-1}(k_5))[i']\in V_5^{(0)}$, or
			\item $(y\xor t\xor k_6)[i]\in V_6^{(0)}$ and $(y\xor t\xor k_6)[i']\in V_6^{(0)}$.
		\end{itemize}
		\item[\cthree] there exist $\vect \in \mathcal{Q}_{C}'$ and $i, j \in \{1, \ldots, w\}$ such that:
		\begin{itemize}
			\item $(x\xor t\xor k_0)[i]\in U_1^{(0)}$ and $(y\xor t\xor k_6)[j]\in V_6^{(0)}$, or
			\item $(x\xor t\xor k_0)[i]\in U_1^{(0)}$ and $(a\xor t \oplus k_1)[j]\in U_{2}^{(0)}$, or
			\item $(a\xor t \oplus k_1)[i]\in U_{2}^{(0)}$ and $(c\xor t \oplus k_2)[j]\in U_{3}^{(0)}$, or
			\item $(d\xor t\xor T^{-1}(k_4))[i]\in V_4^{(0)}$ and
			$(b\xor t\xor T^{-1}(k_5))[j]\in V_5^{(0)}$, or
			\item $(b\xor t\xor T^{-1}(k_5))[i]\in V_5^{(0)}$ and
			$(y\xor t\xor k_6)[j]\in V_6^{(0)}$.
		\end{itemize}
	\end{itemize}
	%
	%
	Regarding $k_2,S_1,S_4$, and $\mathcal{Q}_S'$:
	%
	%
	\begin{itemize}[leftmargin=10mm]
		\item[\cfour] there exist $\vect \in \mathcal{Q}_{C}'$ and $i, j \in\{1, \ldots, w\}$ such that:
		\begin{itemize}
			\item $(c\xor k_2\xor t)[i]\in U_3^{(1)}$ and $(d\xor T^{-1}(k_4\xor t))[j]\in V_4^{(1)}$, or
			\item $(c\xor k_2\xor t)[i]\in U_3^{(1)}$ and $(T(\overline{S_3}(c\xor k_2\xor t))\xor k_3\xor t)[j]\in U_4^{(1)}$, or
			\item $(T^{-1}(\overline{S_4^{-1}}(d\xor T^{-1}(k_4\xor t))\xor k_3\xor t))[i]\in V_3^{(1)}$ and $(d\xor T^{-1}(k_4\xor t))[j]\in V_4^{(1)}$.
		\end{itemize}
		\item[\cfive] there exist $\vect,\vect' \in \mathcal{Q}_{C}'$ and $i, i^{\prime},j, j^{\prime} \in\{1, \ldots, w\}$, $(a,b, j) \neq \left(a^{\prime}, b',j^{\prime}\right)$, such that $(a \oplus k_2\xor t)[i]\in U_3^{(1)}, (a' \oplus k_2\xor t)[i']\in U_3^{(1)}$, and
		%
		$$\big(T(\overline{S_2}(a\xor k_1\xor t))\xor k_2\xor t\big)[j]=\big(T(\overline{S_2}(a'\xor k_1\xor t))\xor k_2\xor t\big)[j'].
		$$
		\item[\csix] there exist $\vect,\vect' \in \mathcal{Q}_{C}'$ and $i, i^{\prime}, j, j^{\prime} \in\{1, \ldots, w\}$, $(a,b, j) \neq \left(a',b^{\prime}, j^{\prime}\right)$, such that $\big(b \oplus T^{-1}(k_3\xor t)\big)[i]\in V_4^{(1)}, \big(b' \oplus T^{-1}(k_3\xor t)\big)[i']\in V_4^{(1)}$, and         {\small
		%
		$$\big(T^{-1}(\overline{S_3^{-1}}(b \oplus T^{-1}(k_3\xor t))\xor k_2\xor t)\big)[j]=\big(T^{-1}(\overline{S_3^{-1}}(b' \oplus T^{-1}(k_3\xor t))\xor k_2\xor t)\big)[j'].
		$$
	}%
	\end{itemize}
	Any extended transcript that is not bad will be called good. Given an original transcript $\tau$, we denote $\Theta_{\mathrm{good}}(\tau)$ (resp. $\Theta_{\mathrm{bad}}(\tau)$) the set of good (resp. bad) extended transcripts of $\tau$ and $\Theta'(\tau)$ the set of all extended transcripts of $\tau$.
\end{definition}



%
%
%\begin{definition}
%	\label{defn:bad-tau-6-rounds}
%	
%	We say an extended transcript $\tau^{\prime}$ is bad if at least one of the following conditions is fulfilled:
%	\begin{itemize}
%		\item[\bone] 
%	\end{itemize}
%	Any extended transcript that is not bad will be called good.
%\end{definition}


\begin{lemma}
	\label{lemma:bad-tau-6-rounds}
	
	One has
	\begin{align}
	{\Pr}\big[\tau^{\prime} \in \Theta_{bad}(\tau)\big] \leq \frac{3w^{2} q \left(p+w q\right)^{2}}{N^{2}} + \frac{w^{2} q}{N}.
	\label{eq:bound-bad-tau-6-rounds}
	\end{align}
\end{lemma}
\begin{proof}
The analyses of the conditions just follow Lemma \ref{lemma:bad-tau-4-rounds} and thus we omit. The crux lies in the conditions \cfour, \cfive, and \csix.


\newpage

\arrangespace

\noindent \textsc{\cfour}. Consider any of the $w^2q$ choices of $(t,x,a,b,y)$ and $i, j$. We define five predicates as follows.
%
\begin{itemize}
	\item $\pred_1(\vect,i)$ is fulfilled, if
	\begin{itemize}
		\item $(c\xor k_2\xor t)[i]\in U_3^{(0)}$, or
		\item there exists $\vect'$ and $i'\neq i$ such that $(c\xor k_2\xor t)[i]=(c'\xor k_2\xor t')[i']$.
	\end{itemize}
	\item $\pred_2(\vect,i)$ is fulfilled, if there exists $(t',x',a',b',y')\neq(t,x,a,b,y)$ such that $(c\xor t)[i]=(c'\xor t')[i]$.
	\item $\pred_3(\vect,i)$ is fulfilled, if
	\begin{itemize}
		\item $(d\xor T^{-1}(k_4\xor t))[j]\in V_4^{(0)}$, or
		\item there exists $\vect'$ and $j'\neq j$ such that $(d\xor T^{-1}(k_4\xor t))[j]=(d'\xor T^{-1}(k_4\xor t'))[j']$.
	\end{itemize}
	\item $\pred_4(\vect,i)$ is fulfilled, if there exists $(t',x',a',b',y')\neq(t,x,a,b,y)$ such that $(d\xor T^{-1}(t))[j]=(d'\xor T^{-1}(t'))[j]$.
	\item $\free(\vect)$ is fulfilled, if $(a\xor t\xor k_1)[i_0]\notin U_1^{(1)}$ for all $i\in\{1,\ldots,w\}$.
\end{itemize}
%
Since $k_2[i]$ and $k_2[i']$ are uniform and independent, it is easy to see $\Pr[\pred_1(\vect,i)]\leq(p+wq)/N$ and $\Pr[\pred_3(\vect,i)]\leq(p+wq)/N$. Then,           {\small
%
\begin{align*}
{\Pr}\big[\pred_2(\vect,i)\big] = & \sum_{\vect'\neq\vect}{\Pr}\big[(c\xor t)[i]=(c'\xor t')[i]\mid\big]             \\
= & 
\underbrace{\sum_{\vect'\neq\vect}{\Pr}\big[(c\xor t)[i]=(c'\xor t')[i]\wedge\big(\free(\vect)\vee\free(\vect')\big)\big]}_{B_1}            \\
& + 
\underbrace{\sum_{\vect'\neq\vect}{\Pr}\big[(c\xor t)[i]=(c'\xor t')[i]\wedge\big(\neg\free(\vect)\wedge\neg\free(\vect')\big)\big]}_{B_2}.
\end{align*}
}%
%
For the probability ${\Pr}\big[(c\xor t)[i]=(c'\xor t')[i]\wedge\big(\free(\vect)\vee\free(\vect')\big)\big]$, we have to further distinguish two cases as follows.


\arrangespace

\noindent \textit{\underline{Case 1: $x\xor t\neq x'\xor t'$}}. Then {\bf using the previous idea, it can be show}
%
$$\Pr[a\xor t=a'\xor t'|x\xor t\neq x'\xor t']\leq\frac{1}{N-p-wq}.$$
%
Similarly, it further holds
%
$$\Pr[(c\xor t)[j]\neq(c'\xor t')[j']|a\xor t=a'\xor t']\leq\frac{1}{N-p-wq}.$$
%


\arrangespace

\noindent \textit{\underline{Case 2: $x\xor t=x'\xor t'$}}. Then it necessarily holds $a\xor t\neq a'\xor t'$, and thus
%
$$\Pr[(c\xor t)[j]\neq(c'\xor t')[j']|a\xor t=a'\xor t']\leq\frac{1}{N-p-wq}.$$
%



\arrangespace




On the other hand,
%
\begin{align*}
&\sum_{\vect'\neq\vect}{\Pr}\big[(c\xor t)[i]=(c'\xor t')[i]\wedge\big(\neg\free(\vect)\wedge\neg\free(\vect')\big)\big]       \\
\leq&\sum_{\vect'\neq\vect}{\Pr}\big[(c\xor t)[i]=(c'\xor t')[i]\mid\big(\neg\free(\vect)\wedge\neg\free(\vect')\big)\big] \times\Pr[\neg\free(\vect)]       \\
\leq&\sum_{\vect':(c\xor t)[i]=(c'\xor t')[i]}\frac{p}{N}  \leq  \frac{p}{N}.
\end{align*}
%


Summing over the above, we have
%
\begin{align*}
{\Pr}\big[\pred_2(\vect,i)\big]  \leq  \frac{2p+wq}{N}+\frac{q}{N-p-wq}  \leq  \frac{2p+wq+2q}{N} .
\end{align*}
%
Similarly, ${\Pr}\big[\pred_4(\vect,i)\big]\leq(2p+wq+2q)/N$. These establish
%
$$
{\Pr}\big[\bsix\mid\neg\bthree\wedge\neg\bfour\big] \leq
w^2q\cdot\Big(\frac{2(p+wq)}{N}\Big)^2\leq \frac{4w^2q(p+wq)^2}{N}.
$$


%
	%
	\begin{align*}
	\Pr[\beight] 
	=   &  \sum_{(x,a,b,y),(x',a',b',y')\in\mathcal{Q}_{C}'}\sum_{i,i',j,j'}\bigg(\underbrace{{\Pr}\big[(a\xor k_1)[i]\in U_2^{(1)}\big]}_{\leq(p+wq)/(N-p-wq)}     \\
	& \midindent\times
	\underbrace{{\Pr}\big[(a'\xor k_1)[i]\in U_2^{(1)}|(a\xor k_1)[i]\in U_2^{(1)}\big]}_{\leq1}\times\underbrace{\pcoll_{2}^+(2,a,a',i,i',j,j')}_{\leq1/(N-p-wq)}   \bigg)      \\
	\leq  &  {wq\choose 2}\cdot w^2\cdot\frac{p+wq}{N-p-wq}\cdot\frac{1}{N-p-wq}\leq
	\frac{w^4q^2(p+wq)}{2(N-p-wq)^2}.
	\end{align*}
	%
	
	Similarly by symmetry,
	%
	\begin{align*}
	\Pr[\bnine] 
	\leq
	\frac{w^4q^2(p+wq)}{2(N-p-wq)^2}.
	\end{align*}
	%
	%
	%The argument slightly resembles that of \bfour. In detail, \beight consists of two subevents as follows:
	%
	%\begin{itemize}[leftmargin=15mm]
	%	\item[(B-11.1)] there exist $(x,a,b,y),(x',a',b',y') \in \mathcal{Q}_{C}'$ and $i, i^{\prime},j\neq j^{\prime} \in\{1, \ldots, w\}$, $(a,b, j) \neq \left(a^{\prime}, b',j^{\prime}\right)$, such that $(a \oplus k_1)[i]\in U_{2}^{(1)}, (a' \oplus k_1)[i']\in U_{2}^{(1)}$, and $\big(T(\overline{S_2}(a\xor k_1))\xor k_2\big)[j]=\big(T(\overline{S_2}(a'\xor k_1))\xor k_2\big)[j']$.
	%	\item[(B-11.2)] there exist $(x,a,b,y),(x',a',b',y') \in \mathcal{Q}_{C}'$ and $i, i^{\prime},j \in\{1, \ldots, w\}$, $(a,b) \neq \left(a^{\prime}, b'\right)$, such that $(a \oplus k_1)[i]\in U_{2}^{(1)}, (a' \oplus k_1)[i']\in U_{2}^{(1)}$, and $\big(T(\overline{S_2}(a\xor k_1))\big)[j]=\big(T(\overline{S_2}(a'\xor k_1))\big)[j]$.
	%\end{itemize}
	
	
	%
	%For the case of $j\neq j'$, we have $\Pr\big[(T(\overline{S_2}(a\xor k_1))\xor k_2)[j]=(T(\overline{S_2}(a'\xor k_1))\xor k_2)[j']\big]=1/N$ by that $k_2[j]$ and $k_2[j']$ are uniform and independent. Therefore, $\Pr[\text{[(B-11.1)}]\leq{w\choose2}w^2q^2p/N^2\leq w^4q^2p/2N^2$.
	%
	%
	%
	%For (B-11.2), we have
	%define $\pcoll_{22}(a,a',i,i',j)$ as the conditional probability $\Pr\big[\big(T(\overline{S_2}(a\xor k_1))\big)[j]=\big(T(\overline{S_2}(a'\xor k_1))\big)[j]|   
	%(a' \oplus k_1)[i']\in U_2^{(1)}\wedge(a \oplus k_1)[i]\in U_2^{(1)}\big]$. Then we derive the probability as follows.       
	%
	%\begin{align*}
	%\Pr[\text{(B-11.2)}] 
	%=   &  \sum_{(x,a,b,y),(x',a',b',y')\in\mathcal{Q}_{C}'}\sum_{i,i',j}\bigg(\underbrace{{\Pr}\big[(a\xor k_1)[i]\in U_2^{(1)}\big]}_{\leq(p+wq)/(N-p-wq)}     \\
	%& \midindent\times
	%\underbrace{{\Pr}\big[(a'\xor k_1)[i]\in U_2^{(1)}|(a\xor k_1)[i]\in U_2^{(1)}\big]}_{\leq1}\times\underbrace{\pcoll_{2}^+(2,a,a',i,i',j)}_{\leq1/(N-p-wq)}   \bigg)      \\
	%\leq  &  {q\choose 2}\cdot w^3\cdot\frac{p+wq}{N-p-wq}\cdot\frac{1}{N-p-wq}\leq
	%\frac{w^3q^2(p+wq)}{2(N-p-wq)^2}.
	%%\underbrace{\sum_{(x,a,b,y),(x',a',b',y')\in\mathcal{Q}_{C}'}\sum_{i\neq i'}\sum_{j}\pcoll_{22}(a,a',i,i',j)}_{B_2}      \\ 
	%%
	%%
	%%\leq  &  \frac{p+wq}{N-p-wq}\cdot\bigg(
	%%\underbrace{\sum_{(x,a,b,y),(x',a',b',y')\in\mathcal{Q}_{C}'}\sum_{i\neq i'}\sum_{j}\pcoll_{22}(a,a',i,i',j)}_{B_2}      \\ 
	%%& \midindent\midindent\midindent\hugeindent   +  \underbrace{\sum_{(x,a,b,y),(x',a',b',y')\in\mathcal{Q}_{C}'}\sum_{i}\sum_{j}\pcoll_{22}(a,a',i,i,j)}_{B_3}\bigg)     .
	%\end{align*}
	%
	%
	
	%First, consider $\pcoll_{22}(a,a',i,i',j)$ with $i\neq i'$. Since $a\neq a'$, there exists $i_0$ such that $(a\xor k_1)[i_0]\neq(a'\xor k_1)[i_0]$. Then either $i\neq i_0$ or $i'\neq i_0$. Wlog assume $i\neq i_0$. Note that this means $(a'\xor k_1)[i]\neq(a\xor k_1)[i_0]$, as otherwise both $(a'\xor k_1)[i]$ and $(a\xor k_1)[i_0]$ fall in $U_2^{(0)}$ and it contradicts $\neg\btwo$. \textbf{The remaining discussion resembles that for (B-43) before (which consists of 3 cases), the uniformness of the value $S_2((a\xor k_1)[i_0])$ is sufficient to ensure that ${\Pr}[T(\overline{S_2}(a\xor k_1))[j]\xor T(\overline{S_2}(a'\xor k_1))[j]]\leq\frac{1}{N-p-wq}$}, which means $B_2\leq w^3q^2p/2N(N-p-wq)$.
	%
	%
	%
	%Then, consider $\pcoll_{22}(a,a',i,i,j)$. Assume that $S_2((a\xor k_1)[i])=u_2$ and $S_2((a'\xor k_1)[i])=u_2'$ for $(u_2,v_2),(u_2',v_2')\in\mathcal{Q}_{S_2}^{(0)}$. Then it holds      {\small
	%	%
	%	\begin{align}
	%	&  T(\overline{S_2}(a\xor k_1))[j]\xor T(\overline{S_2}(a'\xor k_1))[j]       \notag   \\
	%	= &
	%	(t_{j,i}\cdot v_2)
	%	\xor
	%	(t_{j,i}\cdot v_2')
	%	\xor
	%	\Big(\bigoplus_{1\leq\ell\leq w,\ell\neq i}t_{j,\ell}\cdot
	%	\big(S_2((a\xor k_1)[\ell])\xor S_2((a'\xor k_1)[\ell])\big)\Big)    .
	%	\label{eq:interm-eq-b11}
	%	\end{align}
	%}%
	%%
	%Now:
	%\begin{itemize}
	%	\item If $a[\ell]=a'[\ell]$ for any $\ell\neq i$, then Eq. (\ref{eq:interm-eq-b11}) collapses to $t_{j,i}\cdot v_2=t_{j,i}\cdot v_2'$ which is not possible since $t_{j,i}\neq 0$ \textbf{and $v_2\neq v_2'$};
	%	\item Else, there exists $i_0\neq i$ such that $(a\xor k_1)[i_0]\neq(a'\xor k_1)[i_0]$. This means $(a'\xor k_1)[i]\neq(a\xor k_1)[i_0]$ by $\neg\btwo$. \textbf{The remaining discussion resembles that for (B-43) before (which consists of 3 cases), the uniformness of the value $S_2((a\xor k_1)[i_0])$ is sufficient to ensure that Eq. (\ref{eq:interm-eq-b11}) holds with probability at most $1/(N-p-wq)$}.
	%\end{itemize}
	%%
	%Therefore, in this case, it still holds $\pcoll_{22}(x,x',i,i,j)\leq1/(N-p-wq)$, which means $A_3\leq{q\choose2}w^2p/N(N-p-wq)\leq w^2q^2p/2N(N-p-wq)$.
	
	
	%
	%Summing over the above, we reach
	%%
	%$$\Pr[\cfive]\leq\frac{4w^3q^2p}{N(N-p-wq)}.$$
	%%
	%
	%The analysis of \cseven is similar by symmetry, resulting in
	%%
	%%
	%$$\Pr[\cseven]\leq\frac{4w^3q^2p}{N(N-p-wq)}.$$
	%%
	
	
	\arrangespace
	
	
	Summing over the above yields
	%
	\begin{align*}
	&  \Pr\big[ \tau' \in \Theta_{\text {good }}(\tau)\big]  \leq \sum_{i=1}^{9}\Pr[\bi]       \\
	\leq  & \frac{w^2 q (p+w q)^{2}}{(N-p-wq)^{2}}+
	\frac{2w^2 q (p+w q)^{2}}{N  (N-p-wq)}+
	\frac{w^2qp^2}{N^2}+
	\frac{8w^3q^2p}{N(N-p-wq)}      \\
	\leq  & \frac{9w^2 q (p+w q)^{2}}{N^2}+ \frac{16w^3q^2p}{N^2} .
	\end{align*}
	%
	as claimed.        \qed
\end{proof}


%
%\textcolor{red}{Similar to the outermost two round, we will extend the inner two round (the two and the five round).  Pick a pair of S-box $(S_1, S_6)$ such that $S_{1} \vdash \mathcal{Q}_{S_{1}}^{(0)}$ and $S_{6} \vdash \mathcal{Q}_{S_{6}}^{(0)}$, and for each $ (t, x, y) \in \mathcal{Q}_{C}$ we set $a=S_{1}\left(x \oplus k_{0} \oplus t\right)$, $b=S_{6}^{-1}\left(y \oplus k_{6} \oplus t\right)$. In this way we obtain $\mathnormal{q}$ tuples of the form $(t, a, b)$; for convenience we denote the set of such induced tuples by $\mathcal{Q}_{C}^{*}\left(S_{1}, S_{6}\right)$. Then we choose a pair of keys $\left(k_{1}, k_{5}\right) \in \mathcal{K}^{2}$ uniformly at random. Once these keys have been chosen, some construction queries will become involved in collisions. A colliding query is defined as a construction query $(t, a, b) \in \mathcal{Q}_{C}^{*}\left(S_{1}, S_{6}\right)$. After that, we build a new set $\mathcal{Q}_{S_{outer}}^{\prime}$ of S-box evaluations that will play the role of an extension of $\mathcal{Q}_{S}$. Then we choose the key $k_2, k_3, k_4$ uniformly at random. An extended transcript of $\tau$ will be defined as a tuple $\tau''=\left(\mathcal{Q}_{C}, \mathcal{Q}_{S}, \mathcal{Q}_{S_{outer}}^{\prime}, \mathbf{k}\right)$ where $\mathbf{k}=\left(k_{1}, k_{2},k_{3},k_{4}, k_{5}\right)$. We also let...}

%$$
%\begin{aligned}
%&\mathcal{Q}_{S_{2}}^{(1)}=\left\{(u, v) \in\{0,1\}^{n} \times\{0,1\}^{n}:(2, u, v) \in \mathcal{Q}_{S} \cup \mathcal{Q}_{S_{outer}}^{\prime}\right\}\\
%&\mathcal{Q}_{S_{5}}^{(1)}=\left\{(u, v) \in\{0,1\}^{n} \times\{0,1\}^{n}:(5, u, v) \in \mathcal{Q}_{S} \cup \mathcal{Q}_{S_{outer}}^{\prime}\right\}\\
%\end{aligned}
%$$
%
%\noindent In words, $\mathcal{Q}_{S_{i}}^{(1)}$ summarizes each constraint that is forced on $S_{i}$ by $\mathcal{Q}_{S}$ and $\mathcal{Q}_{S_{outer}}^{\prime}$. Let
%
%$$
%\begin{aligned}
%&U_{2}=\left\{u_{2} \in\{0,1\}^{n}:\left(2, u_{2}, v_{2}\right) \in \mathcal{Q}_{S_{2}}^{(1)}\right\}, \quad V_{2}=\left\{v_{2} \in\{0,1\}^{n}:\left(2, u_{2}, v_{2}\right) \in \mathcal{Q}_{S_{2}}^{(1)}\right\},\\
%&U_{5}=\left\{u_{5} \in\{0,1\}^{n}:\left(5, u_{5}, v_{5}\right) \in \mathcal{Q}_{S_{5}}^{(1)}\right\}, \quad V_{5}=\left\{v_{5} \in\{0,1\}^{n}:\left(5, u_{5}, v_{5}\right) \in \mathcal{Q}_{S_{5}}^{(1)}\right\}\\
%\end{aligned}
%$$
%
%\noindent be the domains and ranges of $\mathcal{Q}_{S_{1}}^{(1)}$, $\mathcal{Q}_{S_{6}}^{(1)}$ respectively. After define the extended transcript $\tau''$ is bad, \\

%\begin{itemize}
%  \item[\feai]
%  there exists $(t, a, b) \in \mathcal{Q}_{C}^{*}\left(S_{1}, S_{6}\right), \left(u_{2}, v_{2}\right) \in \mathcal{Q}_{S_{2}}^{(1)}, \left(u_{5}, v_{5}\right) \in \mathcal{Q}_{S_{5}}^{(1)}$, and index $i, j \in \{1, \ldots, w\}$ such that $(T_1\left(a \oplus k_{1} \oplus t\right))[i]=u_2$ and $(T_5^{-1}\left(b\right) \oplus k_{5} \oplus t)[j]=v_5$.
%  \item[\feaii]
%  there exists $(t, a, b) \in \mathcal{Q}_{C}^{*}\left(S_{1}, S_{6}\right), \left(u_{2}, v_{2}\right) \in \mathcal{Q}_{S_{2}}^{(1)}, \left(u_{3}, v_{3}\right) \in \mathcal{Q}_{S_{3}}$, and index $i, j \in \{1, \ldots, w\}$ such that $(T_1\left(a \oplus k_{1} \oplus t\right))[i]=u_2$ and 
%  $$
%  \left(T_{2}\left(S_{2}\left(T_1(a \oplus k_{1} \oplus t)\right) \oplus k_{2} \oplus t\right)\right)[j]=u_3.
%  $$
%  \item[\feaiii]
%  there exists $(t, a, b) \in \mathcal{Q}_{C}^{*}\left(S_{1}, S_{6}\right), \left(u_{4}, v_{4}\right) \in \mathcal{Q}_{S_{4}}, \left(u_{5}, v_{5}\right) \in \mathcal{Q}_{S_{5}}^{(1)}$, and index $i, j \in \{1, \ldots, w\}$ such that $(T_5^{-1}\left(b\right) \oplus k_{5} \oplus t)[j]=v_5$ and 
%  $$
%  \left(T_{4}^{-1}\left(S_{5}^{-1}\left(T_5^{-1}(b) \oplus k_{5} \oplus t\right)\right) \oplus k_{4} \oplus t\right)[i]=v_4.
%  $$
%  \item[\feaiv]
%  there exists $(t, x, y) \in \mathcal{Q}_{C}$ and distinct indices $i, j \in \{1, \ldots, w\}$ such that 
%  $$(T_1\left(a \oplus k_{1} \oplus t\right))[i]=(T_1\left(a \oplus k_{1} \oplus t\right))[j], \text{ or }$$
%  $$(T_5^{-1}\left(b\right) \oplus k_{5} \oplus t)[i]=(T_5^{-1}\left(b\right) \oplus k_{5} \oplus t)[j].$$
%\end{itemize}

%\noindent \textbf{Lemma 11} \emph{One has}
%
%\begin{equation}
%\operatorname{Pr}[\tau'' \in \Theta_{bad}(\tau)] \leq \frac{w^2 q (p+w q) (3 p +w q)}{N^{2}} + \frac{w^{2} q}{N}.
%\end{equation}
%
%\noindent The proof is similar to Lemma 10.\\

%Let $\tau_{1} = \tau' \cup \tau''$. Then combine (11), (12), we can get
%
%\begin{equation}
%\operatorname{Pr}[\tau_{1} \in \Theta_{bad}(\tau)] \leq \frac{2w^2 q (p+w q) (3 p +w q)}{N^{2}} + \frac{2w^{2} q}{N}.
%\end{equation}




%
%\subsection{Analysis for Good Transcript}

\arrangespace

\noindent \textbf{Analyzing Good Extensions}.
%
Fix a good transcript and a good round-key vector $\mathnormal{k}$, we are to derive a lower bound for the probability  $\operatorname{Pr}\left[\mathcal{S} \stackrel{\mathbf{s}}{\leftarrow}(\mathcal{S}(n))^{6}: \mathrm{SP}_{k}[\mathcal{S}] \vdash \mathcal{Q}_{C} | \mathcal{S} \vdash \mathcal{Q}_{S}\right]$. We ``peel off'' the outer four rounds. Then assuming $(S_{1}, S_2, S_{5}, S_6)$ is good, we analyze the induced 2-round transcript to yield the final bounds.






\noindent \emph{Proof:} We upper bound the probabilities of the nine conditions in turn. We denote $\Theta_i$ the set of attainable transcripts satisfying condition \hi.\\

The proof of \hone to \hfive and \hseven , \height is similar to the Lemma 8, there are no more details. So we just consider \hsix and \hnine here. We first note that, if the condition is satisfied, we have $\left(S_{2}\left(a \oplus k_{1} \oplus t\right)\right)[i]$ remain uniform in $\{0, 1\}^{n} \verb|\| (\mathcal{Q}_{S_{1}} \cup \mathcal{Q}_{S_{2}} \cup \mathcal{Q}_{S_{5}} \cup \mathcal{Q}_{S_{6}})$.  Moreover if $u_3 =u_3'$, that is $c \oplus t=c' \oplus t'$, then after oplus different tweak, the input of the $S_4$ must be different, so the collision would not happen. Hence we must have $u_3 \neq u_3'$. The condition can be divided into two conditions: the first concerning with $j\neq j'$, while the second concerning with $j=j'$.

For the first case, to make 
$$
  \left(T_{3}\left(S_{3}\left(T_2\left(c \oplus k_{2} \oplus t\right)\right) \oplus k_{3} \oplus t\right)\right)[j] = \left(T_{3}\left(S_{3}\left(T_2\left(c' \oplus k_{2} \oplus t'\right)\right) \oplus k_{3} \oplus t'\right)\right)[j'].
$$
\noindent achieved, we just leverage the fact that $k_3[j]$ and $k_3[j']$ are uniform and independent, so the collision holds with probability $1/N$. Because of $a$remain uniform in $\{0, 1\}^{n} \verb|\| (\mathcal{Q}_{S_{1}} \cup \mathcal{Q}_{S_{2}} \cup \mathcal{Q}_{S_{5}} \cup\mathcal{Q}_{S_{6}})$, let $(a', b')$ be the unique query such that the collision happened. Then the probability that $\left(T_2\left(c \oplus k_{2} \oplus t\right)\right)[i] = u_3, \left(T_2\left(c' \oplus k_{2} \oplus t'\right)\right)[i] = u_3'$ is at most $\frac{1}{N-p}$, because we have at most $w^2 q^2(p+ w q)$ such tuples, one has

$$
\operatorname{Pr}\left[\tau_{inner} \in \Theta_{6}\right] \leq \frac{w^{2} q^{2} (p+w q)}{N \cdot (N-p)}.
$$

For the case of $j=j'$ with distinct $(c,d),(c',d')$, that is there is only one index has different value of input and output. Because of the value $S_{2}\left(T_1\left(a' \oplus k_{1} \oplus t\right)\right)[i]$ also remain uniform in $\{0, 1\}^{n} \verb|\| (\mathcal{Q}_{S_{1}} \cup \mathcal{Q}_{S_{2}} \cup \mathcal{Q}_{S_{5}} \cup \mathcal{Q}_{S_{6}})$, then we leverage the randomness due to lazy sampling $S_3(T_2(c\xor k_2\xor t))$. Conditioned on \feaiv, for $i''\neq i$, the value $T_2(c \oplus k_2 \oplus t)[i'']$ ``does not collide with'' pairs in $\mathcal{Q}_{S_{3}}^{(1)}$, and will be assigned a random outputs during the lazy sampling process. Simultaneously conditioned on $\feav$, for distinct $i'' \neq i$, if $(T_2\left(c \oplus k_{2} \oplus t\right))[i] = u_3$, it holds $(T_2(c\xor k_2 \xor t))[i'']\neq (T_2(c'\xor k_2 \xor t'))[i'']$. Since $T_3$ contain no zero entries, so the value $ \left(T_{3}\left(S_{3}\left(T_2\left(c \oplus k_{2} \oplus t\right)\right) \oplus k_{3} \oplus t\right)\right)[i'']$ could not be disturbed by the value of $ \left(T_{3}\left(S_{3}\left(T_2\left(c' \oplus k_{2} \oplus t'\right)\right) \oplus k_{3} \oplus t'\right)\right)[i'']$ and thus uniform in at least $\frac{1}{N - p- wq}$. One has,
$$
\operatorname{Pr}\left[\tau_{inner} \in \Theta_{6}\right] \leq \frac{w^{2} q^{2} (p+w q)}{(N- p- wq) \cdot (N-p)}.
$$
\noindent So, combine these two subevents, one has
$$
\operatorname{Pr}\left[\tau_{inner} \in \Theta_{6}\right] \leq \frac{w^{2} q^{2} (p+w q)}{N \cdot (N-p)} + \frac{w^{2} q^{2} (p+w q)}{(N- p- wq) \cdot (N- p)}.
$$

\noindent Similarly, we have

$$
\operatorname{Pr}\left[\tau_{inner} \in \Theta_{9}\right] \leq \frac{w^{2} q^{2} (p+w q)}{N \cdot (N-p)} + \frac{w^{2} q^{2} (p+w q)}{(N- p- wq) \cdot (N- p)}.
$$

\noindent Then combining Lemma 9, we complete the proof of Theorem 2.







\arrangespace

\noindent \textbf{Analyzing Good Extensions}.
%
This part just follows the same line as Sect. \ref{sec:good-tau-4-rounds}. In detail, define $\calC_{\bfk}^T[\calS](c):=   \overline{S_4}(T(\overline{S_3}(c\xor k_2))\xor k_3)\xor T^{-1}(k_4)$. Then, for any attainable transcript $\tau$, we have the following upper bound for the the ideal world probability:
%
%
\begin{align*}
\mathsf{p}_{1}(\tau)=&\operatorname{Pr}\left[(\widetilde{P},\mathcal{S})\stackrel{\$}{\leftarrow} \widetilde{{\mathsf{Perm}}}(\mathcal{T}, w n)\times\mathsf{Perm}(n)^6: (\mathcal{S} \vdash \mathcal{Q}_{S}) \wedge(\widetilde{P} \vdash \mathcal{Q}_{C})  \right]		\\
\leq&\frac{1}{(N^w)_q}\cdot\bigg(\frac{1}{(N)_p}\bigg)^6.
\end{align*}



To reach the real world probability $\mathsf{p}_2(\tau)$, for any transcript extension $\tau'=(\mathcal{Q}_{C}',\mathcal{Q}_{S},\mathcal{Q}_{S}',S_1^*,S_2^*,S_5^*,S_6^*,\bfk)$ from $\tau$, denote            {\small
	%
	%
	\begin{align}
	\mathsf{p}_{\mathrm{re}}(\tau') = & \operatorname{Pr}\Big[\left(\mathbf{k}',\mathcal{S}\right) \stackrel{\$}{\leftarrow} \big(\{0,1\}^{wn}\big)^7 \times \mathsf{Perm}(n)^6:
	\Big(\big(S_1=S_1^*\big)\wedge\big(S_2=S_2^*\big)\wedge\big(S_5=S_5^*\big)		\notag 	\\
	&\codeindent\codeindent\codeindent\codeindent\wedge\big(S_6=S_6^*\big)\wedge\big(S_3\vdash\mathcal{Q}_{S_3}^{(1)}\big)\wedge\big(S_4\vdash\mathcal{Q}_{S_4}^{(1)}\big)\wedge\big(\calC_{\bfk'}^T[\calS] \vdash \mathcal{Q}_C'\big)\wedge\big(\bfk'=\bfk\big)\Big)\Big]	 	\notag 	\\
	\mathsf{p}_{\mathrm{mid}}(\tau') = & \operatorname{Pr}\Big[\mathcal{S} \stackrel{\$}{\leftarrow}\mathsf{Perm}(n)^6:(\calC_{\bfk}^T[\calS] \vdash \mathcal{Q}_C')~\Big|~
	(S_1=S_1^*)\wedge (S_2=S_2^*)\wedge\big(S_5=S_5^*\big)	 	\notag 	\\
	&\codeindent\codeindent\codeindent\codeindent \wedge\big(S_6=S_6^*\big)\wedge(S_3\vdash\mathcal{Q}_{S_3}^{(1)})\wedge (S_4\vdash\mathcal{Q}_{S_4}^{(1)})\Big].	 	\notag 	
	%\label{eq:defn-p-mid}
	\end{align}
}%
%
%
and let $\beta_1=|\mathcal{Q}_{S_3}^{(1)}|-p$ and $\beta_2=|\mathcal{Q}_{S_4}^{(1)}|-p$. With these, we have
%
%
\begin{align*}
\mathsf{p}_2(\tau)=&\operatorname{Pr}\left[\left(\mathbf{k},\mathcal{S}\right) \stackrel{\$}{\leftarrow} \big(\{0,1\}^{wn}\big)^7 \times \mathsf{Perm}(n)^6:\big(\spn_{\bfk}^{T}[\mathcal{S}] \vdash \mathcal{Q}_{C}\big) \wedge \big(\mathcal{S} \vdash \mathcal{Q}_{S}\big)\right]		\\
\geq & \sum_{\tau^{\prime} \in \Theta_{\mathrm{good}}(\tau)} \mathsf{p}_{\mathrm{re}}(\tau')  
\geq
\sum_{\tau^{\prime} \in \Theta_{\mathrm{good}}(\tau)}
%
\frac{1}{N^{7w}\big((N)_{N}\big)^4(N)_{p+\beta_1}(N)_{p+\beta_2}}\cdot \mathsf{p}_{\mathrm{mid}}(\tau')  .
\end{align*}
%
%
Therefore,
%
%
\begin{align*}
\frac{\mathsf{p}_{2}(\tau)}{\mathsf{p}_{1}(\tau)}   \geq  &
\sum_{\tau^{\prime} \in \Theta_{\mathrm{good}}(\tau)}
\frac{(N^w)_q\cdot\big((N)_p\big)^6}{N^{7w}\big((N)_{N}\big)^4(N)_{p+\beta_1}(N)_{p+\beta_2}}\cdot \mathsf{p}_{\mathrm{mid}}(\tau')         \\
\geq  &    \min_{\tau' \in \Theta_{\mathrm{good}}(\tau)}\big((N^w)_q\cdot\mathsf{p}_{\mathrm{mid}}(\tau')\big)
\sum_{\tau^{\prime} \in \Theta_{\mathrm{good}}(\tau)}
	\frac{1}{N^{7w}\big((N-p)_{N-p}\big)^4(N-p)_{\beta_1}(N-p)_{\beta_2}}    \\
\geq & \Pr\big[ \tau' \in \Theta_{\text {good }}(\tau)\big]\cdot
\min_{\tau' \in \Theta_{\mathrm{good}}(\tau)}\big((N^w)_q\cdot\mathsf{p}_{\mathrm{mid}}(\tau')\big)        \\
\geq & \Pr\big[ \tau' \in \Theta_{\text {good }}(\tau)\big]\cdot
\bigg(1-\frac{q^2}{N^w}-\frac{q(2wp+6w^2q)^2}{N^2}\bigg)\hugeindent\codeindent(\text{by Lemma }\ref{lemma:bound-middle-two-rounds})        \\
\geq &    \bigg(1-\frac{3w^{2} q \left(p+w q\right)^{2}}{N^{2}} - \frac{w^{2} q}{N} - \frac{9w^2 q (p+w q)^{2}}{N^2}- \frac{16w^3q^2p}{N^2}\bigg)\cdot\bigg(1-\frac{q^2}{N^w}-\frac{q(2wp+6w^2q)^2}{N^2}\bigg)     \notag      \\
\geq  &  1-\frac{3w^{2} q \left(p+w q\right)^{2}}{N^{2}} - \frac{w^{2} q}{N} - \frac{9w^2 q (p+w q)^{2}}{N^2}- \frac{16w^3q^2p}{N^2}-\frac{q^2}{N^w}-\frac{q(2wp+6w^2q)^2}{N^2}     \notag   
\end{align*}
%
as claimed in Eq. (\ref{eq:bound-proximity-6-round}).
