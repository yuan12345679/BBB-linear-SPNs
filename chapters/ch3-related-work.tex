

\section{related works}
\label{section:related works}

Till now, for the beyond birthday security, the wide tweakable block ciphers based on 2-round nonlinear SPN was proved in ~\cite{cogliati2018wide}, we will briefly recall it in this chapter.\\

\textsc{2-ROUND NONLINEAR SPN.} Fix a transcript $\tau = (\mathcal{Q}_C,\mathcal{Q}_S)$, a key $k \in \mathcal{K}^{r+1}$, a tweakable permutation $\widetilde{\mathcal{P}} \in \widetilde{\operatorname{Perm}}(\mathcal{T}, w n)$,  a set of S-boxes $\mathcal{S}=(S_1, \cdots  ,S_r) \in \operatorname{Perm}(n)^r $ and $j \in \{1, \cdots, \ell \}$. Let $k_1, \cdots ,k_{\ell} \in \mathcal{K}^{r+1}$ and $\mathcal{P} = (\operatorname{P}_1, \cdots ,\operatorname{P}_\ell) \in \widetilde{\operatorname{Perm}}(\mathcal{T},wn)^l$, if $\operatorname{SP}_{k_j}^{T}[\mathcal{S}] \vdash \mathcal{Q}_{C_j}$ (resp. $\mathcal{P}_j \vdash \mathcal{Q}_{C_j}$) for every $j = 1, \cdots ,\ell$, then we will write $(\operatorname{SP} ^{T}_{k_j}[\mathcal{S}])_{j = 1, \cdots ,\ell} \vdash \mathcal{Q}_C$ (resp. $\mathcal{P} \vdash \mathcal{Q}_{C_j}$).\\

If there exist $\widetilde{\mathcal{P}} \in \widetilde{\operatorname{Perm}}(\mathcal{T},w n)$ and $\mathcal{S} \in \operatorname{Perm}(\emph{n})^w$ that outputs $\tau$ at the end of the interaction with $\mathcal{D}$, then for an attainable transcript $\tau = (\mathcal{Q}_C,\mathcal{Q}_S)$, let

$$
\begin{aligned}
&\mathrm{p}_{1}\left(\mathcal{Q}_{C} | \mathcal{Q}_{S}\right)=\operatorname{Pr}\left[\widetilde{\mathcal{P}} \stackrel{s}{\leftarrow} \widetilde{{\operatorname{Perm}}}(\mathcal{T}, w n)^{\ell}, \mathcal{S} \stackrel{s}{\leftarrow} \operatorname{Perm}(n)^{r}: \widetilde{\mathcal{P}} \vdash \mathcal{Q}_{C} | \mathcal{S} \vdash \mathcal{Q}_{S}\right],\\
&\mathrm{p}_{2}\left(\mathcal{Q}_{C} | \mathcal{Q}_{S}\right)=\operatorname{Pr}\left[k_{1}, \ldots, k_{\ell} \stackrel{s}{\leftarrow} \mathcal{K}^{r+1}, \mathcal{S} \stackrel{s}{\leftarrow} \operatorname{Perm}(n)^{r}:\left(\mathrm{SP}_{k_{j}}^{T}[\mathcal{S}]\right)_{j} \vdash \mathcal{Q}_{C} | \mathcal{S} \vdash \mathcal{Q}_{S}\right].
\end{aligned}
$$
With these definitions, we can get the following lemma, the core of the H-coefficients technique (without defining ``bad'' transcripts).\\

\noindent \textbf{Lemma 1} \emph{Let $\epsilon \geq 0$, Suppose that for any attainable transcript $\tau = (\mathcal{Q}_C,\mathcal{Q}_S)$,}

$$
\mathrm{p}_{2}\left(\mathcal{Q}_{C} | \mathcal{Q}_{S}\right) \geq (1 - \epsilon) \mathrm{p}_{1}\left(\mathcal{Q}_{C} | \mathcal{Q}_{S}\right),
$$
\emph{Then one has}
$$
\operatorname{Adv}^{mu}_{\mathrm{SP}^T}(\mathcal{D}) \leq \epsilon.
$$

The lower bound is called \emph{$\epsilon$ -point-wise proximity} of the transcript $\tau = (\mathcal{Q}_C, \mathcal{Q}_S)$. The point-wise proximity of a transcript in the multi-user setting is guaranteed by the point-wise proximity of $(\mathcal{Q}_{C_{j}}, \mathcal{Q}_S)$ for each $j = 1, \cdots ,\ell$ in the single user setting. The following lemma is a restatement of Lemma 3  in ~\cite{hoang2016key}.\\

\noindent \textbf{Lemma 2} \emph{Let $\epsilon : \mathbb{N} \times \mathbb{N} \rightarrow \mathbb{R}^{\geq 0}$ be a function such that}

\emph{
\begin{itemize}
 \item[1.]
 $\epsilon (x, y) + \epsilon (x, z) \leq \epsilon (x, y + z)$ for every $x, y, z \in \mathbb{N}$,
 \item[2.]
 $\epsilon (\cdot, z)$ and $\epsilon (z, \cdot)$ are non-decreasing functions on $\mathbb{N}$ for every $z \in \mathbb{N}$.
\end{itemize}
}

\emph{Suppose that for any distinguisher $\mathcal{D}$ in the single-user setting that makes p primitive queries to each of the underlying S-boxes and makes q construction queries, and for any attainable transcript $\tau = (\mathcal{Q}_C, \mathcal{Q}_S)$ obtained by $\mathcal{D}$, one has}

$$
\mathrm{p}_{2}\left(\mathcal{Q}_{C} | \mathcal{Q}_{S}\right) \geq (1 - \epsilon(p,q)) \mathrm{p}_{1}\left(\mathcal{Q}_{C} | \mathcal{Q}_{S}\right).
$$
\emph{Then for any distinguisher $\mathcal{D}$ in the multi-user setting that makes $\mathnormal{p}$ primitive queries to each of the underlying S-boxes and makes total $\mathnormal{q}$ construction queries, and for any attainable transcript $\tau = (\mathcal{Q}_C, \mathcal{Q}_S)$ obtained by $\mathcal{D}$, one has}
$$
\mathrm{p}_{2}\left(\mathcal{Q}_{C} | \mathcal{Q}_{S}\right) \geq (1 - \epsilon(p + wq,q)) \mathrm{p}_{1}\left(\mathcal{Q}_{C} | \mathcal{Q}_{S}\right).
$$

\noindent So\\

\noindent \textbf{Lemma 3} \emph{Let $\delta, \delta^{\prime} \geq 0$ and let $\mathit{n}$ and $\mathit{w}$ be positive integers such that $\mathit{w} \geq 2$. Let $\mathnormal{T}$ be a $(\delta, \delta^{\prime})$ -super blockwise universal tweakable permutation. Then for any integers $\mathit{p}$ and $\mathit{q}$ such that  $wp+3w^2p \leq \frac{2^n}{2}$, one has}
$$
\begin{aligned}
\operatorname{Adv}_{\mathrm{SP}^{T}}^{su}(p, q) & \leq w^{2} q\left(\delta^{\prime} p+\delta w q\right)\left(3 \delta^{\prime} p+3 \delta w q+2 \delta^{\prime} w q\right)+\frac{q^{2}}{2 w n}+\frac{q\left(2 w p+6 w^{2} q\right)^{2}}{2^{2 n}}. \\
\operatorname{Adv}_{\mathrm{SP}^{T}}^{mu}(p, q) & \leq w^{2} q\left(\delta^{\prime} p+\left(\delta+\delta^{\prime}\right) w q\right)\left(3 \delta^{\prime} p+3 \delta w q+5 \delta^{\prime} w q\right) \\
&+\frac{q^{2}}{2^{w n}}+\frac{q\left(2 w p+8 w^{2} q\right)^{2}}{2^{2 n}}.
\end{aligned}
$$

\noindent Remark 2. For the sake of simplicity, we assume that the three keyed layers are actually the same, which is why we require $\mathnormal{T}$ to be $(\delta, \delta^{\prime})$ -super blockwise universal. However, if one looks closely at the proof, only the middle layer has to be super-blockwise-universal. The first and the last layer only need to be $(\delta, \delta^{\prime})$ -super blockwise universal.\\

\noindent Remark 3. When the S-boxes are modeled as block ciphers using secret keys, the security bound (in the standard model) is obtained by setting $\mathit{p} = 0$\\

For any  extended transcript $\tau^{\prime} = (\mathcal{Q}_C, \mathcal{Q}_S, \mathcal{Q}_S^{\prime},k)$, where $\mathcal{Q}_{S}^{(1)} = \mathcal{Q}_S \cup \mathcal{Q}_S^{\prime}$,  denote

$$
\mathrm{p}\left(\tau^{\prime}\right)=\operatorname{Pr}\left[\mathcal{S} \stackrel{\mathrm{s}}{\leftarrow} \operatorname{Perm}(n)^{2}: \mathrm{\operatorname{SP}}^{T}_{k}[\mathcal{S}] \vdash \mathcal{Q}_{C} |\left(S_{1} \vdash \mathcal{Q}_{S_{1}}^{(1)}\right) \wedge\left(S_{2} \vdash \mathcal{Q}_{S_{2}}^{(1)}\right)\right].
$$

Then we will get the following lemma:\\

\noindent \textbf{Lemma 4} \emph{For any good extended transcript $\tau^{\prime}$, one has}

$$
\left(2^{w n}\right)_{q} \mathrm{p}\left(\tau^{\prime}\right) \geq 1-\frac{q^{2}}{2^{w n}}-\frac{q\left(2 w p+6 w^{2} q\right)^{2}}{2^{2 n}}.
$$

if $\mathnormal{T}$ is a super blockwise tweakable universal permutation, then the security of $\operatorname{SP}^{\mathnormal{T}}$ converges to $2^n$ (in terms of the threshold number of queries) as the number of rounds $\mathit{r}$ increases. For detailed certification was show in ~\cite{cogliati2018wide}.\\

\noindent \textbf{Lemma 5} \emph{For an even integer $\mathit{r}$, let $SP^{\mathnormal{T}}$ be an $\mathit{r}$ -round substitution-permutation network based on a $(\delta, \delta^{\prime})$ -super blockwise tweakable universal permutation $\mathnormal{T}$, Then one has}

$$
\operatorname{Adv}_{\mathrm{SP}^{T}}^{\operatorname{mu}}(p, q) \leq 4 \sqrt{q}\left(2 w p \delta^{\prime}+2 w^{2} q\left(\delta^{\prime}+\delta\right)+w^{2} \delta\right)^{\frac{r}{4}}.
$$
Hence, assuming $\delta, \delta^{\prime} \backsimeq 2^{-n}$, and $p = q$, an $\mathit{r}$ -round $\operatorname{SP}^{\mathnormal{T}}$ is secure up to $2^{\frac{r n}{r+2}}$ queries.

