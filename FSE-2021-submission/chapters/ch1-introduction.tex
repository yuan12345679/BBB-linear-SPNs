
\section{Introduction}
\label{section:Introduction}


%\noindent\textbf{Background.} Practical iterative blockcipher (BC) designs roughly fall into
%two classes (with some rare exceptions such as IDEA), namely
%{\it Feistel ciphers and their generalizations}, and {\it
%	substitution-permutation networks} (SPNs). In a Feistel cipher,
%in the $i$-th round, the intermediate state $x=x_L\|x_R$ is
%updated according to $x_L\|x_R\mapsto x_R\|x_L\oplus
%G_i(k_i,x_R)$, where $G_i$ is called the $i$-th round function.
%On the other hand, their counterpart SPNs could be further
%abstracted as the {\it iterated Even-Mansour} (\iem) {\it
%	ciphers}, or {\it key-alternating ciphers}, which consist of
%alternatively applying round-key additions and keyless round
%permutations, i.e.
%$\iem_{k_0,k_1,\ldots,k_t}^{P_1,\ldots,P_t}(M)=k_t\xor
%P_t(\ldots(k_1\xor P_1(k_0\xor M)))$.
%
%
%The traditional security notion for BCs is {\it
%	pseudorandomness}: for any adversary with reasonable resources
%(e.g. polynomial complexity), the BC with {\it a random and
%	secret key} should be indistinguishable from a truly random
%permutation. Proving such security for concrete BCs such as AES
%seems out of the reach of current techniques. Yet, by
%idealizing the underlying round functions, security could be
%proved. Following this line, both idealized
%Feistel~\cite{LRSTOC86,KAFFSE2014} and
%\iem~\cite{EMjoc97,KAprpEC12} have been proposed and analyzed.
%
%
%To obtain a $2n$-bit BC, the \iem model requires $2n$-bit
%permutations. Whereas following the Feistel approach, several
%$n$-to-$n$-bit functions suffice. Moreover, these functions
%need {\it not} to be {\it invertible} (this might be the reason
%why Feistel ciphers were extremely popular before 1990s). In
%all, Feistel ciphers could be built upon primitives with
%smaller domain and less structural properties, which is
%particularly appealing from a theoretical point of view. From
%the security point of view, without any additional hardness
%assumption other than the idealness of round functions,
%provable security is limited by the domain-size of the round
%functions~\cite{patarin04}. Therefore, \textsf{IEM} benefits
%from the use of larger primitives: with $t$ independent
%$2n$-bit random permutations and $2tn$ key bits, $t$-round \iem
%is provably secure up to $2^{\frac{2tn}{t+1}}$ adversarial
%queries~\cite{KAtightboundEC2014} which approaches $2^{2n}$ for
%large $t$. In contrast, Feistel models can only be secure
%against at most $2^n$ queries~\cite{patarin04}, which is far
%less than its domain-size $2^{2n}$. This upper bound is very
%unsatisfying. This issue is even more severe for models built upon small-domain primitives, such as multi-line generalized Feistel~\cite{GFSCRYPTO10} and SPNs~\cite{MV12SPN,SPNEprint:2017:016}.
%Despite this
%limitation as well as the gap between the idealized model and
%the rather weak round functions in practice, this provable
%approach supplies insights into the BC structures, excludes
%generic attacks, and may help refine designs. Due to these,
%this approach is valuable and has received a lot of attention.




\noindent{\bf Substitution-Permutation Networks.}
Most modern blockciphers are

built via two different generic structures: Feistel networks or substitution-permutation
networks (SPNs). These two approaches revolve around the extension of a ``complex'' function or permutation on a small domain to a keyed pseudorandom
permutation on a larger domain by iterating several times simple rounds.
SPNs start with a set of public permutations on the set of n-bit strings which
are called S-boxes. These public permutations are then extended to a keyed
permutation on wn-bit inputs for some integer w by iterating the following steps:
\begin{enumerate}
	\item[1.] break down the state in w n-bit blocks;
	\item[2.] compute an S-box on each block of the state;
	\item[3.] apply a keyed permutation layer to the whole wn-bit state (which is also applied to the plaintext before the first round).
\end{enumerate}


Many well-known block ciphers including AES, Serpent and PRESENT follow
this approach.


Proving the security of a particular concrete block cipher is
currently beyond our techniques. Thus, the usual approach is to prove that the
high-level structure is sound in a relevant security model. As for Feistel networks,
a substantial line of work starting with Luby and Rackoff's seminal work [LR88]
and culminating with Patarin's results [Pat03, Pat04] proves optimal security with
a sufficient number of rounds. Numerous other articles [Pat10, HR10, HKT11,
Tes14, CHK+16] study the security of (variants of) Feistel networks in various
security models. On the other hand, SPNs have comparatively seen very little
interest which seems rather surprising.






\subsection{Linear vs Non-linear}


The investigation of the theoretical soundness of this design strategy was initiated in three
recent papers. First, Cogliati and Seurin [CS15], and independently Farshim and Procter [FP15],
analyzed the simple case of an n-bit key k and an n-bit tweak t simply xored together at each
round, i.e., $f_i(k,t) = k_t$ for each $i = 0, . . . , r$. They gave attacks up to two rounds, and proved
birthday-bound security for three rounds. In fact, the security of this construction caps at $2^{n/2}$
queries independently of the number of rounds. Indeed, it can be written $\widehat{E}(k, t, x) = E(k\xor t, x)$,
where $E$ is the conventional iterated Even-Mansour cipher with the trivial key-schedule (i.e.,
the same round key is xored between each round), and by a result of Bellare and Kohno [BK03,
Corollary 5.7], a tweakable block cipher of this form can never offer more than $\kappa/2$ bits of
security, where $\kappa$ is the key-length of E (i.e., $\kappa = n$ in the case at hand). Hence, if we want
beyond-birthday-bound security, we have no choice but to consider more complex functions $f_i$
(at the bare minimum, these functions, even if linear, should prevent the TBC construction
from being of the form $E(k\xor t, x)$ for some block cipher E with n-bit keys).




This was undertaken by Cogliati, Lampe, and Seurin [CLS15], who considered nonlinear
ways of mixing the key and the tweak. More specifically, they studied the case where $f_i(\bfk, t) =
H_{k_i}(t)$, where the family of functions $(H_k)$ is uniform and almost XOR-universal, and the
master key is $\bfk = (k_0, . . . , k_r)$. Cogliati et al. showed that one round is secure up to the birthday bound,
and that two rounds are secure up to roughly $2^{2n/3}$ adversarial queries. They also provided a
(non-tight) asymptotic security bound improving as the number of rounds grows. However,
implementing a xor-universal hash function might be costly, and linear functions $f_i$'s would be highly preferable for obvious efficiency reasons.




\subsection{Our Results}

%In this paper, we ask whether it is possible to come with a tweakable Even-
%Mansour construction achieving both:
%1. a linear mixing of the tweak and the key to the state;
%2. beyond-birthday-bound security.
%We answer positively, by providing a construction with 2n-bit keys and n-bit tweaks.

In this paper, we ask whether it is possible to achieve security beyond the birthday barrier with linear SPN structures. In detail, we focus on linear SPNs with independent S-boxes and independent round keys, and we will focus on the case where $w\geq2$, since, when $w = 1$, we recover the standard Even-Mansour construction that has already been the focus
of a long line of work (as briefly reviewed later). For such linear SPNs, we prove the first beyond-birthday-bound (BBB) result on 4 rounds.





We first characterize the security of linear SPNs, where the
permutation layer is a linear function (over GF(2n), where n is the size of the
S-box) of the current wn-bit round key and the current wn-bit state. Indeed,
most current SPN-based block ciphers (e.g., AES, Serpent, PRESENT, etc.)
use linear permutation step, which involves a simple key-mixing step followed
by an invertible linear transformation. For this widely used setting we give a
general against any 2-round linear SPNs with w ≥ 2.1 Complementing this
attack, we show that a 3-round linear SPNs are secure, for any w, if the keyed
linear permutations satisfy some very mild technical requirements. This result
critically uses the H-coefficients technique [Pat08,CS14].


In an effort to reduce the number of rounds (and get other
benefits we explain below), we then turn our attention to non-linear SPNs, where
the permutation step does not have to be linear (although must remain efficient
and “non-cryptographic”). Here we show that even a 1-round SPN can be secure,
if appropriate keyed permutations are used.We identify a combinatorial property
on the permutations — which we term blockwise universality — that suffices for
security in this case, and then study the efficiency of constructing permutations
satisfying this property. Specifically, we show a construction of a satisfactory
permutation with n-bit keys (but having high degree), and another construction
with longer keys but having degree 3.


We then show that, by using such blockwise independent permutations, the
security of resulting SPNs increases when we increase the number of rounds:
while 1 round already achieves “birthday security”, as our main technical result
we show that 2-round non-linear SPNs (with independent S-boxes and keys in
different rounds) achieves “beyond-birthday” security (for up to 22n/3 queries).
This result uses the refinement of the H-coefficient technique due to [HT16].






Implications for small block size. While our results are directly meaningful
when the length n of public S-boxes in at least security parameter (e.g., for
building wide tweakable block ciphers), our bounds are too weak for regular SPNbased
ciphers, such as AES, which use very low values of n for their S-boxes.
This “2n provable barrier” is inherent using our current modeling, where the
S-box of size 2n is providing the only source of cryptographic hardness. More
generally, establishing a sound theory of building block ciphers from small S-boxes is one of the biggest and most important open problems in symmetric-key
cryptography.We hope that our structural results for reduced-round SPN ciphers
will be useful in establishing such theory, despite not crossing the fundamental
``$2^n$ barrier'' mentioned above.




%
%To tweak such linear SPNs, we consider the simplest approach, i.e., directly xoring a $wn$-bit tweak with each round key, and prove BBB result on 6 rounds. We will elaborate in detail as follows.
%
%\medskip\noindent{\bf BBB Security for 4-round linear SPNs.}
%




\subsection{Related Work}


There are only a few prior papers looking at provable security of SPNs. The vast
majority of such work analyzes the case of secret, key-dependent S-boxes (rather
than public S-boxes as we consider here), and so we survey that work first.



{\bf SPNs with secret S-boxes}. Naor and Reingold [NR99] prove security for
what can be viewed as a non-linear, 1-round SPN. Their ideas were further
developed, in the context of domain extension for block ciphers (see further
discussion below), by Chakraborty and Sarkar [CS06] and Halevi [Hal07].


Iwata and Kurosawa [IK00] analyze SPNs in which the linear permutation
step is based on the specific permutations used in the block cipher Serpent. They
show an attack against 2-round SPNs of this form, and prove security for 3-round
SPNs against non-adaptive adversaries. In addition to the fact that we consider
public S-boxes, our linear SPN model considers generic linear permutations and
we prove security against adaptive attackers.

Miles and Viola [MV15] study SPNs from a complexity-theoretic viewpoint.
Two of their results are relevant here. First, they analyze the security of linear
SPNs using S-boxes that are not necessarily injective (so the resulting keyed
functions are not, in general, invertible). They show that r-round SPNs of this
type (for r ≥ 2) are secure against chosen-plaintext attacks. (In contrast, our
results show that 2-round, linear SPNs are not secure against a combination of
chosen-plaintext and chosen-ciphertext attacks when w ≥ 2.) They also analyze
SPNs based on a concrete set of S-boxes, but in this case they only show security
against linear/differential attacks (a form of chosen-plaintext attack), rather
than all possible attacks, and only when the number of rounds is r = Θ(log n).


{\bf SPNs with public S-boxes}. A difference between our work and all the work
discussed above is that we treat the S-boxes as public. We are aware of only
one prior work analyzing the provable security of SPNs in this setting. Dodis
et al. [DSSL16] recently studied the indifferentiability [MRH04] of confusiondiffusion
networks, which can be viewed as unkeyed SPNs. One could translate
their results to the keyed setting, but that would require using multiple, keydependent
S-boxes (rather than a fixed, public S-box) and so would not imply
our results. We remark further that they show positive results only for 5 rounds
and above.


As observed earlier, the Even-Mansour construction [EM97] of a (keyed)
pseudorandom permutation from a public random permutation can be viewed
as a 1-round, linear SPN in the degenerate case where w = 1 (i.e., no domain
extension) and all round permutations are instantiated using simple key mixing.
Security of the 1-round Even-Mansour construction against adaptive chosenplaintext/
ciphertext attacks, using independent keys for the initial and final key mixing, was shown in the original paper [EM97]. Our positive results imply
security of the 1-round Even-Mansour construction (with similar concrete security
bounds) as a special case. The r-round generalization of the Even-Mansour
cipher has seen a lot of interest over the years, culminating with [CS14,HT16]
where it was proved that the r-round Even-Mansour construction is secure up to
roughly 2rn/(r+1) adversarial queries, when the public S-boxes are uniformly random
and independent permutations and the round keys are independent. Chen et
al. [CLL+14] also proved that several minimized variants of the 2-round Even-
Mansour construction are also secure up to roughly 22n/3 adversarial queries.
None of these results extend to the setting w > 1 considered in this work.





{\bf Cryptanalysis of SPNs}. Researchers have also explored cryptanalytic attacks
on generic SPNs [BS10,BBK14,DDKL,BK]. These works generally consider a
model of SPNs in which round permutations are secret, random (invertible) linear
transformations, and S-boxes may be secret as well; this makes the attacks
stronger but positive results weaker. In many cases the complexities of the
attacks are exponential in n (though still faster than a brute-force search for
the key), and hence do not rule out asymptotic security results. On the positive
side, Biryukov et al. [BBK14] show that 2-round SPNs (of the stronger form just
mentioned) are secure against some specific types of attacks, but other attacks
on such schemes have recently been identified [DDKL].


{\bf Attacks}. Attacks due to Joux [Jou03] and to Halevi and Rogaway [HR04], originally
developed in the afore-mentioned context of block cipher domain extension
(or more exactly, in the construction of tweakable block ciphers with large
domains from standard block ciphers with “small” domains) can be translated
to the context of linear SPNs as well. Specifically, these attacks imply that linear
2-round SPNs of width w ≥ 2 are insecure, as long as the underlying field has
characteristic 2.



\floatstyle{boxed}
\restylefloat{figure}

