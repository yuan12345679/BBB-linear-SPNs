
\section{Introduction}
\label{section:Introduction}

Modern blockciphers roughly fall into two classes (with some rare exceptions such as IDEA~\cite{EC:LaiMas90} and KATAN~\cite{CHES:DeCDunKne09}), namely {\it Feistel networks and their generalizations}, and {\it substitution-permutation networks} (SPNs). A Feistel round applies a domain-preserving function on half of the data, and then executes XOR and swap operations. This paradigm may be generalized to using compression functions, expansion functions, and smaller functions. Popular examples include many blockcipher standards such as DES~\cite{DESDesign}, GOST~\cite{GOSTDesign}, and Camellia~\cite{ISOIEC-18033-3:2010}. On the other hand, the latter paradigm SPNs start with a set of public permutations on the set of n-bit strings which are called S-boxes. These public permutations are then extended to a keyed permutation on $wn$-bit inputs for some integer $w$ by iterating the following steps:
\begin{enumerate}
	\item[1.] {\it Substitution step}: break down the $wn$-bit state into $w$ disjoint chunks of $n$ bits, and evaluate an S-box on each chunk;
	\item[2.] {\it Permutation step}: apply a non-cryptographic, keyed permutation to the whole $wn$-bit state (which is also applied to the plaintext before the first round).
\end{enumerate}
Various popular blockciphers including AES~\cite{AESDesign}, Serpent~\cite{serpentProposal}, and PRESENT~\cite{CHES:BKLPPR07} follow this approach. Both of the two approaches revolve around the extension of a ``complex'' function or permutation on a small domain to a keyed pseudorandom permutation on a larger domain by iterating several times simple rounds. {\bf Philosophically, SPNs expand domains significantly.}



The traditional security notion for blockciphers is pseudorandomness: for any adversary with reasonable resources, the blockcipher with {\it a random and secret key} should be indistinguishable from a truly random
permutation. Proving such security for concrete blockciphers such as AES
seems out of the reach of current techniques. The usual approach is to idealize some underlying primitives and prove that the
high-level structure is sound in a relevant security model. As for Feistel networks, a substantial line of work starting with Luby and Rackoff's seminal work~\cite{DBLP:journals/siamcomp/LubyR88}
and culminating with Patarin's results~\cite{C:patarin03} proves optimal security with
a sufficient number of rounds. Numerous other articles~\cite{C:HoaRog10,JC:CHKPST16} study the security of (variants of) Feistel networks in various
security models.




Researches on the internal structure of SPNs are relatively rare and have just been initiated in the past 5 years. We will elaborate below.




\subsection{Linear vs Non-linear Permutation steps}

Modern blockciphers tend to use linear or affine mappings for the {\it Permutation step}~\cite{DBLP:reference/crypt/Biryukov11aa}, which involves a simple key-mixing step followed
by an invertible linear or affine transformation. More precisely, their permutation steps are {\it linear} or affine with respect to additions on $GF(2^n)$, where $n$ is the size of the S-box. This actually includes all the aforementioned SPN ciphers. A small subset of them using MDS linear transformations allows for effective provable security against certain types of attacks~\cite{IMA:DaeRij01,AC:PSCYL02,FSE:PSLL03,miles2015substitution,EC:SLGRL16}.



On the other hand, as noticed by Dodis et al.~\cite{EPRINT:DKSSZZ18} (the idea of which might further date back to~\cite{FSE:ChaSar06,C:Halevi07}), the {\it Permutation step} could actually be {\it non-linear}. In detail, Dodis et al. identified a combinatorial property on the permutations that suffices for security in this case, called blockwise universality. Informally, a keyed permutation $\pi_k$ is blockwise universal if, for any distinct inputs $x,x'$ and any constant $c$, the probability (taken over uniform $k$) of each of the following events is low: (1) a block of $\pi(k,x)$ is equal to a block of $\pi(k,x')$, (2) two different blocks of $\pi(k,x)$ are equal, (3) a block of $\pi(k,x)$ is equal to $c$. Using such non-linear permutations, they showed that even one round is already sufficient for the classical birthday-bound, i.e., security up to roughly $2^{2n/3}$ adversarial queries. Later, Cogliati and Lee improved this result by proving beyond-birthday-bound results~\cite{EPRINT:DKSSZZ18}. They showed that two rounds of such non-linear SPNs are secure up to roughly $2^{2n/3}$ adversarial queries. They also provided a (non-tight) asymptotic security bound improving as the number of rounds grows.



Such models have two shortages. First, ... implementing a blockwise universal permutations might be costly, and linear functions $f_i$'s would be highly preferable for obvious efficiency reasons. More importantly, {\it far from realistic}. In fact, the idea of using non-linear transformations in real blockciphers was only recently investigated by Liu et al.~\cite{DBLP:journals/dcc/LiuRL18}.



Regarding the classical SPN model with linear permutation layers, Dodis et al. has characterized its SPRP security.
They exhibited attacks against 2 rounds using only 4 queries, and proved $n/2$ birthday security at 3 rounds.





\subsection{Our Results}

%In this paper, we ask whether it is possible to come with a tweakable Even-
%Mansour construction achieving both:
%1. a linear mixing of the tweak and the key to the state;
%2. beyond-birthday-bound security.
%We answer positively, by providing a construction with 2n-bit keys and n-bit tweaks.

In this paper, we ask whether it is possible to achieve security beyond the birthday barrier with linear SPN structures. In detail, we focus on linear SPNs with independent S-boxes and independent round keys, and we will focus on the case where $w\geq2$, since, when $w = 1$, we recover the standard Even-Mansour construction that has already been the focus
of a long line of work (as briefly reviewed later). For such linear SPNs, we prove the first beyond-birthday-bound (BBB) result on 4 rounds.





We first characterize the security of linear SPNs, where the
permutation layer is a linear function of the current $wn$-bit round key and the current $wn$-bit state. For this widely used setting we give a
general against any 2-round linear SPNs with $w\geq2$. Complementing this
attack, we show that a 3-round linear SPNs are secure, for any $w$, if the keyed linear permutations satisfy some very mild technical requirements. This result critically uses the H-coefficient technique [Pat08,CS14].





\paragraph{Implications for small block size.}

While our results are directly meaningful
when the length n of public S-boxes in at least security parameter (e.g., for
building wide block ciphers), our bounds are too weak for regular SPN-based
ciphers, such as AES, which use very low values of n for their S-boxes.
This ``$2^n$ provable barrier'' is inherent using our current modeling, where the
S-box of size $2^n$ is providing the only source of cryptographic hardness. More
generally, establishing a sound theory of building block ciphers from small S-boxes is one of the biggest and most important open problems in symmetric-key
cryptography. We hope that our structural results for reduced-round SPN ciphers will be useful in establishing such theory, despite not crossing the fundamental ``$2^n$ barrier'' mentioned above.



%
%To tweak such linear SPNs, we consider the simplest approach, i.e., directly xoring a $wn$-bit tweak with each round key, and prove BBB result on 6 rounds. We will elaborate in detail as follows.
%
%\medskip\noindent{\bf BBB Security for 4-round linear SPNs.}
%



\subsection{Related Work}


There are only a few prior papers looking at provable security of SPNs. The vast
majority of such work analyzes the case of secret, key-dependent S-boxes (rather
than public S-boxes as we consider here), and so we survey that work first.



{\bf SPNs with secret S-boxes}. Naor and Reingold [NR99] prove security for
what can be viewed as a non-linear, 1-round SPN. Their ideas were further
developed, in the context of domain extension for block ciphers (see further
discussion below), by Chakraborty and Sarkar [CS06] and Halevi [Hal07].


Iwata and Kurosawa [IK00] analyze SPNs in which the linear permutation
step is based on the specific permutations used in the block cipher Serpent. They
show an attack against 2-round SPNs of this form, and prove security for 3-round
SPNs against non-adaptive adversaries. In addition to the fact that we consider
public S-boxes, our linear SPN model considers generic linear permutations and
we prove security against adaptive attackers.

Miles and Viola [MV15] study SPNs from a complexity-theoretic viewpoint.
Two of their results are relevant here. First, they analyze the security of linear
SPNs using S-boxes that are not necessarily injective (so the resulting keyed
functions are not, in general, invertible). They show that r-round SPNs of this
type (for $r\geq 2$) are secure against chosen-plaintext attacks. (In contrast, our
results show that 2-round, linear SPNs are not secure against a combination of
chosen-plaintext and chosen-ciphertext attacks when $w\geq 2$.) They also analyze
SPNs based on a concrete set of S-boxes, but in this case they only show security
against linear/differential attacks (a form of chosen-plaintext attack), rather
than all possible attacks.


{\bf SPNs with public S-boxes}. A difference between our work and all the work
discussed above is that we treat the S-boxes as public. We are aware of only
one prior work analyzing the provable security of SPNs in this setting. Dodis
et al.~\cite{EC:DSSL16} recently studied the indifferentiability~\cite{TCC:MauRenHol04} of confusion-diffusion
networks, which can be viewed as unkeyed SPNs. One could translate
their results to the keyed setting, but that would require using multiple, key-dependent
S-boxes (rather than a fixed, public S-box) and so would not imply
our results. We remark further that they show positive results only for 5 rounds
and above.


As observed earlier, the Even-Mansour construction~\cite{JC:EveMan97} of a (keyed) pseudorandom permutation from a public random permutation can be viewed
as a 1-round, linear SPN in the degenerate case where w = 1 (i.e., no domain
extension) and all round permutations are instantiated using simple key mixing.
Security of the 1-round Even-Mansour construction against adaptive chosen-plaintext/ciphertext attacks, using independent keys for the initial and final key mixing, was shown in the original paper~\cite{JC:EveMan97}. Our positive results imply
security of the 1-round Even-Mansour construction (with similar concrete security
bounds) as a special case. The r-round generalization of the Even-Mansour
cipher has seen a lot of interest over the years, culminating with [CS14,HT16]
where it was proved that the r-round Even-Mansour construction is secure up to
roughly 2rn/(r+1) adversarial queries, when the public S-boxes are uniformly random
and independent permutations and the round keys are independent. Chen et
al. [CLL+14] also proved that several minimized variants of the 2-round Even-
Mansour construction are also secure up to roughly 22n/3 adversarial queries.
None of these results extend to the setting w > 1 considered in this work.





{\bf Cryptanalysis of SPNs}. Researchers have also explored cryptanalytic attacks
on generic SPNs [BS10,BBK14,DDKL,BK]. These works generally consider a
model of SPNs in which round permutations are secret, random (invertible) linear
transformations, and S-boxes may be secret as well; this makes the attacks
stronger but positive results weaker. In many cases the complexities of the
attacks are exponential in n (though still faster than a brute-force search for
the key), and hence do not rule out asymptotic security results. On the positive
side, Biryukov et al. [BBK14] show that 2-round SPNs (of the stronger form just
mentioned) are secure against some specific types of attacks, but other attacks
on such schemes have recently been identified [DDKL].


%{\bf Attacks}. Attacks due to Joux [Jou03] and to Halevi and Rogaway [HR04], originally developed in the afore-mentioned context of block cipher domain extension (or more exactly, in the construction of tweakable block ciphers with large domains from standard block ciphers with ``small'' domains) can be translated to the context of linear SPNs as well. Specifically, these attacks imply that linear 2-round SPNs of width $w\geq 2$ are insecure, as long as the underlying field has characteristic 2.



\floatstyle{boxed}
\restylefloat{figure}

