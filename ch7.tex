\noindent
\textbf{PEELING OFF THE OUTER TWO ROUNDS}.Pick a pair of S-box $(S_1,S_4)$ such that $\mathbf{S}_{1} \vdash \mathcal{Q}_{S_{1}}^{(0)}$ and $\mathbf{S}_{4} \vdash
\mathcal{Q}_{S_{4}}^{(0)}$ ,and for each $(x, y) \in \mathcal{Q}_{C}$ we set $a=T_{1}\left(\mathbf{S}_{1}\left(x \oplus k_{0}\right)\right)$,$b=\mathbf{S}_{4}^{-1}\left(y \oplus k_{4}4\right)$.In this way we obtain q tuples of the form $(a,b)$;for convenience we denote the set of such induced tuples by $Q_{C}^{*}\left(\mathbf{S}_{1},\mathbf{S}_{4}\right)$. Similarly,we also extended the inner two rounds:
We begin with choosing a pair of keys $\left(k_{1}, k_{3}\right) \in \mathcal{K}^{2}$ uniformly at random. Once these keys have been chosen, some construction queries will become involved in collisions. A colliding query is defined as a construction query $(a, b) \in Q_{C}^{*}\left(\mathbf{S}_{1},\mathbf{S}_{4}\right)$ such that one of the following conditions holds:

\begin{itemize}
  \item[1.]
  there exist an S-box query $(2, u, v) \in Q_{C}^{*}\left(\mathbf{S}_{1},\mathbf{S}_{4}\right)$ and an integer $i \in\{1, \ldots, w\}$ such that $\left(a \oplus k_{1}\right)[i]=u$.
  \item[2.]
  there exist an S-box query $(3, u, v) \in Q_{C}^{*}\left(\mathbf{S}_{1},\mathbf{S}_{4}\right)$ and an integer $i \in\{1, \ldots, w\}$ such that $\left(T_{3}^{-1}\left(b \oplus k_{3}\right)\right)[i]=v$.
  \item[3.]
  there exist a construction query $\left(a^{\prime}, b^{\prime}\right) \in Q_{C}^{*}\left(\mathbf{S}_{1},\mathbf{S}_{4}\right)$ and an integer $i,j \in\{1, \ldots, w\}$ such that $(a, b, i) \neq\left(a^{\prime}, b^{\prime}, j\right)$ and $\left(a \oplus k_{1}\right)[i] \neq \left(a^{\prime} \oplus k_{1}\right)[j]$.
  \item[4.]
  there exist a construction query $\left(a^{\prime}, b^{\prime}\right) \in Q_{C}^{*}\left(\mathbf{S}_{1},\mathbf{S}_{4}\right)$ and an integer $i,j \in\{1, \ldots, w\}$ such that $(a, b, i) \neq\left(a^{\prime}, b^{\prime}, j\right)$ such that $\left(T_{3}^{-1}\left(b \oplus k_{3}\right)\right)[i] \neq \left(T_{3}^{-1}\left(b^{\prime} \oplus k_{3}\right)\right)[j]$.
\end{itemize}

Then we build a new set $\mathcal{Q}_{S_{inner}}^{\prime}$ of S-box evaluations that will play the role of an extension of $Q_{C}^{*}\left(\mathbf{S}_{1},\mathbf{S}_{4}\right)$. For each colliding query $(a, b) \in Q_{C}^{*}\left(\mathbf{S}_{1},\mathbf{S}_{4}\right)$, we will add tuples $\left(2, \left(a \oplus k_{1}\right)[i], v^{\prime}\right)_{1 \leq i \leq w}$(if $(a,b)$ collides at the input of $S_2$ ) or $\left(3, u^{\prime}, \left(T_{3}^{-1}\left(b \oplus k_{3}\right)[i]\right)\right)_{1 \leq i \leq w}$(if $(a,b)$ collides at the input of $S_3$ ) by lazy sampling $v^{\prime}=S_{2}(\left(a \oplus k_{1}\right)[i])$ or $u^{\prime}=S_{3}^{-1}(\left(T_{3}^{-1}\left(b \oplus k_{3}\right) \right)[i])$,as long as it has not been determined by any existing query in $Q_{C}^{*}\left(\mathbf{S}_{1},\mathbf{S}_{4}\right)$.Then we choose the key $k_2$ uniformly at random. An extended transcript of $\tau_{inner}$ will be defined as a tuple $\tau_{inner}^{\prime}=\left(Q_{C}^{*}\left(\mathbf{S}_{1},\mathbf{S}_{4}\right), \mathcal{Q}_{S_inner}, \mathcal{Q}_{S_{inner}}^{\prime}, \mathbf{k}\right)$ where $\mathbf{k}=\left(k_{1}, k_{2},k_{3}\right)$. For each collision between a construction query and a primitive query, or between two construction queries, the extended transcript will contain enough information to compute a complete round of the evaluation of the SPN.This will be useful to lower bound the probability to get the transcript $\tau_{inner}$ in the real world.\\

for any attainable transcript $\tau_{inner}=\left(Q_{C}^{*}\left(\mathbf{S}_{1},\mathbf{S}_{4}\right), \mathcal{Q}_{S_inner}\right)$,let

$$
\begin{aligned}
&\mathrm{p}_{1}\left(\mathcal{Q}_{C} | \mathcal{Q}_{S}\right)=\operatorname{Pr}\left[\widetilde{\mathcal{P}} \stackrel{s}{\leftarrow} \widetilde{\operatorname{Per}}(\mathcal{T}, w n)^{\ell}, \mathcal{S} \stackrel{s}{\leftarrow} \operatorname{Perm}(n)^{r}: \tilde{\mathcal{P}} \vdash \mathcal{Q}_{C} | \mathcal{S} \vdash \mathcal{Q}_{S}\right]\\
&\mathrm{p}_{2}\left(\mathcal{Q}_{C} | \mathcal{Q}_{S}\right)=\operatorname{Pr}\left[\mathbf{k}_{1}, \ldots, \mathbf{k}_{\ell} \leftarrow \mathcal{K}^{r+1}, \mathcal{S} \stackrel{S}{\leftarrow} \operatorname{Perm}(n)^{r}:\left(\mathrm{SP}_{\mathbf{k}_{j}}^{T}[\mathcal{S}]\right)_{j} \vdash \mathcal{Q}_{C} | \mathcal{S} \vdash \mathcal{Q}_{S}\right]
\end{aligned}
$$

$$
\begin{aligned}
&\mathcal{Q}_{S_{2}}^{(1)}=\left\{(u, v) \in\{0,1\}^{n} \times\{0,1\}^{n}:(2, u, v) \in Q_{C}^{*}\left(\mathbf{S}_{1},\mathbf{S}_{4}\right) \cup \mathcal{Q}_{S_{inner}}^{\prime}\right\}\\
&\mathcal{Q}_{S_{3}}^{(1)}=\left\{(u, v) \in\{0,1\}^{n} \times\{0,1\}^{n}:(3, u, v) \in Q_{C}^{*}\left(\mathbf{S}_{1},\mathbf{S}_{4}\right) \cup \mathcal{Q}_{S_{inner}}^{\prime}\right\}\\
\end{aligned}
$$

$$
\begin{aligned}
&U_{2}=\left\{u_{2} \in\{0,1\}^{n}:\left(2, u_{2}, v_{2}\right) \in \mathcal{Q}_{S_{2}}^{(1)}\right\}, \quad V_{2}=\left\{v_{2} \in\{0,1\}^{n}:\left(2, u_{2}, v_{2}\right) \in \mathcal{Q}_{S_{2}}^{(1)}\right\}\\
&U_{3}=\left\{u_{3} \in\{0,1\}^{n}:\left(3, u_{3}, v_{3}\right) \in \mathcal{Q}_{S_{3}}^{(1)}\right\}, \quad V_{3}=\left\{v_{3} \in\{0,1\}^{n}:\left(3, u_{3}, v_{3}\right) \in \mathcal{Q}_{S_{3}}^{(1)}\right\}
\end{aligned}
$$



\subsection{Six Rounds on Feistel Ciphers}

\noindent \textbf{Definition 5 (Suitable Round-Key Vector for 6 Rounds)} \emph{A round-key vector $\mathit{k} = (\mathit{k}_1, \mathit{k}_2, \mathit{k}_3, \mathit{k}_4, \mathit{k}_5, \mathit{k}_6)$  is suitable if it satisfies the following conditions:}

\emph{
\begin{itemize}
  \item[1.]
  $\mathit{k}_1, \mathit{k}_3, \mathit{k}_5$ are uniformly distributed in $\{0,1\}^n$;
  \item[2.]
  $\mathit{k}_2, \mathit{k}_4, \mathit{k}_6$ are uniformly distributed in $2^{n-r}$ possibilities;
  \item[3.]
  for $(\mathit{i},\mathit{j}) \in {(1,2), (2,3), (3,4), (4,5), (5,6), (1,6)}$, $\mathit{k}_i$ and $\mathit{k}_j$ are independent.
\end{itemize}
}

\noindent in the subsequent analysis we find the uniformness of every round-key crucial. This is why we require all of them to be uniform (this is also understandable, since beyond-birthday security requires various types of collisions can be bounded by small enough probability, and thus requiring a larger amount of randomness). The (mild) independence is also crucially used in the analysis.\\

Instantiated with such a suitable round-key vector, KAF ensures beyond-birthday security.\\

\noindent \textbf{Lemma 6} \emph{For the 6-round idealized cipher KAF with a suitable round-key vector as specified in Definition 1, it holds}

$$
\begin{aligned}
&\mathbf{A} \mathrm{d} \mathbf{v}_{K A F}^{S U}\left(q_{f}, q_{e}\right) \leq \frac{7 q_{e}^{3}+13 q_{e} q_{f}^{2}+22 q_{e}^{2} q_{f}}{N^{2}}+\frac{2^{r}\left(8 q_{e} q_{f}^{2}+2 q_{e}^{2} q_{f}\right)}{N^{2}}, \text { and }\\
&\mathbf{A} \mathrm{d} \mathbf{v}_{K A F}^{M U}\left(q_{f}, q_{e}\right) \leq \frac{1214 q_{e}^{3}+26 q_{e} q_{f}^{2}+356 q_{e}^{2} q_{f}}{N^{2}}+\frac{2^{r}\left(600 q_{e}^{3}+16 q_{e} q_{f}^{2}+196 q_{e}^{2} q_{f}\right)}{N^{2}}
\end{aligned}
$$

Note that when $r \leq n/2$, the security is beyond-birthday and when $r=0$, the bound is of ``typical`` beyond-birthday form $\mathnormal{O}(\frac{q^3}{N^2})$.\\

For all possible transcript $\tau$ that describes a possible interaction with either a tuple of oracles $(\operatorname{P}^{(1)}, \cdots, \operatorname{P}^{(m)},\operatorname{F})$ or $(\operatorname{KAF}^F_{k_1}, \cdots, \operatorname{KAF}^F_{k_m}, \operatorname{F})$, we denote $\operatorname{Pr}_{re}(\tau)$, resp.$\operatorname{Pr}_{id}(\tau)$, the probability that $\mathnormal{D}$  interaction with the real world, resp. the ideal world, produces $\tau$. Formally,

$$
\begin{aligned}
\operatorname{Pr}_{r e}(\tau)=\operatorname{Pr}\left[\left(k^{(1)}, \ldots, k^{(m)}\right)\right.& \leftarrow(\mathcal{K})^{m}, \mathbf{F} \stackrel{\mathbf{S}}{\leftarrow}(\mathcal{F}(n))^{t}: \\
&\left.\mathrm{KAF}_{k^{(1)}}^{\mathrm{F}} \vdash \mathcal{Q}_{E_{1}} \wedge \ldots \wedge \mathrm{K} \mathrm{AF}_{k^{(m)}}^{\mathrm{F}} \vdash \mathcal{Q}_{E_{m}} \wedge \mathbf{F} \vdash \mathcal{Q}_{F}\right]\\
\operatorname{Pr}_{i d}(\tau)=\operatorname{Pr}\left[\left(\mathbf{P}^{(1)}, \ldots, \mathbf{P}^{(m)}\right)\right.& \stackrel{\mathbf{g}}{\leftarrow}(\mathcal{P}(2 n))^{m}, \mathbf{F} \stackrel{\mathbf{S}}{\leftarrow}(\mathcal{F}(n))^{t}: \\
&\left.\mathbf{P}^{(1)} \vdash \mathcal{Q}_{E_{1}} \wedge \ldots \wedge \mathbf{P}^{(m)} \vdash \mathcal{Q}_{E_{m}} \wedge \mathbf{F} \vdash \mathcal{Q}_{F}\right]
\end{aligned}
$$

With these definitions, we devote to prove the following point-wise proximity result for the SU setting: for any transcript $\tau$, it holds
$$
\operatorname{Pr}_{i d}(\tau)-\operatorname{Pr}_{r e}(\tau) \leq \operatorname{Pr}_{i d}(\tau) \cdot \frac{7 q_{e}^{3}+13 q_{e} q_{f}^{2}+22 q_{e}^{2} q_{f}+2^{r}\left(8 q_{e} q_{f}^{2}+2 q_{e}^{2} q_{f}\right)}{N^{2}}
$$

Fix a transcript $\tau = (\mathcal{Q}_E, \mathcal{Q}_F)$ with $\mathcal{Q}_F = (\mathcal{Q}_{F_1}, \mathcal{Q}_{F_2}, \mathcal{Q}_{F_3}, \mathcal{Q}_{F_4}, \mathcal{Q}_{F_5}, \mathcal{Q}_{F_6})$, $\left|\mathcal{Q}_E \right| = \mathnormal{q}_\mathnormal{e}$ and $\left|\mathcal{Q}_{F_i} \right| = \mathnormal{q}_\mathnormal{f}$ for $\mathnormal{i} = 1, \cdots, 6$, then first need to define bad key-vectors, then lower bound the probability $Pr_{re}(\tau, k)$.

after define the bad key-vectors,we can get the probability of it
$$
\operatorname{Pr}\left[k \stackrel{s}{\leftarrow} \mathcal{K}: k \in \mathcal{K}_{b a d}\right] \leq \frac{3 \cdot 2^{r} \cdot q_{e} a_{f}^{2}}{N^{2}}
$$

Then analysis for good keys. Fix a good round-key vector $\mathnormal{k}$, we are to derive a lower bound for the probability $\operatorname{Pr}\left[\mathbf{F} \stackrel{\mathrm{S}}{\leftarrow}(\mathcal{F}(n))^{6}: \mathrm{K} \mathrm{AF}_{k}^{\mathrm{F}} \vdash \mathcal{Q}_{E} | \mathbf{F} \vdash \mathcal{Q}_{F}\right]$. It consists of two steps. In the first step, lower bound the probability that a pair of functions  $(\mathrm{F}_1, \mathrm{F}_6)$ satisfies certain ``bad'' conditions that will be defined. With the values given by a ``good'' pair of functions $(\mathrm{F}_1, \mathrm{F}_6)$, a transcript of the distinguisher on 6 rounds can be transformed into a special transcript on 4 rounds; in this sense, we ``peel off'' the outer two rounds. Then in the second step, assuming $(\mathrm{F}_1, \mathrm{F}_6)$ is good, we analyze the induced 4-round transcript to yield the final bounds. In the following, each step would take a subsubsection. \\

\noindent Then\\

\noindent \textbf{Lemma 7} \emph{ It holds}
$$
\begin{aligned}
& \operatorname{Pr}_{\mathbf{F}_{1}, \mathbf{F}_{6}}\left[\operatorname{Bad}\left(\mathbf{F}_{1}, \mathbf{F}_{6}\right) | \mathbf{F}_{1} \vdash \mathcal{Q}_{F_{1}} \wedge \mathbf{F}_{6} \vdash \mathcal{Q}_{F_{6}}\right] \\
\leq & \frac{q_{e} q_{f}^{2}}{N^{2}}+\frac{4 q_{e}^{2} q_{f}}{N^{2}}+\frac{\alpha_{2,3}(k)+\alpha_{4,5}(k)}{N}+\frac{q_{f}\left(\alpha_{1}(k)+\alpha_{2}(k)\right)}{N}
\end{aligned}
$$

Then analyse the inner four rounds. Let $\mathrm{F}^* = (\mathrm{F}_2, \mathrm{F}_3, \mathrm{F}_4, \mathrm{F}_5)$, we denote $\mathrm{p}\left(\tau, \mathbf{F}_{1}, \mathbf{F}_{6}\right)=\operatorname{Pr}\left[\mathbf{F}^{*} \stackrel{\mathrm{s}}{\leftarrow}(\mathcal{F}(n))^{4}: \mathrm{KAF}_{k}^{\mathrm{F}^{*}} \vdash \mathcal{Q}_{E}^{*}\left(\mathbf{F}_{1}, \mathbf{F}_{6}\right) | \mathbf{F}_{i} \vdash \mathcal{Q}_{F_{i}}, i=1,2,3,4,5,6\right]$. This captures the probability that the inner four rounds of KAF ``extend''  the tuples in $\mathcal{Q}_{E}^{*}\left(\mathbf{F}_{1}, \mathbf{F}_{6}\right)$. The probability $Pr_{re}(\tau, k)$ can be related to it.\\

\noindent \textbf{Lemma 7} \emph{Assume that there exists a function $\epsilon : ((\mathbf{F}(n)))^2 \times \mathbf{K} \rightarrow [ 0, \infty)$ such that for any good $(\mathbf{F}_{1}, \mathbf{F}_{6})$, it holds}

$$
p\left(\tau, \mathbf{F}_{1}, \mathbf{F}_{6}\right) / \prod_{i=0}^{q_{e}-1}\left(\frac{1}{N^{2}-i}\right) \geq 1-\epsilon\left(\mathbf{F}_{1}, \mathbf{F}_{6}, k\right)
$$

\emph{Then we have}
$$
\begin{aligned}
\frac{\operatorname{Pr}_{r e}(\tau, k)}{\operatorname{Pr}_{i d}(\tau, k)} \geq 1 &-\operatorname{Pr}\left[\operatorname{Bad}\left(\mathbf{F}_{1}, \mathbf{F}_{6}\right) | \mathbf{F}_{1} \vdash \mathcal{Q}_{F_{1}}, \mathbf{F}_{6} \vdash \mathcal{Q}_{F_{6}}\right] \\
&-\mathbb{E}_{\mathbf{F}_{1}, \mathbf{F}_{6}}\left[\epsilon\left(\mathbf{F}_{1}, \mathbf{F}_{6}, k\right) | \mathbf{F}_{1} \vdash \mathcal{Q}_{F_{1}}, \mathbf{F}_{6} \vdash \mathcal{Q}_{F_{6}}\right]
\end{aligned}
$$

Therefore\\

\noindent \textbf{Lemma 8} \emph{ For any fixed good tuple $(\mathbf{F}_{1}, \mathbf{F}_{6})$, there exists a function $\epsilon\left(\mathbf{F}_{1}, \mathbf{F}_{6}, k\right)$ of the function pair and the round-key vector $\mathnormal{k}$ such that the inequality mentioned in Lemma 7 holds. Moreover,}

$$
\mathbb{E}_{\mathbf{F}_{1}, \mathbf{F}_{6}, k}\left[\epsilon\left(\mathbf{F}_{1}, \mathbf{F}_{6}, k\right)\right] \leq \frac{7 q_{e}^{3}+10 q_{e} q_{f}^{2}+18 q_{e}^{2} q_{f}+3 \cdot 2^{r} \cdot q_{e} q_{f}^{2}+2 \cdot 2^{r} \cdot q_{e}^{2} q_{f}}{N^{2}}
$$

Then combine all of this lemma will get the conclusion.






A colliding query is defined as a construction query $(a, b) \in Q_{C}^{*}\left(\mathbf{S}_{1},\mathbf{S}_{4}\right)$ such that one of the following conditions holds:

\begin{itemize}
  \item[1.]
  there exist an S-box query $(2, u, v) \in Q_{C}^{*}\left(\mathbf{S}_{1},\mathbf{S}_{4}\right)$ and an integer $i \in\{1, \ldots, w\}$ such that $\left(a \oplus k_{1}\right)[i]=u$.
  \item[2.]
  there exist an S-box query $(3, u, v) \in Q_{C}^{*}\left(\mathbf{S}_{1},\mathbf{S}_{4}\right)$ and an integer $i \in\{1, \ldots, w\}$ such that $\left(T_{3}^{-1}\left(b \oplus k_{3}\right)\right)[i]=v$.
  \item[3.]
  there exist a construction query $\left(a^{\prime}, b^{\prime}\right) \in Q_{C}^{*}\left(\mathbf{S}_{1},\mathbf{S}_{4}\right)$ and an integer $i,j \in\{1, \ldots, w\}$ such that $(a, b, i) \neq\left(a^{\prime}, b^{\prime}, j\right)$ and $\left(a \oplus k_{1}\right)[i] \neq \left(a^{\prime} \oplus k_{1}\right)[j]$.
  \item[4.]
  there exist a construction query $\left(a^{\prime}, b^{\prime}\right) \in Q_{C}^{*}\left(\mathbf{S}_{1},\mathbf{S}_{4}\right)$ and an integer $i,j \in\{1, \ldots, w\}$ such that $(a, b, i) \neq\left(a^{\prime}, b^{\prime}, j\right)$ such that $\left(T_{3}^{-1}\left(b \oplus k_{3}\right)\right)[i] \neq \left(T_{3}^{-1}\left(b^{\prime} \oplus k_{3}\right)\right)[j]$.
\end{itemize}

Then we build a new set $\mathcal{Q}_{S_{inner}}^{\prime}$ of S-box evaluations that will play the role of an extension of $Q_{C}^{*}\left(\mathbf{S}_{1},\mathbf{S}_{4}\right)$. For each colliding query $(a, b) \in Q_{C}^{*}\left(\mathbf{S}_{1},\mathbf{S}_{4}\right)$, we will add tuples $\left(2, \left(a \oplus k_{1}\right)[i], v^{\prime}\right)_{1 \leq i \leq w}$(if $(a,b)$ collides at the input of $S_2$ ) or $\left(3, u^{\prime}, \left(T_{3}^{-1}\left(b \oplus k_{3}\right)[i]\right)\right)_{1 \leq i \leq w}$ (if $(a,b)$ collides at the input of $S_3$ ) by lazy sampling, as long as it has not been determined by any existing query in $Q_{C}^{*}\left(\mathbf{S}_{1},\mathbf{S}_{4}\right)$, $v^{\prime}=S_{2}(\left(a \oplus k_{1}\right)[i])$ or $u^{\prime}=S_{3}^{-1}(\left(T_{3}^{-1}\left(b \oplus k_{3}\right) \right)[i])$. Then we choose the key $k_2$ uniformly at random. An extended transcript of $\tau_{inner}$ will be defined as a tuple $\tau_{inner}^{\prime}=\left(Q_{C}^{*}\left(\mathbf{S}_{1},\mathbf{S}_{4}\right), \mathcal{Q}_{S_{inner}}, \mathcal{Q}_{S_{inner}}^{\prime}, \mathbf{k}\right)$ where $\mathbf{k}=\left(k_{1}, k_{2},k_{3}\right)$. For each collision between a construction query and a primitive query, or between two construction queries, the extended transcript will contain enough information to compute a complete round of the evaluation of the SPN. This will be useful to lower bound the probability to get the transcript $\tau_{inner}$ in the real world.\\






\item[(B-4)]
  there exist $(x, y) \in \mathcal{Q}_{C}$,$i,i^{\prime},j,j^{\prime} \in\{1, \ldots, w\}$, $\left(u_{2}, v_{2}\right) \in \mathcal{Q}_{S_{2}},\left(u_{3}, v_{3}\right) \in \mathcal{Q}_{S_{3}}$, such that $\left(T_{1}\left(\mathbf{S}_{1}\left(x \oplus k_{0}\right) \oplus k_{1}\right)\right)[i]=u_2$ and $$\left(T_{1}\left(\mathbf{S}_{1}\left(x \oplus k_{0}\right) \oplus k_{1}\right)\right)[i] = \left(T_{1}\left(\mathbf{S}_{1}\left(x \oplus k_{0}\right) \oplus k_{1}\right)\right)[i^{\prime}]$$ or $\left(T_{3}^{-1}\left(\mathbf{S}_{4}^{-1}\left(y \oplus k_{4}\right) \oplus k_{3}\right)\right)[j]=v_3$ and

  $$
  \begin{aligned}
  \left(T_{3}^{-1}\left(\mathbf{S}_{4}^{-1}\left(y \oplus k_{4}\right) \oplus k_{3}\right)\right)[j]=\left(T_{3}^{-1}\left(\mathbf{S}_{4}^{-1}\left(y \oplus k_{4}\right) \oplus k_{3}\right)\right)[j^{\prime}]
  \end{aligned}
  $$.
  \item[(B-5)]
  there exist $(x, y),(x^{\prime}, y^{\prime}) \in Q_{C}$,$i,i^{\prime},j,j^{\prime} \in\{1, \ldots, w\}$,$\left(u_{2}, v_{2}\right) \in \mathcal{Q}_{S_{2}},\left(u_{3}, v_{3}\right) \in \mathcal{Q}_{S_{3}}$ such that $\left(T_{1}\left(\mathbf{S}_{1}\left(x \oplus k_{0}\right) \oplus k_{1}\right)\right)[i]=u_2$ and $$\left(T_{1}\left(\mathbf{S}_{1}\left(x \oplus k_{0}\right) \oplus k_{1}\right)\right)[i]=\left(T_{1}\left(\mathbf{S}_{1}\left(x^{\prime} \oplus k_{0}\right) \oplus k_{1}\right)\right)[i^{\prime}]$$ or $\left(T_{3}^{-1}\left(\mathbf{S}_{4}^{-1}\left(y \oplus k_{4}\right) \oplus k_{3}\right)\right)[j]=v_3$ and $$\left(T_{3}^{-1}\left(\mathbf{S}_{4}^{-1}\left(y \oplus k_{4}\right) \oplus k_{3}\right)\right)[j]=\left(T_{3}^{-1}\left(\mathbf{S}_{4}^{-1}\left(y^{\prime} \oplus k_{4}\right) \oplus k_{3}\right)\right)[j^{\prime}]$$.
  
  
  
  
  \noindent \textsc{Consider (B-4) and (B-5)}. If there exist $(x, y) \in \mathcal{Q}_{C}$, $\left(u_{2}, v_{2}\right) \in \mathcal{Q}_{S_{2}}, \left(u_{3}, v_{3}\right) \in \mathcal{Q}_{S_{3}}$, for distinct $i,i^{\prime} \in\{1, \ldots, w\}$ (or, $j,j^{\prime} \in\{1, \ldots, w\}$), then $\left(T_{1}\left(\mathbf{S}_{1}\left(x \oplus k_{0}\right) \oplus k_{1}\right)\right)[i] = u_2$ and $\left(T_{1}\left(\mathbf{S}_{1}\left(x \oplus k_{0}\right) \oplus k_{1}\right)\right)[i] = \left(T_{1}\left(\mathbf{S}_{1}\left(x \oplus k_{0}\right) \oplus k_{1}\right)\right)[i^{\prime}]$ (or,$\left(T_{3}^{-1}\left(\mathbf{S}_{4}^{-1}\left(y \oplus k_{4}\right) \oplus k_{3}\right)\right)[j] = v_3$ and $\left(T_{3}^{-1}\left(\mathbf{S}_{4}^{-1}\left(y \oplus k_{4}\right) \oplus k_{3}\right)\right)[j] = \left(T_{3}^{-1}\left(\mathbf{S}_{4}^{-1}\left(y \oplus k_{4}\right) \oplus k_{3}\right)\right)[j^{\prime}]$), because every entry in the $i_{0}$th column of $T_{1}$ ( or,$T_{3}$) is nonzero, the probability that $\left(T_{1}\left(\mathbf{S}_{1}\left(x \oplus k_{0}\right) \oplus k_{1}\right)\right)[i] = \left(T_{1}\left(\mathbf{S}_{1}\left(x \oplus k_{0}\right) \oplus k_{1}\right)\right)[i^{\prime}]$ (or, $\left(T_{3}^{-1}\left(\mathbf{S}_{4}^{-1}\left(y \oplus k_{4}\right) \oplus k_{3}\right)\right)[j] = \left(T_{3}^{-1}\left(\mathbf{S}_{4}^{-1}\left(y \oplus k_{4}\right) \oplus k_{3}\right)\right)[j^{\prime}]$) conditioned on $\left(T_{1}\left(\mathbf{S}_{1}\left(x \oplus k_{0}\right) \oplus k_{1}\right)\right)[i] = u_2$( or,$\left(T_{3}^{-1}\left(\mathbf{S}_{4}^{-1}\left(y \oplus k_{4}\right) \oplus k_{3}\right)\right)[j] = v_3$ ) is upper bounded by $\frac{w-1}{N}$, Thus, by summing over every construction query, one has

$$
\operatorname{Pr}\left[(B-4)\right] \leq \frac{w (w-1) q p}{N^{2}}
$$

\noindent similar to (B-5),

$$
\operatorname{Pr}\left[(B-5)\right] \leq \frac{w (w-1) p q (q-1)}{N^{2}}
$$









\noindent \textsc{Condition (C-4) and Condition (C-5)}. Fix distinct $(a,b,j),(a^{\prime},b^{\prime},j^{\prime}) \in Q_{C}^{*}\left(\mathbf{S}_{1},\mathbf{S}_{4}\right) \times \{1, \ldots, w\}$. Then either $(a,b) = (a^{\prime},b^{\prime}), j \neq j^{\prime}$ or $(a,b) \neq (a^{\prime},b^{\prime}), but $j = j^{\prime}$.
Assume first that $a \neq a^{\prime}$. Then $a[i] \neq a^{\prime}[i]$ for some $i \in \{1, \ldots, w\}$. By the definition of a good transcript, $\left(T_{1}\left(\mathbf{S}_{1}\left(x \oplus k_{0}\right) \oplus k_{1}\right)\right)[i] \neq \left(T_{1}\left(\mathbf{S}_{1}\left(x \oplus k_{0}\right) \oplus k_{1}\right)\right)[i^{\prime}]$ for all $i^{\prime} \neq i$ and $\left(T_{1}\left(\mathbf{S}_{1}\left(x \oplus k_{0}\right) \oplus k_{1}\right)\right)[i^{\prime}] \neq \left(T_{1}\left(\mathbf{S}_{1}\left(x^{\prime} \oplus k_{0}\right) \oplus k_{1}\right)\right)[i]$ for all $i$. So after conditioning on $\left(a \oplus k_{1}\right)[i]=u_2,\left(a^{\prime} \oplus k_{1}\right)[i^{\prime}]=u_2^{\prime}$, that is $\left(T_{1}\left(\mathbf{S}_{1}\left(x \oplus k_{0}\right) \oplus k_{1}\right)\right)[i]=u_2, \left(T_{1}\left(\mathbf{S}_{1}\left(x^{\prime} \oplus k_{0}\right) \oplus k_{1}\right)\right)[i^{\prime}]=u_{2}^{\prime}$,  the value of $\left(S_{2}\left(a \oplus k_{1}\right)\right)[j]$ is uniform in a set of size at least $ N -p -2 w +1$. Because every entry in the $i_{0}$th column of $T_{2}$ is nonzero, we have

$$
\operatorname{Pr}\left[\left(T_{2}\left(\mathbf{S}_{2}\left(a \oplus k_{1}\right)\right) \oplus k_{2}\right)[j]=\left(T_{2}\left(\mathbf{S}_{2}\left(a^{\prime} \oplus k_{1}\right)\right) \oplus k_{2}\right)[j^{\prime}]\right] \leq \frac{1}{N -p -2 w -1}
$$

Assume next that $a = a^{\prime}$ and so $j \neq j^{\prime}$.Since $T_{2}$ and $T_{3}$ is invertible, the $j$ ${\text{th}}$ and $j^{\prime}$ ${\text{th}}$ rows of $T_{2}$ and $T_{3}$ are linearly independent and, in particular, there exists an index $i \in \{1, \ldots, w\}$  such that the $(j,i)$ ${\text{th}}$ and $(j^{\prime},i)$ ${\text{th}}$ entries of $T_{2}$ and $T_{3}$ are not equal. After conditioning on $\left(T_{1}\left(\mathbf{S}_{1}\left(x \oplus k_{0}\right) \oplus k_{1}\right)\right)[i] \neq \left(T_{1}\left(\mathbf{S}_{1}\left(x \oplus k_{0}\right) \oplus k_{1}\right)\right)[i^{\prime}]$ for all $i^{\prime} \neq i$, the value of $\left(S_{2}\left(a \oplus k_{1}\right)\right)[j]$ is uniform in a set of size at least $ N -p -w +1$. It follows that

$$
\operatorname{Pr}\left[\left(T_{2}\left(\mathbf{S}_{2}\left(a \oplus k_{1}\right)\right) \oplus k_{2}\right)[j]=\left(T_{2}\left(\mathbf{S}_{2}\left(a^{\prime} \oplus k_{1}\right)\right) \oplus k_{2}\right)\right][j^{\prime}] \leq \frac{1}{N -p -w -1}
$$

The statement now follows by taking a union bound over all possible pairs of distinct elements $(a,b,j),(a^{\prime},b^{\prime},j^{\prime}) \in Q_{C}^{*}\left(\mathbf{S}_{1},\mathbf{S}_{4}\right) \times \{1, \ldots, w\}$.$(u_{2},v_{2}) \in \mathcal{Q}_{S_{2}}^{(1)},(u_{3},v_{3}) \in \mathcal{Q}_{S_{3}}^{(1)}$. So

$$
\operatorname{Pr}\left[\tau_{inner} \in \Theta_{4}\right] \leq \frac{w^{2} q^{2} (p+w q)^{2}}{(N -p -2 w) \cdot (N-p)^{2}}.
$$

\noindent Similarly, one has

$$
\operatorname{Pr}\left[\tau_{inner} \in \Theta_{5}\right] \leq \frac{w^{2} q^{2} (p+w q)^{2}}{(N -p -2 w) \cdot (N-p)^{2}}.
$$



Similar to the above mentioned, conditioned on $S_{1} \vdash \mathcal{Q}_{S_{1}}$ and $S_{4} \vdash \mathcal{Q}_{S_{4}}$, the value $\left(S_{1}\left(x \oplus k_{0}\right)\right)[i]$ remain uniform in $\{0, 1\}^{n} \verb|\| (\mathcal{Q}_{S_{1}} \cup \mathcal{Q}_{S_{4}})$. Fix distinct $(a,b,i),(a^{\prime},b^{\prime},i^{\prime}) \in \mathcal{Q}_{C}^{*}\left(S_{1},S_{4}\right) \times \{1, \ldots, w\}$ such that $\left(a \oplus k_{1}\right)[i] = u_2, \left(a^{\prime} \oplus k_{1}\right)[i^{\prime}] = u_2^{\prime}$, then either $u_2 = u_2^{\prime}$ or $u_2 \neq u_2^{\prime}$. Assume first that $u_2 = u_2^{\prime}$. Then conditioned on $\left(a \oplus k_{1}\right)[i] = u_2, \left(a^{\prime} \oplus k_{1}\right)[i^{\prime}] = u_2^{\prime}$ and $u_2 = u_2^{\prime}$, there must be $j, j^{\prime} \in\{1, \ldots, w\}$ such that
$$
\left(T_{2}\left(S_{2}\left(a \oplus k_{1}\right)\right) \oplus k_{2}\right)[j] = \left(T_{2}\left(S_{2}\left(a^{\prime} \oplus k_{1}\right)\right) \oplus k_{2}\right)[j^{\prime}].
$$
since we have at most $w^2 q^2 (p+w q)$ such tuples, so in this situation, we have
$$
\operatorname{Pr}\left[\tau_{inner} \in \Theta_{4}\right] \leq \frac{w^{2} q^2 (p+w q)}{(N-p)^2}.
$$

Assume next that $u_2 \neq u_2^{\prime}$. Because there are $w q (p+w q)$  tuples such that $\left(a \oplus k_{1}\right)[i] = u_2$, and if $u_2$ is fixed, to make 
$$
\left(T_{2}\left(S_{2}\left(a \oplus k_{1}\right)\right) \oplus k_{2}\right)[j] = \left(T_{2}\left(S_{2}\left(a^{\prime} \oplus k_{1}\right)\right) \oplus k_{2}\right)[j^{\prime}].
$$
\noindent achieved, the value of $\left(a^{\prime} \oplus k_{1}\right)[i^{\prime}] = u_2^{\prime}$ must be fixed. Thus conditioned $\left(a \oplus k_{1}\right)[i] = u_2$, since the random choice of $k_{2}$, then,
$$
\operatorname{Pr}\left[\left(T_{2}\left(S_{2}\left(a \oplus k_{1}\right)\right) \oplus k_{2}\right)[j]=\left(T_{2}\left(S_{2}\left(a^{\prime} \oplus k_{1}\right)\right) \oplus k_{2}\right)[j^{\prime}]\right] \leq \frac{w q}{N}
$$ 
\noindent so 
$$
\operatorname{Pr}\left[\tau_{inner} \in \Theta_{4}\right] \leq \frac{w^{2} q^2 (p+w q)}{N (N-p)}.
$$
\noindent compare these two situation, we have
$$
\operatorname{Pr}\left[\tau_{inner} \in \Theta_{4}\right] \leq \frac{w^{2} q^2 (p+w q)}{(N-p)^2}.
$$

\noindent Similarly, one has
$$
\operatorname{Pr}\left[\tau_{inner} \in \Theta_{5}\right] \leq \frac{w^{2} q^2 (p+w q)}{(N-p)^2}.





\textsc{Tweakable Permutations.} For an integer $m \geq 1$, the set of all permutations
on $\{0,1\}^{m}$ will be denoted $\operatorname{Perm}(m)$. A tweakable permutation with tweak space $\mathcal{T}$ and message space $\mathcal{X}$ is a mapping $\widetilde{P} :\mathcal{T} \times \mathcal{X} \rightarrow \mathcal{X}$ such that, for any tweak $t \in \mathcal{T}$,
$$
x \mapsto \widetilde{P}(t,x)
$$
\noindent is a permutation of $\mathcal{X}$. The set of all tweakable permutations with tweak space $\mathcal{T}$ and message space $\{0,1\}^{m}$ will be denoted $\widetilde{\operatorname{Perm}}(\mathcal{T}, m)$.
  A keyed tweakable permutation with key space $\mathcal{K}$, tweak space $\mathcal{T}$ and message space $\mathcal{X}$ is a mapping $T : \mathcal{K} \times \mathcal{T} \times \mathcal{X} \rightarrow \mathcal{X}$ such that, for any key $k \in \mathcal{K}$,
$$
(t,x) \mapsto T(k,t,x)
$$
\noindent is a tweakable permutation with tweak space $\mathcal{T}$ and message space $\mathcal{X}$. We will sometimes write $T(k,t,x)$ as $T_k(t,x)$ or $T_{k,t}(x)$. For an integer $s \geq 1$, let $\mathbf{t}=(t_1, \cdots, t_s) \in \mathcal{T}^s$, and let $\mathbf{x}=(x_1, \ldots, x_s) \in \mathcal{X}^{\ast s}$. We will write $(T(k, t_i, x_i))_{1 \leq i \leq s}$ as $T_k(\mathbf{t}, \mathbf{x})$ or $T_{k, \mathbf{t}}(\mathbf{x})$.
\textsc{Tweakable SPNs.} For fixed parameters $w$ and $n$, let
$$
T: \mathcal{K} \times \mathcal{T} \times \{0,1\}^{w n} \rightarrow \{0,1\}^{w n}
$$
\noindent be a keyed tweakable permutation with key space $\mathcal{K}$, tweak space $\mathcal{T}$ and message space $\{0,1\}^{w n}$.
  For a fixed number of rounds $r$, an $r$-round substitution-permutation network (SPN) based on $T$, denoted $\operatorname{SP}^T$, takes as input a set of $n$-bit permutations $\mathcal{S} = (S_1, \ldots, S_r)$, and defines a keyed tweakable permutation $\operatorname{SP}^{T}[S]$ operating on $w n$-bit blocks with key space $\mathcal{K}^{r+1}$ and tweak space $\mathcal{T}$: on input $x \in \{0,1\}^{w n}$, key $\mathbf{k} = (k_0, k_1, \ldots, k_r) \in \mathcal{K}^{r+1}$ and tweak $t \in \mathcal{T}$, the output of $\operatorname{SP}^T[\mathcal{S}]$ is computed as follows.
  
\begin{itemize}
  \item[--]
  $y \leftarrow x$
  \item[--]
  For $i \leftarrow 1$ to $\mathnormal{r}$ do:
  \begin{itemize}
    \item[1.]
    $y \leftarrow T_{k_{i-1}, t}(y)$.
    \item[2.]
    Break $y = y_1 \| \cdots \| y_w$ into $n$-bit blocks.
    \item[3.]
   $y \leftarrow S_i(y_1) \| \cdots \| S_i(y_w)$.
  \end{itemize}
  \item[--]
  $y \leftarrow T_{k_{r}, t}(y)$.
\end{itemize}

\textsc{Remark 1.} Both of the permutation layer $T$ and the entire construction $\operatorname{SP}^T$ can be viewed as keyed tweakable permutations. However, $T$ will typically be built upon non-cryptographic operations such as filed multiplications, while  $\operatorname{SP}^T$ are based on S-boxes which are modeled as public random permutations.

\textsc{Blockwise Universal Tweakable Permutations.} A keyed tweakable permutation
$$
T: \mathcal{K} \times \mathcal{T} \times \{0,1\}^{w n} \rightarrow \{0,1\}^{w n}
$$
\noindent is called $(\delta, \delta')$- \emph{blockwise universal} if the following hold.

\begin{itemize}
  \item[1.]
  For all distinct $(t, x, i), (t', x', i') \in \mathcal{T} \times \{0,1\}^{w n} \times {1, \ldots, w}$, we have
  $$
  \operatorname{Pr}\left[\mathbf{k} \stackrel{s}{\leftarrow} \mathcal{K}: T_{k,t}(x)_i = T_{k,t'}(x')_{i'}\right] \leq \delta.
  $$
  \item[2.]
  For all $(t, x, i, c) \in \mathcal{T} \times  \{0,1\}^{w n} \times {1, \ldots, w} \times \{0,1\}^{n}$, we have
  $$
  \operatorname{Pr}\left[\mathbf{k} \stackrel{s}{\leftarrow} \mathcal{K}: T_{k,t}(x)_i = c \right] \leq \delta'.
  $$
\end{itemize}

\noindent Since each pair of key $k \in \mathcal{K}$ and tweak $t \in \mathcal{T}$ defines a permutation $T_{k, t}$ on $\{0,1\}^{w n}$, one can define a keyed tweakable permutation
$$
T^{-1} : \mathcal{K} \times \mathcal{T} \times \{0,1\}^{w n} \rightarrow \{0,1\}^{w n}
$$
\noindent such that $T^{-1}(k, t, x) = (T_{k,t})^{-1}(x)$. If $T$ and $T^{-1}$ are both $(\delta, \delta')$- \emph{blockwise universal}, then $T$ is called $(\delta, \delta')$- \emph{super blockwise universal}.


We denote $a=T_{1}\left(S_{1}\left(x \oplus k_{0}\right)\right)$, $b=\S_{4}^{-1}\left(y \oplus k_{4}\right)$.

\item[2.]
  there exist an S-box query $(2, u, v) \in \mathcal{Q}_{S}$ and an integer $i \in\{1, \ldots, w\}$ such that $\left(a \oplus k_{1} \opus t \right)[i]=u$.

  \item[6.]
  there exist a construction query $\left(t', a^{\prime}, b^{\prime}\right) \in \mathcal{Q}_{C}$ and an integer $i,j \in\{1, \ldots, w\}$ such that $(t, a, b, i) \neq\left(t', a^{\prime}, b^{\prime}, j\right)$ and $\left(a \oplus k_{1} \oplus t \right)[i] = \left(a' \oplus k_{1} \oplus t \right)[j]$.

 \item[7.]
  there exist a construction query $\left(t', a^{\prime}, b^{\prime}\right) \in \mathcal{Q}_{C}$ and an integer $i,j \in\{1, \ldots, w\}$ such that $(t, a, b, i) \neq\left(t, a^{\prime}, b^{\prime}, j\right)$ and $i \in\{1, \ldots, w\}$ such that $\left(b \oplus k_{5}\right)[i] = \left(b' \oplus k_{5}\right)[j]$.

$\left(2, \left(a \oplus k_{1} \oplus t\right)[i], v_2^{\prime}\right)_{1 \leq i \leq w}$ (if ($t, \mathit{a}$, $\mathit{b}$) collides at the input of $S_2$), $\left(6, u_6^{\prime}, \left(y \oplus k_{6} \oplus t\right)[i]\right)_{1 \leq i \leq w}$ (if ($t, \mathit{x}$, $\mathit{y}$) collides at the output of $S_6$), $v_2^{\prime}=S_{2}(\left(a \oplus k_{1} \oplus t\right)[i])$ 

&\mathcal{Q}_{S_{2}}^{(1)}=\left\{(u, v) \in\{0,1\}^{n} \times\{0,1\}^{n}:(2, u, v) \in \mathcal{Q}_{S} \cup \mathcal{Q}_{S_{outer}}^{\prime}\right\}\\
&\mathcal{Q}_{S_{5}}^{(1)}=\left\{(u, v) \in\{0,1\}^{n} \times\{0,1\}^{n}:(5, u, v) \in \mathcal{Q}_{S} \cup \mathcal{Q}_{S_{outer}}^{\prime}\right\}\\

&U_{2}=\left\{u_{2} \in\{0,1\}^{n}:\left(2, u_{2}, v_{2}\right) \in \mathcal{Q}_{S_{2}}^{(1)}\right\}, \quad V_{2}=\left\{v_{2} \in\{0,1\}^{n}:\left(2, u_{2}, v_{2}\right) \in \mathcal{Q}_{S_{2}}^{(1)}\right\},\\
&U_{5}=\left\{u_{5} \in\{0,1\}^{n}:\left(5, u_{5}, v_{5}\right) \in \mathcal{Q}_{S_{5}}^{(1)}\right\}, \quad V_{5}=\left\{v_{5} \in\{0,1\}^{n}:\left(5, u_{5}, v_{5}\right) \in \mathcal{Q}_{S_{5}}^{(1)}\right\}\\

&\alpha_{2} \stackrel{\text { def }}{=} |\left\{(t, a, b) \in \mathcal{Q}_{C}: \left(a \oplus k_{1} \oplus t\right)[i] \in U_{2} \text { for some } i \in\{1, \ldots, w\}\right\} |\\
&\alpha_{5} \stackrel{\text { def }}{=} |\left\{(t, a, b) \in \mathcal{Q}_{C}: \left(b \oplus k_{5} \oplus t\right)[i] \in V_{5} \text { for some } i \in\{1, \ldots, w\}\right\} |\\



  \item\etwo
  there exists $(t, a, b) \in \mathcal{Q}_{C}, \left(u_{2}, v_{2}\right) \in \mathcal{Q}_{S_{2}}^{(1)}, \left(u_{5}, v_{5}\right) \in \mathcal{Q}_{S_{5}}^{1}$, and index $i, j \in \{1, \ldots, w\}$ such that $\left(a \oplus k_{1} \oplus t\right)[i]=u_2$ and$\left(b \oplus k_{5} \oplus t\right)[j]=v_5$
  
  
  
  
  
  
  
  
  
  
  
  
  
  
  
  
  
  \section{beyond birthday bound for tweakable linear SPNs}
\label{section:beyond birthday bound for tweakable linear SPNs}

We also explore conditions under which 6-round, tweakable linear SPNs are secure. A 6-round SPN has seven round permutations $\{\pi_i\}_{i=0}^6$, and without loss of generality we may assume

$$
\pi_{i}\left(k_{i}, t, x\right)=\left\{\begin{array}{ll}
{x \oplus k_{i} \oplus t} & {i \in\{0,6\}} \\
{T_{i} \cdot\left(x \oplus k_{i} \oplus t \right)} & {i \in\{1,2,3,4,5\}}
\end{array}\right.
$$

where $T_{1}, T_{2}, T_{3} T_{4} T_{5}\in \mathbb{F}^{w \times w}$ are invertible linear transformations. We prove that a 6-round, linear SPN is secure so long as (i) $T_1$, $T_2$ $T_3$and $T_{3}^{-1}$, $T_{4}^{-1}$ $T_{5}^{-1}$ contain no zero entries(Miles and Viola ~\cite{miles2015substitution} show that matrices with maximal branch number ~\cite{daemen1995cipher} satisfy this property), and (ii) round keys $\{k_i\}(i=0, 1, 2, 3, 4, 5, 6)$ are (individually) uniform.

We will prove the following theorem firstly.\\

\noindent
\textbf{Theorem 2}. Assume $w>1$, Let $\mathcal{C}$ be a 6-round, tweakable linear SPN with round permutations as in subsection 2.1 showed and with distribution $\mathcal{K}$ over keys $k_{0}, k_{1}, k_{2}, k_{3}, k_{4}, k_{5}, k_{6}$. If round keys $\{k_i\}(i=0, 1, 2, 3, 4, 5, 6)$ are (individually) uniform and $T_1$ and $T_{5}^{-1}$ contain no zero entries, then for any integers $\mathit{p}$ and $\mathit{q}$ such that $p+wq \leq \frac{2^n}{2}$, one has

\begin{equation}
\begin{aligned}
\operatorname{Adv}_{\mathcal{C}}\left(p, q\right) &\leq \frac{w^2q(p+wq)(38p+34wq+16q)+4w^2q(p+3wq)^2}{2^{2n}}\\
& + \frac{32 w^2 q^2 (p+w q)}{2^{3n}}+\frac{q^2}{2^{nw}} + \frac{2 w^2 q + 4 w^2 q^2}{N}.
\end{aligned}
\end{equation}


\noindent \textbf{Outline of Proof of Theorem 2}. Throughout the proof, we will write a 6-round SP construction as $\operatorname{SP}^T_{k}[\mathcal{S}](x)$, where $\mathcal{S}=(S_1, S_2, S_3, S_4, S_5, S_6)$  is a pair of six public random permutations of $\{0,1\}^{n}$, and $k = (k_{0}, k_{1}, k_{2}, k_{3}, k_{4}, k_{6}, k_{7}) \in \mathcal{K}^{7}$ is the key, $x \in \{0,1\}^{w n}$ is the plaintext. and, for i = 1, 2, 3, 4,

$$
\begin{array}{c}
{S_{i}^{\|}:\{0,1\}^{w n} \rightarrow\{0,1\}^{w n}} \\
{x=x_{1}\left\|x_{2}\right\| \ldots\left\|x_{w} \longmapsto S_{i}\left(x_{1}\right)\right\| S_{i}\left(x_{2}\right)\|\ldots\| S_{i}\left(x_{w}\right)}.
\end{array}
$$

We also fix a distinguisher $\mathcal{D}$ as described in the statement and fix an attainable transcript $\tau =\left(\mathcal{Q}_{C}, \mathcal{Q}_{S}\right)$ obtained $\mathcal{D}$. Let

$$
\begin{aligned}
&\mathcal{Q}_{S_{1}}^{(0)}=\left\{(u, v) \in\{0,1\}^{n} \times\{0,1\}^{n}:(1, u, v) \in \mathcal{Q}_{S} \right\},\\
&\mathcal{Q}_{S_{2}}^{(0)}=\left\{(u, v) \in\{0,1\}^{n} \times\{0,1\}^{n}:(2, u, v) \in \mathcal{Q}_{S} \right\},\\
&\mathcal{Q}_{S_{3}}^{(0)}=\left\{(u, v) \in\{0,1\}^{n} \times\{0,1\}^{n}:(3, u, v) \in \mathcal{Q}_{S} \right\},\\
&\mathcal{Q}_{S_{4}}^{(0)}=\left\{(u, v) \in\{0,1\}^{n} \times\{0,1\}^{n}:(4, u, v) \in \mathcal{Q}_{S} \right\}
&\mathcal{Q}_{S_{5}}^{(0)}=\left\{(u, v) \in\{0,1\}^{n} \times\{0,1\}^{n}:(5, u, v) \in \mathcal{Q}_{S} \right\}
&\mathcal{Q}_{S_{6}}^{(0)}=\left\{(u, v) \in\{0,1\}^{n} \times\{0,1\}^{n}:(6, u, v) \in \mathcal{Q}_{S} \right\}
\end{aligned}
$$

\noindent and let

$$
\begin{aligned}
&U_{1}^{(0)}=\left\{u_{1} \in\{0,1\}^{n}:\left(1, u_{1}, v_{1}\right) \in \mathcal{Q}_{S_{1}}^{(0)}\right\}, \quad V_{1}^{(0)}=\left\{v_{1} \in\{0,1\}^{n}:\left(1, u_{1}, v_{1}\right) \in \mathcal{Q}_{S_{1}}^{(0)}\right\},\\
&U_{2}^{(0)}=\left\{u_{2} \in\{0,1\}^{n}:\left(2, u_{2}, v_{2}\right) \in \mathcal{Q}_{S_{2}}^{(0)}\right\}, \quad V_{2}^{(0)}=\left\{v_{2} \in\{0,1\}^{n}:\left(2, u_{2}, v_{2}\right) \in \mathcal{Q}_{S_{2}}^{(0)}\right\},\\
&U_{3}^{(0)}=\left\{u_{3} \in\{0,1\}^{n}:\left(3, u_{3}, v_{3}\right) \in \mathcal{Q}_{S_{3}}^{(0)}\right\}, \quad V_{3}^{(0)}=\left\{v_{3} \in\{0,1\}^{n}:\left(3, u_{3}, v_{3}\right) \in \mathcal{Q}_{S_{3}}^{(0)}\right\},\\
&U_{4}^{(0)}=\left\{u_{4} \in\{0,1\}^{n}:\left(4, u_{4}, v_{4}\right) \in \mathcal{Q}_{S_{4}}^{(0)}\right\}, \quad V_{4}^{(0)}=\left\{v_{4} \in\{0,1\}^{n}:\left(4, u_{4}, v_{4}\right) \in \mathcal{Q}_{S_{4}}^{(0)}\right\},\\
&U_{5}^{(0)}=\left\{u_{5} \in\{0,1\}^{n}:\left(5, u_{5}, v_{5}\right) \in \mathcal{Q}_{S_{5}}^{(0)}\right\}, \quad V_{5}^{(0)}=\left\{v_{5} \in\{0,1\}^{n}:\left(5, u_{5}, v_{5}\right) \in \mathcal{Q}_{S_{5}}^{(0)}\right\},\\
&U_{6}^{(0)}=\left\{u_{6} \in\{0,1\}^{n}:\left(6, u_{6}, v_{6}\right) \in \mathcal{Q}_{S_{6}}^{(0)}\right\}, \quad V_{6}^{(0)}=\left\{v_{6} \in\{0,1\}^{n}:\left(6, u_{6}, v_{6}\right) \in \mathcal{Q}_{S_{6}}^{(0)}\right\}
\end{aligned}
$$

\noindent denote the domains and ranges of $\mathcal{Q}_{S_{1}}^{(0)}, \mathcal{Q}_{S_{2}}^{(0)}, \mathcal{Q}_{S_{3}}^{(0)}, \mathcal{Q}_{S_{4}}^{(0)}, \mathcal{Q}_{S_{5}}^{(0)}, \mathcal{Q}_{S_{6}}^{(0)}$, respectively.

Similar to the 4-round SPN, we will first define what we mean by an extension of the transcript $\tau$. Then we will extension the outer four rounds and the inner two rounds transcripts respectively. Next, we will define bad transcripts, which is the most important step in our proof. Finally, we will peel off he outer two rounds(for the outer four round, we will peel off the outermost two round first) and analyzing the inner two rounds. We stress that extended transcripts are completely virtual and are not disclosed to the adversary. They are just an artificial intermediate step to lower bound the probability to observe transcript $\tau$ in the real world.\\

\noindent
\textsc{EXTENSION OF A TRANSCRIPT(OUTER FOUR ROUNDS)}. We will extend the transcript $\tau$ of the attack via a certain randomized process. We begin with choosing a pair of keys $\left(k_{0}, k_{6}\right) \in \mathcal{K}^{2}$ uniformly at random. Once these keys have been chosen, some construction queries will become involved in collisions. Then a colliding query is defined as a construction query $(t, x, y) \in \mathcal{Q}_{C}$ such that one of the following conditions holds:

\begin{itemize}
  \item[1.]
  there exist an S-box query $(1, u, v) \in \mathcal{Q}_{S}$ and an integer $i \in\{1, \ldots, w\}$ such that $\left(x \oplus k_{0} \opus t \right)[i]=u$.
  \item[2.]
  there exist an S-box query $(6, u, v) \in \mathcal{Q}_{S}$ and an integer $i \in\{1, \ldots, w\}$ such that $\left(y \oplus k_{6} \oplus t \right)[i]=v$.
  \item[3.]
  there exist a construction query $\left(t', x^{\prime}, y^{\prime}\right) \in \mathcal{Q}_{C}$ and an integer $i,j \in\{1, \ldots, w\}$ such that $(t, x, y, i) \neq\left(t', x^{\prime}, y^{\prime}, j\right)$ and $\left(x \oplus k_{0} \oplus t \right)[i] = \left(x' \oplus k_{0} \oplus t \right)[j]$.
  \item[4.]
  there exist a construction query $\left(t', x^{\prime}, y^{\prime}\right) \in \mathcal{Q}_{C}$ and an integer $i,j \in\{1, \ldots, w\}$ such that $(t, x, y, i) \neq\left(t, x^{\prime}, y^{\prime}, j\right)$ and $i \in\{1, \ldots, w\}$ such that $\left(y \oplus k_{6} \oplus t \right)[i] = \left(y' \oplus k_{6} \oplus t\right)[j]$.
\end{itemize}

We are now going to build a new set $\mathcal{Q}_{S_{1}}^{\prime}$ of S-box evaluations that will play the role of an extension of $\mathcal{Q}_{S}$. For each colliding query $(t, x, y) \in \mathcal{Q}_{C}$, we will add tuples $\left(1, \left(x \oplus k_{0} \oplus t\right)[i], v_1^{\prime}\right)_{1 \leq i \leq w}$ (if ($t, \mathit{x}$, $\mathit{y}$) collides at the input of $S_1$), or $\left(6, u_6^{\prime}, \left(y \oplus k_{6} \oplus t\right)[i]\right)_{1 \leq i \leq w}$ (if ($t, \mathit{x}$, $\mathit{y}$) collides at the output of $S_6$), by lazy sampling $v_1^{\prime}=S_{1}(\left(x \oplus k_{0} \oplus t\right)[i])$, or $u_6^{\prime}=S_{6}^{-1}(\left(y \oplus k_{6} \oplus t\right)[i])$, as long as it has not been determined by any existing query in $\mathcal{Q}_{S}$. Then we choose the key $k_1, k_2, k_3, k_4, k_5$ uniformly at random. An extended transcript of $\tau$ will be defined as a tuple $\tau^{\prime}=\left(\mathcal{Q}_{C}, \mathcal{Q}_{S}, \mathcal{Q}_{S_{1}}^{\prime}, \mathbf{k}\right)$ where $\mathbf{k}=\left(k_{0}, k_{1}, k_{2},k_{3},k_{4}, k_{5}, k_{6}\right)$. For each collision between a construction query and a primitive query, or between two construction queries, the extended transcript will contain enough information to compute a complete round of the evaluation of the SPN. This will be useful to lower bound the probability to get the transcript $\tau$ in the real world.\\

\noindent Let

$$
\begin{aligned}
&\mathcal{Q}_{S_{1}}^{(1)}=\left\{(u, v) \in\{0,1\}^{n} \times\{0,1\}^{n}:(1, u, v) \in \mathcal{Q}_{S} \cup \mathcal{Q}_{S_{1}}^{\prime}\right\}\\
&\mathcal{Q}_{S_{6}}^{(1)}=\left\{(u, v) \in\{0,1\}^{n} \times\{0,1\}^{n}:(6, u, v) \in \mathcal{Q}_{S} \cup \mathcal{Q}_{S_{1}}^{\prime}\right\}\\
\end{aligned}
$$

\noindent In words, $\mathcal{Q}_{S_{i}}^{(1)}$ summarizes each constraint that is forced on $S_{i}$ by $\mathcal{Q}_{S}$ and $\mathcal{Q}_{S_{1}}^{\prime}$. Let

$$
\begin{aligned}
&U_{1}=\left\{u_{1} \in\{0,1\}^{n}:\left(1, u_{1}, v_{1}\right) \in \mathcal{Q}_{S_{1}}^{(1)}\right\}, \quad V_{1}=\left\{v_{1} \in\{0,1\}^{n}:\left(1, u_{1}, v_{1}\right) \in \mathcal{Q}_{S_{1}}^{(1)}\right\},\\
&U_{6}=\left\{u_{6} \in\{0,1\}^{n}:\left(6, u_{6}, v_{6}\right) \in \mathcal{Q}_{S_{6}}^{(1)}\right\}, \quad V_{6}=\left\{v_{6} \in\{0,1\}^{n}:\left(6, u_{6}, v_{6}\right) \in \mathcal{Q}_{S_{6}}^{(1)}\right\}\\
\end{aligned}
$$

\noindent be the domains and ranges of $\mathcal{Q}_{S_{1}}^{(1)}$, $\mathcal{Q}_{S_{6}}^{(1)}$ respectively. We define two quantities characterizing an extended transcript $\tau^{\prime}$, namely

$$
\begin{aligned}
&\alpha_{1} \stackrel{\text { def }}{=} |\left\{(t, x, y) \in \mathcal{Q}_{C}: \left(x \oplus k_{0} \oplus t\right)[i] \in U_{1} \text { for some } i \in\{1, \ldots, w\}\right\} |\\
&\alpha_{6} \stackrel{\text { def }}{=} |\left\{(t, x, y) \in \mathcal{Q}_{C}: \left(y \oplus k_{6} \oplus t\right)[i] \in V_{6} \text { for some } i \in\{1, \ldots, w\}\right\} |
\end{aligned}
$$

In words, $\alpha_1$ (resp. $\alpha_6$) is the number of queries $(t, x, y) \in \mathcal{Q}_{C}$ which collide with a query $\left(u_{1}, v_{1}\right) \in \mathcal{Q}_{S_{1}}^{(1)}$ (resp. $\left(u_{6}, v_{6}\right) \in \mathcal{Q}_{S_{6}}^{(1)}$) in the extended transcript. This corresponds to the number of queries $(t, x, y) \in \mathcal{Q}_{C}$ which collide with either an original query $\left(u_{1}, v_{1}\right) \in \mathcal{Q}_{S_{1}}^{(0)}$ (resp. which collide with a query $\left(u_{6}, v_{6}\right) \in \mathcal{Q}_{S_{6}}^{(0)}$) or with a query $\left(t' x^{\prime}, y^{\prime}\right) \in \mathcal{Q}_{C}$ at an input of $S_1$ (resp. at the output of $S_6$), once the choice of $\left(k_{0}, k_{6}\right)$  has been made. We will also denote

$$
\beta_{i}=\left|\mathcal{Q}_{S_{i}}^{(1)}\right|-\left|\mathcal{Q}_{S_{i}}^{(0)}\right|=\left|\mathcal{Q}_{S_{i}}^{(1)}\right|-p.
$$

for $i=1, 6$ the number of additional queries included in the extended transcript.\\

\subsection{Bad Transcript for 6-rounds tweakable linear SPN and Probability}

\noindent We say an extended transcript $\tau^{\prime}$ is bad if at least one of the following conditions is fulfilled:

\begin{itemize}
  \item\eone
  there exists $(t, x, y) \in \mathcal{Q}_{C}, \left(u_{1}, v_{1}\right) \in \mathcal{Q}_{S_{1}}^{(1)}, \left(u_{6}, v_{6}\right) \in \mathcal{Q}_{S_{6}}^{(1)}$, and index $i, j \in \{1, \ldots, w\}$ such that $\left(x \oplus k_{0} \oplus t\right)[i]=u_1$ and $\left(y \oplus k_{6} \oplus t\right)[j]=v_6$.
  \item\etwo
  there exists $(t, x, y) \in \mathcal{Q}_{C}, \left(u_{1}, v_{1}\right) \in \mathcal{Q}_{S_{1}}^{(1)}, \left(u_{2}, v_{2}\right) \in \mathcal{Q}_{S_{2}}$, and index $i, j \in \{1, \ldots, w\}$ such that $\left(x \oplus k_{0} \oplus t\right)[i]=u_1$ and $\left(T_{1}\left(S_{1}\left(x \oplus k_{0} \oplus t\right) \oplus k_{1} \oplus t\right)\right)[j]=u_2$.
  \item\ethree
  there exists $(t, x, y) \in \mathcal{Q}_{C}, \left(u_{5}, v_{5}\right) \in \mathcal{Q}_{S_{5}}, \left(u_{6}, v_{6}\right) \in \mathcal{Q}_{S_{6}}^{(1)}$, and index $i, j \in \{1, \ldots, w\}$ such that $\left(y \oplus k_{6} \oplus t\right)[j]=v_6$ and $\left(T_{5}^{-1}\left(S_{6}^{-1}\left(y \oplus k_{6} \oplus t\right) \oplus k_{5} \oplus t\right)\right)[i]=v_5$.
  \item\efour 
  there exists $(t, x, y) \in \mathcal{Q}_{C}$ and distinct indices $i, j \in \{1, \ldots, w\}$ such that $(x\xor k_0 \xor t)[i]=(x\xor k_0 \xor t)[j]$, or $(y\xor k_6 \xor t)[i]=(y\xor k_6 \xor t)[j]$.
  \item\efive
  there exist $(t, x, y), (t', x', y') \in \mathcal{Q}_{C}$ and distinct indices $i, j \in \{1, \ldots, w\}$ such that (x\xor k_0 \xor t)[i]=(x'\xor k_0 \xor t')[j]$, or $(y\xor k_6 \xor t)[i]=(y'\xor k_6 \xor t')[j]$.
\end{itemize}

\noindent \textbf{Lemma 10} \emph{One has}

\begin{equation}
\operatorname{Pr}[\tau^{\prime} \in \Theta_{bad}(\tau)] \leq \frac{w^2 q (p+w q) (3 p +w q)}{N^{2}} + \frac{w^{2} q + 2 w^2 q^2}{N}.
\end{equation}

\noindent \emph{Proof:} We fix any extended transcript, denoted $\left(\mathcal{Q}_{C}, \mathcal{Q}_{S}, \mathcal{Q}_{S_{1}}^{\prime}\right)$. For any fixed construction query $(t, x, y) \in \mathcal{Q}_{C}$, now we upper bound the probabilities of the bad extended transcript.\\

\noindent Consider \eone first: Since we have at most $w^{2} q \left(p+w q\right)^{2}$ choices for $(t, x, y) \in \mathcal{Q}_{C}, \left(u_{1}, v_{1}\right) \in \mathcal{Q}_{S_{1}}^{(1)}, \left(u_{6}, v_{6}\right) \in \mathcal{Q}_{S_{6}}^{(1)}$ and index $i, j \in \{1, \ldots, w\}$ and since the random choice of $k_{0}$ and $k_{6}$ are independent, one has

$$
\operatorname{Pr}\left[\eone\right] \leq \frac{w^{2} q \left(p+w q\right)^{2}}{N^{2}}.
$$

Similarly, since $k_{0}$ and $k_{1}$ are random and independent, and we have at most $w^{2} q p \left(p+w q\right)$ for $(t, x, y) \in \mathcal{Q}_{C}, \left(u_{1}, v_{1}\right) \in \mathcal{Q}_{S_{1}}^{(1)}, \left(u_{2}, v_{2}\right) \in \mathcal{Q}_{S_{2}}$ and index $i, j \in \{1, \ldots, w\}$, we have $\operatorname{Pr}\left[\etwo\right] \leq \frac{w^{2} q p \left(p+w q\right)}{N^{2}}$; by symmetry, $\operatorname{Pr}\left[\ethree\right] \leq \frac{w^{2} q p \left(p+w q\right)}{N^{2}}$. \\

\noindent Then consider \bfour. We assume that $w \neq 2$, because of $w = 1$ does not belong to the primary problem of the SP-networks. Since the random choice of $k_{0}$ and $k_{6}$ are independent, then we have

$$
\operatorname{Pr}\left[\efour\right] \leq \frac{w^{2} q}{N}.
$$

Similarly, we have $\operatorname{Pr}\left[\efive\right] \leq \frac{2 w^{2} q^2}{N}$.

\noindent  Then sum of yields (11).


Similar to the outermost two round, we will extend the inner two round (the two and the five round).  Pick a pair of S-box $(S_1, S_6)$ such that $S_{1} \vdash \mathcal{Q}_{S_{1}}^{(0)}$ and $S_{6} \vdash \mathcal{Q}_{S_{6}}^{(0)}$, and for each $ (t, x, y) \in \mathcal{Q}_{C}$ we set $a=T_{1}\left(S_{1}\left(x \oplus k_{0} \oplus t\right)\right)$, $b=\S_{6}^{-1}\left(y \oplus k_{6} \oplus t\right)$. In this way we obtain $\mathnormal{q}$ tuples of the form $(t, a, b)$; for convenience we denote the set of such induced tuples by $\mathcal{Q}_{C}^{*}\left(S_{1}, S_{6}\right)$. Then we choose a pair of keys $\left(k_{1}, k_{5}\right) \in \mathcal{K}^{2}$ uniformly at random. Once these keys have been chosen, some construction queries will become involved in collisions. A colliding query is defined as a construction query $(t, a, b) \in \mathcal{Q}_{C}^{*}\left(S_{1}, S_{6}\right)$. After that, we build a new set $\mathcal{Q}_{S_{2}}^{\prime}$ of S-box evaluations that will play the role of an extension of $\mathcal{Q}_{S}$. Then we choose the key $k_2, k_3, k_4$ uniformly at random. An extended transcript of $\tau$ will be defined as a tuple $\tau''=\left(\mathcal{Q}_{C}, \mathcal{Q}_{S}, \mathcal{Q}_{S_{2}}^{\prime}, \mathbf{k}\right)$ where $\mathbf{k}=\left(k_{0}, k_{1}, k_{2},k_{3},k_{4}, k_{5}, k_{6}\right)$. After define the extended transcript $\tau''$ is bad, 

\noindent \textbf{Lemma 11} \emph{One has}

\begin{equation}
\operatorname{Pr}[\tau'' \in \Theta_{bad}(\tau)] \leq \frac{w^2 q (p+w q) (3 p +w q)}{N^{2}} + \frac{w^{2} q + 2 w^2 q^2}{N}.
\end{equation}

\noindent The proof is similar to Lemma 10.

Then combine (11), (12), we can get

\begin{equation}
\operatorname{Pr}[\tau \in \Theta_{bad}(\tau)] \leq \frac{2w^2 q (p+w q) (3 p +w q)}{N^{2}} + \frac{2w^{2} q + 4 w^2 q^2}{N}.
\end{equation}


\subsection{Analysis for Good Transcript}

\noindent Fix a good transcript and a good round-key vector $\mathnormal{k}$, we are to derive a lower bound for the probability  $\operatorname{Pr}\left[\mathcal{S} \stackrel{\mathbf{s}}{\leftarrow}(\mathcal{S}(n))^{6}: \mathrm{SP}_{k}[\mathcal{S}] \vdash \mathcal{Q}_{C} | \mathcal{S} \vdash \mathcal{Q}_{S}\right]$. We ``peel off'' the outer four rounds. Then assuming $(S_{1}, S_2, S_{5}, S_6)$ is good, we analyze the induced 2-round transcript to yield the final bounds.\\

\noindent
\textsc{PEELING OFF THE OUTER FOUR ROUNDS}. Pick a pair of S-box $(S_1, S_2, S_{5}, S_6)$ such that $S_{1} \vdash \mathcal{Q}_{S_{1}}^{(0)}$, S_{2} \vdash \mathcal{Q}_{S_{2}}^{(0)}$, S_{5} \vdash \mathcal{Q}_{S_{5}}^{(0)}$   and $S_{6} \vdash \mathcal{Q}_{S_{6}}^{(0)}$, and for each $ (t, a, b) \in \mathcal{Q}_{C}^{*}\left(S_{1}, S_{6}\right)$ we set $c=T_{2}\left(S_{2}\left(a \oplus k_{1} \oplus t\right)\right)$, $d=S_{5}^{-1}\left(b \oplus k_{5} \oplus t\right)$. In this way we obtain $\mathnormal{q}$ tuples of the form $(c, d)$; for convenience we denote the set of such induced tuples by $\mathcal{Q}_{C}^{**}\left(S_{2}, S_{5}\right)$. Similarly, we also extended the innermost two rounds:\\

Then we build a new set $\mathcal{Q}_{S_{inner}}^{\prime}$ of S-box evaluations that will play the role of an extension of $\mathcal{Q}_{C}^{**}\left(S_{2}, S_{5}\right)$. Then we choose the key $k_3$ uniformly at random. An extended transcript of $\tau_{inner}$ will be defined as a tuple $\tau_{inner}^{\prime}=\left(\mathcal{Q}_{C}^{**}\left(S_{2}, S_{5}\right), \mathcal{Q}_{S_{inner}}, \mathcal{Q}_{S_{inner}}^{\prime}, \mathbf{k}\right)$ where $\mathbf{k}=\left(k_{2}, k_{3}, k_{4}\right)$.

\noindent \textbf{Lemma 12} \emph{ For any extended $S_{1} \vdash \mathcal{Q}_{S_{1}}, S_{2} \vdash \mathcal{Q}_{S_{2}}, S_{5} \vdash \mathcal{Q}_{S_{5}}, S_{6} \vdash \mathcal{Q}_{S_{6}}$, we have}

\begin{equation}
\begin{aligned}
1-&\operatorname{Pr}\left[\operatorname{Bad}\left(S_{1}, S_2, S_{5}, S_6\right) | S_{1} \vdash \mathcal{Q}_{S_{1}}, S_{2} \vdash \mathcal{Q}_{S_{2}}, S_{5} \vdash \mathcal{Q}_{S_{5}}, S_{6} \vdash \mathcal{Q}_{S_{6}}\right] \geq 1- \frac{w^{2} q (p+w q)^{2}}{(N-p)^{2}}\\
& -\frac{2 w^{2} q (p+w q)^{2}}{N \cdot (N-p)} - \frac{2 w^{2} q (p+w q)^{2}}{(N-p)^2} - \frac{2 w^{2} q^{2} (p+w q)}{(N-p)^2} - \frac{2 w^{2} q^{2} (p+w q)}{(N- p- wq) \cdot (N-p)^2}.
\end{aligned}
\end{equation}

\noindent \emph{Proof:} Then we define a predicate $\operatorname{Bad}\left(S_{1}, S_2, S_{5}, S_6\right)$ on the pair $(S_1,S_2, S_5, S_6)$, which holds if the corresponding induced set $\mathcal{Q}_{C}^{**}\left(S_{2}, S_{5}\right)$ fulfills at least one of the following seven ``collision'' conditions:

\begin{itemize}
  \item[\feai]
  there exist $(t, c, d) \in \mathcal{Q}_{C}^{**}\left(S_{2}, S_{5}\right)$, $i, j \in\{1, \ldots, w\}$, $u_{3} \in U_{3}$ and $v_{4} \in V_{4}$ such that $\left(c \oplus k_{2} \oplus t\right)[i] = u_3$ and $\left(T_{4}^{-1}\left(d \oplus k_{4} \oplus t\right)\right)[i] = v_4$.
  \item[\feaii]
  there exist $(t, c, d) \in \mathcal{Q}_{C}^{**}\left(S_{2}, S_{5}\right)$, $i, j \in\{1, \ldots, w\}$, $u_{3} \in U_{3}$ and $u_{4} \in U_{4}$ such that $\left(c \oplus k_{2} \oplus t\right)[i] = u_3$ and $\left(T_{3}\left(S_{3}\left(c \oplus k_{2} \oplus t\right)\right) \oplus k_{3} \oplus t\right)[j] = u_4$.
  \item[\feaiii]
  there exist $(t, c, d) \in \mathcal{Q}_{C}^{**}\left(S_{2}, S_{5}\right)$, $i, j \in\{1, \ldots, w\}$, $v_{3} \in V_{3}$ and $v_{4} \in V_{4}$ such that $\left(T_{4}^{-1}\left(d \oplus k_{4} \oplus t\right)\right)[i] = v_4$ and $\left(T_{3}^{-1}\left(S_{4}^{-1}\left(T_{4}^{-1}\left(d \oplus k_{4} \oplus t\right)\right) \oplus k_{3} \oplus t\right)\right)[j] = v_3$.
  \item[\feaiv]
  there exist $(t, c, d) \in \mathcal{Q}_{C}^{**}\left(S_{2}, S_{5}\right)$, distinct $i, i^{\prime}\in\{1, \ldots, w\}$, $u_{3},u_{3}' \in U_{3}$ such that
  $$\left(c \oplus k_{2} \oplus t\right)[i] = u_3,\text{ and }
  \left(c \oplus k_{2} \oplus t\right)[i']= u_3'.$$
  \item[\feav]
  there exist distinct $(t, c, d),(t', c', d') \in \mathcal{Q}_{C}^{**}\left(S_{2}, S_{5}\right)$, distinct $i, i^{\prime}\in\{1, \ldots, w\}$, $u_{3} \in U_{3}$ such that
  $$\left(c \oplus k_{2} \oplus t\right)[i] = u_3,\text{ and }
  \left(c \oplus k_{2} \oplus t\right)[i']=\left(c' \oplus k_{2} \oplus t'\right)[i'].$$
  \item[\feavi]
  there exist $(t, c, d), (t', c^{\prime}, d^{\prime}) \in \mathcal{Q}_{C}^{**}\left(S_{2}, S_{5}\right)$, $i, i^{\prime},j, j^{\prime} \in\{1, \ldots, w\}$, with$(t, c, j) \neq \left(t', c^{\prime}, j^{\prime}\right)$, $u_{3}, u_{3}^{\prime} \in U_{3}$ such that $\left(c \oplus k_{2} \oplus t\right)[i] = u_3, \left(c^{\prime} \oplus k_{2} \oplus t'\right)[i^{\prime}] = u_3^{\prime}$ and
$$
  \left(T_{3}\left(S_{3}\left(c \oplus k_{2} \oplus t\right)\right) \oplus k_{3} \oplus t\right)[j] = \left(T_{3}\left(S_{3}\left(c^{\prime} \oplus k_{2} \oplus t'\right)\right) \oplus k_{3} \oplus t'\right)[j^{\prime}].
$$
  \item[\feavii]
  there exist $(t, c, d) \in \mathcal{Q}_{C}^{**}\left(S_{2}, S_{5}\right)$, distinct $i, i^{\prime}\in\{1, \ldots, w\}$, $v_{4},v_{4}' \in V_{4}$ such that
  $$\left(T_{3}^{-1}\left(S_{4}^{-1}\left(T_{4}^{-1}\left(d \oplus k_{4} \oplus t\right)\right) \oplus k_{3} \oplus t\right)\right)[j] = v_3,\text{ and }
  \left(T_{3}^{-1}\left(S_{4}^{-1}\left(T_{4}^{-1}\left(d \oplus k_{4} \oplus t\right)\right) \oplus k_{3} \oplus t\right)\right)[j'] = v_3'.$$
  \item[\feaviii]
  there exist distinct $(t, c, d),(t', c', d') \in \mathcal{Q}_{C}^{**}\left(S_{2}, S_{5}\right)$, distinct $i, i^{\prime}\in\{1, \ldots, w\}$, $v_{4} \in V_{4}$ such that
  $$\left(T_{3}^{-1}\left(S_{4}^{-1}\left(T_{4}^{-1}\left(d \oplus k_{4} \oplus t\right)\right) \oplus k_{3} \oplus t\right)\right)[j] = v_3,\text{ and }
\left(T_{3}^{-1}\left(S_{4}^{-1}\left(T_{4}^{-1}\left(d \oplus k_{4} \oplus t\right)\right) \oplus k_{3} \oplus t\right)\right)[j]=\left(T_{3}^{-1}\left(S_{4}^{-1}\left(T_{4}^{-1}\left(d' \oplus k_{4} \oplus t'\right)\right) \oplus k_{3} \oplus t'\right)\right)[j'].$$
  \item[\feaviiii]
  there exist $(t, c, d), (t', c', d') \in \mathcal{Q}_{C}^{**}\left(S_{2}, S_{5}\right)$, $i, i^{\prime}, j, j^{\prime} \in\{1, \ldots, w\}$,$u_{3} \in U_{3}$ with$(t, d, j) \neq \left(t', d^{\prime}, j^{\prime}\right)$, $v_{4},v_{4}^{\prime} \in V_{4}$ such that $\left(T_{4}^{-1}\left(d \oplus k_{4} \oplus t\right)\right)[i] = v_4, \left(T_{4}^{-1}\left(d' \oplus k_{4} \oplus t'\right)\right)[i'] = v_4^{\prime}$ and
$$
  \left(T_{3}^{-1}\left(S_{4}^{-1}\left(T_{4}^{-1}\left(d \oplus k_{4} \oplus t\right)\right) \oplus k_{3} \oplus t\right)\right)[j] =  \left(T_{3}^{-1}\left(S_{4}^{-1}\left(T_{4}^{-1}\left(d' \oplus k_{4} \oplus t'\right)\right) \oplus k_{3} \oplus t'\right)\right)[j'].
$$
\end{itemize}

\noindent The other proof is similar to Lemma 8. Then combine all of these lemma, we can get the Theorem 2



