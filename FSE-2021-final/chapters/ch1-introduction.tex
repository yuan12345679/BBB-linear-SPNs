
\section{Introduction}
\label{section:Introduction}

Modern blockciphers roughly fall into two classes (with some rare exceptions such as IDEA~\cite{EC:LaiMas90} and KATAN~\cite{CHES:DeCDunKne09}), namely {\it Feistel networks and their generalizations}, and {\it substitution-permutation networks} (SPNs). A Feistel round applies a domain-preserving function on half of the data, and then executes XOR and swap operations. This paradigm may be generalized to using compression functions, expansion functions, and smaller functions. Popular examples include many blockcipher standards such as DES~\cite{DESDesign}, GOST~\cite{GOSTDesign}, and Camellia~\cite{ISOIEC-18033-3:2010}. On the other hand, the latter paradigm SPNs start with a set of public permutations on the set of $n$-bit strings which are called S-boxes. These public permutations are then extended to a keyed permutation on $wn$-bit inputs for some integer $w$ by iterating the following steps:
\begin{enumerate}
	\item[1.] {\it Substitution step}: break down the $wn$-bit state into $w$ disjoint chunks of $n$ bits, and evaluate an S-box on each chunk;
	\item[2.] {\it Permutation step}: apply a non-cryptographic, keyed permutation to the whole $wn$-bit state (which is also applied to the plaintext before the first round).
\end{enumerate}
%
S-boxes are typically highly non-linear, and, in fact, serve as the only source of non-linearity in many blockciphers. There is no a priori restriction on the (non-)linearity of the {\it Permutation step}, and the use and advantages of non-linear permutations was recently explored~\cite{DBLP:journals/dcc/LiuRL18}. Though, modern blockciphers still tend to use linear or affine mappings for the {\it Permutation step}~\cite{DBLP:reference/crypt/Biryukov11aa}, which involves a simple key-mixing step followed
by an invertible linear or affine transformation. More precisely, their permutation steps are {\it linear} or affine with respect to additions on $\text{GF}(2^n)$, where $n$ is the size of the S-box. Various popular blockciphers including the AES~\cite{AESDesign}, Serpent~\cite{serpentProposal}, and the ISO/IEC lightweight standard PRESENT~\cite{CHES:BKLPPR07} follow this approach. Furthermore, a subset of them using maximum distance separable linear transformations allows for effective provable security against certain types of attacks~\cite{IMA:DaeRij01,AC:PSCYL02,FSE:PSLL03,miles2015substitution,EC:SLGRL16}.
%On the other hand, the {\it Permutation step} could also be {\it non-linear}, on which we will elaborate later.


%From a theoretical point of view, both of the two approaches revolve around the extension of a ``complex'' function or permutation on a small domain to a keyed pseudorandom permutation on a larger domain by iterating several times simple rounds.

%SPNs extend domain more efficiently than Feistel networks, in the sense of $wn$-bit SPN cipher versus the $2n$-bit.


The traditional security notion for blockciphers is (strong) pseudorandomness: for any adversary with reasonable resources, the blockcipher with {\it a random and secret key} should be indistinguishable from a truly random
permutation. Proving such security for concrete blockciphers such as AES
seems out of the reach of current techniques. The usual approach is to idealize some underlying primitives and prove that the high-level structure is sound, in the sense of being a strong pseudorandom permutation (SPRP) or others. Typically, to prove security for Feistel networks, the Feistel round functions are idealized, resulting in schemes such as the seminal Luby-Rackoff model~\cite{DBLP:journals/siamcomp/LubyR88,EC:MauPie03,C:patarin03,C:patarin04,C:HoaRog10,JC:CHKPST16}. To prove security for SPNs, the ``S-boxes'' may be idealized as secret random functions or permutations, leaving the permutation layers as efficient ``non-cryptographic'' functions~\cite{FSE:IwaKur00,miles2015substitution}. In this case, the S-boxes act as the only source of cryptographic hardness. This limits the provable security to the domain-size of the S-boxes, which is unfortunately as small as 8 bits in, e.g., the AES. Consequently, provable results on SPNs do not relate to any concrete SPN-based block ciphers. Instead, they should be viewed as theoretical support for the SPN approach to constructing blockciphers.\footnote{Similar limitation exists in Feistel schemes, though it appears more acceptable, being, e.g., 32 bits in DES.}


Recently, initiated by Dodis et al.~\cite{EC:DSSL16,EPRINT:DKSTZ17}, a series of works investigated a new model of SPNs, in which the S-boxes
are small {\it public} ideal primitives and the permutation layers remain non-cryptographic. In detail, it was~\cite{EC:DSSL16} that for the first time investigated the {\it indifferentiability}~\cite{TCC:MauRenHol04}
of confusion-diffusion networks or keyless SPN models  combining public random S-boxes and non-cryptographic permutation layers. It was also~\cite{EC:DSSL16} that for the first time confirmed (in a widely recognized theoretical model) that, the use of non-linear permutation layers ensures more security than linear ones. The SPRP security of {\it keyed} SPN models has to be deferred to later in~\cite{EPRINT:DKSTZ17,C:CDKLST18}. In detail, regarding the (more common) SPN model with linear permutation layers, Dodis et al.~\cite{EPRINT:DKSTZ17} exhibited a chosen-ciphertext boomerang attack against 2 rounds using only 4 queries. On the positive side, they proved that 3 rounds ensure the classical birthday-bound security, i.e., security up to $2^{n/2}$ adversarial queries, where $n$ is the size of the idealized S-boxes. These characterized its SPRP security. To ensure this birthday-bound security, the linear permutation layers shall satisfy a quite mild condition of ``zero-freeness'', meaning that all entries in the matrix representations of the linear permutation layers and their inverses shall be non-zero.



Regarding the SPN model with non-linear permutation layers, Dodis et al.~\cite{EPRINT:DKSTZ17} identified a combinatorial property on the permutations that suffices for security in this case, named blockwise universality. Informally, a keyed permutation $\pi_k$ is blockwise universal if, for any distinct inputs $x,x'$ and any constant $c$, the probability (taken over uniform $k$) of each of the following events is low: (i) a block of $\pi(k,x)$ is equal to a block of $\pi(k,x')$, (ii) two different blocks of $\pi(k,x)$ are equal, (iii) a block of $\pi(k,x)$ is equal to $c$. Using such non-linear permutations, they showed that even one round is already sufficient for birthday-bound. Later, Cogliati and Lee improved this result by: (i) adding {\it tweaks} into the non-linear transformations
to obtain {\it tweakable non-linear SPNs}, and (ii) proving beyond-birthday-bound results~\cite{EPRINT:CogLee18}. They showed that two rounds of such tweakable non-linear SPNs are secure tweakable blockciphers~\cite{JC:LisRivWag11} up to roughly $2^{2n/3}$ adversarial queries. They also provided a (non-tight) asymptotic security bound improving as the number of rounds grows.


%
%\subsection{Linear vs Non-linear Permutation steps}
%
%Modern blockciphers tend to use linear or affine mappings for the {\it Permutation step}~\cite{DBLP:reference/crypt/Biryukov11aa}, which involves a simple key-mixing step followed
%by an invertible linear or affine transformation. More precisely, their permutation steps are {\it linear} or affine with respect to additions on $GF(2^n)$, where $n$ is the size of the S-box. This actually includes all the aforementioned SPN ciphers. A small subset of them using MDS linear transformations allows for effective provable security against certain types of attacks~\cite{IMA:DaeRij01,AC:PSCYL02,FSE:PSLL03,miles2015substitution,EC:SLGRL16}.
%
%
%
%On the other hand, as noticed by Dodis et al.~\cite{EPRINT:DKSSZZ18} (the idea of which might further date back to~\cite{FSE:ChaSar06,C:Halevi07}), the {\it Permutation step} could actually be {\it non-linear}. As mentioned before, the security of such non-linear SPN models goes beyond the birthday barrier with more than 2 rounds. Though, such models have two shortages. First, ... implementing a blockwise universal permutations might be costly, and linear functions $f_i$'s would be highly preferable for obvious efficiency reasons. More importantly, {\it far from realistic}. In fact, the idea of using non-linear transformations in real blockciphers was only recently investigated by Liu et al.~\cite{DBLP:journals/dcc/LiuRL18}.
%
%
%
%Regarding the classical SPN model with linear permutation layers, Dodis et al. has characterized its SPRP security.
%They exhibited attacks against 2 rounds using only 4 queries, and proved $n/2$ birthday security at 3 rounds.
%




\subsection{Our Results}

%In this paper, we ask whether it is possible to come with a tweakable Even-
%Mansour construction achieving both:
%1. a linear mixing of the tweak and the key to the state;
%2. beyond-birthday-bound security.
%We answer positively, by providing a construction with 2n-bit keys and n-bit tweaks.

As briefed before, with more than two rounds, non-linear SPNs could ensure beyond-birthday-bound security. Though, practitioners prefer linear SPNs, the security of which is only proved up to birthday-bound at 3 rounds.
%
%
%In fact, beyond-birthday-bound security of linear SPNs with 3 or more rounds was left as an open question in~\cite{C:CDKLST18}.
%
Observing this gap, we ask whether it is possible to achieve security beyond the birthday barrier with linear SPN structures. For this, we focus on linear SPNs with {\it independent S-boxes} and {\it independent round keys}, and we will focus on the case where $w\geq2$, since, when $w = 1$, we recover the standard Even-Mansour construction that has already been well investigated (see the related works below). For such linear SPNs, we answer our main question positively and prove the first beyond-birthday-bound (BBB) $2n/3$-bit security result on 4 rounds.


Concretely, we first characterize conditions on the linear layers that are sufficient for $2n/3$-bit security. For a linear transformation $T$ to meet this, it has to be ``zero-free'' in the aforementioned sense. In addition, in both $T$ and $T^{-1}$, the sum of every 2 entries from the same row shall be non-zero. Thus, the conditions are slightly stronger than that for birthday-bound, and may be viewed as a second order extension of the aforementioned ``zero-freeness'' condition.
%
%A careful thinking of common proofs for $2n/3$-bit security results indicate that such conditions are somewhat within the expectations.
%

With this, we show that a 4-round linear SPN is beyond-birthday-bound secure, if: (i) 4 independent public random S-boxes are used in the four rounds respectively, and (ii) such a ``second order zero-free'' linear permutation layer is used in every round, and (iii) the round keys are uniform and independent. Our proof employs the H-coefficient technique~\cite{SAC:Patarin08}. Moreover, we prove the notion of {\it point-wise proximity}~\cite{C:HoaTes16}, thus establishing $2n/3$-bit {\it multi-user security} for 4-round linear SPNs as well.


Our proof crucially relies on a technical lemma of Cogliati and Lee~\cite{EPRINT:CogLee18} on two SPN rounds. In some sense, in our 4-round linear SPNs, the 1st and 4th round play similar role as the so-called blockwise universal permutations in the 2-round non-linear SPNs of Cogliati and Lee. The situation somewhat resembles that of tweakable Even-Mansour ciphers~\cite{C:CogLamSeu15,AC:CogSeu15}. See Section \ref{section:security of 4-round SPNs} for details.




\paragraph{Interpretation.}

We view our result as extending a sound theory for constructing ciphers from small S-boxes and providing additional theoretical support for the SPN approach (particularly for the real world ``linear SPNs''). As mentioned before, the $n$-bit idealize S-boxes are the only cryptographic hardness in the current SPN models with non-cryptographic permutation layers, and this enforces the inherent ``$2^n$ provable barrier''. Neither this $2^n$ bound nor our inferior $2^{2n/3}$ bound (though improved upon $2^{n/2}$ of~\cite{C:CDKLST18}) is meaningful for regular SPN blockciphers, in which very low values of $n$ are typically chosen for the S-boxes. For example, the S-box of the AES is based on the inverse of $\text{GF}(2^8)$, and has $n=8$.
Though, this series of theory results should be viewed as important complementary to the more coarse iterated Even-Mansour model~\cite{EC:BKLSST12}.

On the other hand, as provable security (mostly against differential and linear properties) of the ARX ciphers advances, recent works have put forward practical choices of 11-~\cite{DBLP:journals/tosc/16-bit-Sbox} or even 64-bit~\cite{cryptoeprint:2019:1378} bigger S-boxes. The bound becomes more meaningful with such parameters.




\subsection{Other Related Work}


Here we survey some other related works besides the aforementioned ones on SPNs with {\it public} S-boxes~\cite{EC:DSSL16,EPRINT:DKSTZ17,EPRINT:CogLee18,C:CDKLST18}. First, when $w=1$,
%
\begin{itemize}
	\item Linear SPNs collapse to the iterated Even-Mansour construction, the SPRP security was initiated in~\cite{JC:EveMan97} and subsequently extended to multiple rounds~\cite{EC:BKLSST12,EPRINT:Steinberger12,AC:LamPatSeu12,EC:CheSte14,JC:CLLSS18,C:HoaTes16} and multi-user setting~\cite{C:HoaTes16}. In detail, with $t$ rounds, the $n$-bit iterated Even-Mansour cipher is tightly secure up to $2^{\frac{tn}{t+1}}$ adversarial queries~\cite{EC:BKLSST12,EC:CheSte14,C:HoaTes16};
	\item Non-linear tweakable SPNs collapse to {\it tweakable Even-Mansour ciphers} with non-linear tweaking functions~\cite{C:CogLamSeu15} (with follow-ups such as~\cite{AC:CogSeu15,EC:GJMN16,C:Mennink16}).
\end{itemize}
%
Provable security of the earlier non-linear SPN models with {\it secret, key-dependent S-boxes} were (partly) addressed by Naor and Reingold~\cite{JC:NaoRei99}, Chakraborty and Sarkar~\cite{FSE:ChaSar06}, and Halevi~\cite{C:Halevi07}. Security of linear SPN models with such secret S-boxes were proved by Iwata and Kurosawa~\cite{FSE:IwaKur00}, though for specific permutation layers and birthday-bound security only. Subsequently, Miles and Viola~\cite{miles2015substitution} proved chosen-plaintext security for linear SPNs with PRF S-boxes, ``zero-free'' permutations, and more than 2 rounds.


Finally, on the cryptanalytic side, attacks against SPNs could be found in~\cite{EC:Joux03,RSA:HalRog04,JC:BirSha10,AC:BirBouKho14,cryptoeprint:2015:646,cryptoeprint:2015:646}, while provable security has been addressed by~\cite{IMA:DaeRij01,AC:PSCYL02,FSE:PSLL03,miles2015substitution} against differential/linear cryptanalysis and~\cite{EC:SLGRL16} against others such as impossible differential attacks, etc. In addition, it was shown in~\cite{DBLP:journals/dcc/LiuRL18} that the use of non-linear permutation layers may indeed increase security against differential/linear attacks.


\floatstyle{boxed}
\restylefloat{figure}

